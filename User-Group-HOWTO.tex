\documentclass{HOWTO}

\def\addbibtoc{
\addcontentsline{toc}{section}{\numberline{\mbox{}}\relax\bibname}
}%end-preamble
\setcounter{page}{1}
\urldef{\aaaurl} \url{mailto:rick@linuxmafia.com}
\urldef{\aacurl} \url{http://web.archive.org/web/20050308130348/http://ngc891.blogdns.net/kernel/docs/arch.txt}
\urldef{\aadurl} \url{http://www.linuxfordevices.com/}
\urldef{\aaeurl} \url{http://www.uclinux.org/ports/}
\urldef{\aafurl} \url{http://www.arm.linux.org.uk/}
\urldef{\aagurl} \url{http://www.linuxfordevices.com/c/a/News/Another-nativeDSP-Linux-port-this-one-to-ADIs-Blackfin/}
\urldef{\aahurl} \url{https://en.wikipedia.org/wiki/ETRAX_CRIS}
\urldef{\aaiurl} \url{http://www.freescale.com/files/netcomm/doc/data_sheet/MC68EN302.pdf}
\urldef{\aajurl} \url{http://ecos.sourceware.org/hardware.html#FR-V}
\urldef{\aakurl} \url{http://www.uclinux.org/pub/uClinux/ports/h8/}
\urldef{\aalurl} \url{http://web.archive.org/web/20070626190436/http://www.stmcu.com/forums-cat-132-6.html}
\urldef{\aamurl} \url{http://ecos.sourceware.org/hardware.html#MatsushitaAM3x}
\urldef{\aanurl} \url{http://www.linux-mips.org/wiki/NEC_VR4100}
\urldef{\aaourl} \url{http://ds.dial.pipex.com/town/way/fr30/}
\urldef{\aapurl} \url{http://www.uclinux.org/ports/coldfire/}
\urldef{\aaqurl} \url{http://penguinppc.org/embedded/}
\urldef{\aarurl} \url{http://ecos.sourceware.org/tools/linux-v850-elf.html}
\urldef{\aasurl} \url{http://wiki.debian.org/SH4}
\urldef{\aaturl} \url{http://ecos.sourceware.org/hardware.html#CalmRISC}
\urldef{\aauurl} \url{http://www.linuxfordevices.com/c/a/News/Worlds-first-nativeDSP-Linux-port/}
\urldef{\aavurl} \url{http://www.linuxfordevices.com/c/a/News/Embedded-Linux-distro-supports-TI-DSPbased-digital-media-processors/}
\urldef{\aawurl} \url{http://www.xilinx.com/tools/petalinux-sdk.htm}
\urldef{\aaxurl} \url{http://elks.sourceforge.net/}
\urldef{\aayurl} \url{http://www.ia64-linux.org/}
\urldef{\aazurl} \url{http://www.linux-m68k.org/}
\urldef{\abaurl} \url{http://www.mac.linux-m68k.org/}
\urldef{\abburl} \url{http://www.tazenda.demon.co.uk/phil/linux-hp/}
\urldef{\abcurl} \url{http://ftp4.de.freesbie.org/pub/misc/tsx-11/680x0/q40/install/}
\urldef{\abdurl} \url{http://penguinppc.org/mac/}
\urldef{\abeurl} \url{http://linux-apus.sourceforge.net/}
\urldef{\abfurl} \url{http://www.linux-mips.org/}
\urldef{\abgurl} \url{http://decstation.unix-ag.org/}
\urldef{\abhurl} \url{http://www.alphalinux.org/}
\urldef{\abiurl} \url{http://www.parisc-linux.org/}
\urldef{\abjurl} \url{http://vax-linux.org/}
\urldef{\abkurl} \url{https://www.ibm.com/developerworks/linux/linux390/}
\urldef{\ablurl} \url{http://www.netbsd.org/}
\urldef{\abmurl} \url{http://www.debian.org/ports/kfreebsd-gnu/}
\urldef{\abnurl} \url{https://en.wikipedia.org/wiki/OpenIndiana}
\urldef{\abourl} \url{https://en.wikipedia.org/wiki/Illumos#Relatives}
\urldef{\abpurl} \url{http://www.tldp.org/}
\urldef{\abqurl} \url{http://apcug2.org/}
\urldef{\abrurl} \url{http://www.gnu.org/gnu/gnu-history.html}
\urldef{\absurl} \url{http://linuxmafia.com/faq/Essays/meetup.html}
\urldef{\abturl} \url{http://unix.stackexchange.com/}
\urldef{\abuurl} \url{http://stackoverflow.com/}
\urldef{\abvurl} \url{http://doctype.com/}
\urldef{\abwurl} \url{http://www.codeproject.com/}
\urldef{\abxurl} \url{http://serverfault.com/}
\urldef{\abyurl} \url{http://linuxmafia.com/faq/Essays/meetup.html}
\urldef{\abzurl} \url{https://curlie.org/en/Computers/Software/Operating_Systems/Linux/User_Groups}
\urldef{\acaurl} \url{http://www.lea-linux.org/documentations/index.php/Annuaire:LUG_%26_assos_nationales}
\urldef{\acburl} \url{http://www.lug.org.uk/}
\urldef{\accurl} \url{http://linux.org.au/usergroups}
\urldef{\acdurl} \url{http://www.wikiwikiweb.de/LugsList}
\urldef{\aceurl} \url{http://lugmap.linux.it/}
\urldef{\acfurl} \url{https://www.linuxlinks.com/?s=User+Group}
\urldef{\acgurl} \url{http://libreplanet.org/wiki/Group_list}
\urldef{\achurl} \url{https://www.linuxusersgroups.org/}
\urldef{\aciurl} \url{news:comp.os.linux.announce}
\urldef{\acjurl} \url{news:comp.os.linux.misc}
\urldef{\ackurl} \url{http://www.tldp.org/HOWTO/Advocacy.html}
\urldef{\aclurl} \url{http://www.tldp.org/}
\urldef{\acmurl} \url{http://zgp.org/~dmarti/linuxmanship/}
\urldef{\acnurl} \url{http://web.archive.org/web/20120131000749/http://www.itworld.com/print/36449}
\urldef{\acourl} \url{http://www.tldp.org/}
\urldef{\acpurl} \url{http://www.linuxjournal.com/}
\urldef{\acqurl} \url{http://www.linuxformat.com}
\urldef{\acrurl} \url{http://www.linux-magazine.com/}
\urldef{\acsurl} \url{https://www.opensourceforu.com}
\urldef{\acturl} \url{http://fullcirclemagazine.org/}
\urldef{\acuurl} \url{http://linuxvoice.com/}
\urldef{\acvurl} \url{http://www.easylinux.de/}
\urldef{\acwurl} \url{https://www.linux-user.de/}
\urldef{\acxurl} \url{http://www.ubuntu-user.com/}
\urldef{\acyurl} \url{http://freesoftwaremagazine.com/}
\urldef{\aczurl} \url{http://www.ubuntu-user.com/}
\urldef{\adaurl} \url{http://lwn.net/}
\urldef{\adburl} \url{http://distrowatch.com/weekly.php}
\urldef{\adcurl} \url{http://linuxtoday.com}
\urldef{\addurl} \url{http://www.freshnews.org/}
\urldef{\adeurl} \url{https://www.kernel.org/}
\urldef{\adfurl} \url{http://www.tldp.org/}
\urldef{\adgurl} \url{mailto:cbbrowne@cbbrowne.com}
\urldef{\adhurl} \url{http://www.linuxfoundation.org/about}
\urldef{\adiurl} \url{http://www.debian.org/donations.html}
\urldef{\adjurl} \url{https://my.fsf.org/associate/support_freedom}
\urldef{\adkurl} \url{http://www.kde.org/community/donations/}
\urldef{\adlurl} \url{http://www.gnome.org/friends/}
\urldef{\admurl} \url{https:/sfconservancy.org/}
\urldef{\adnurl} \url{https://foundation.mozilla.org/en/}
\urldef{\adourl} \url{http://www.eff.org/}
\urldef{\adpurl} \url{http://www.tug.org/}
\urldef{\adqurl} \url{https://www.tug.org/donate.html}
\urldef{\adrurl} \url{http://www.gutenberg.org/}
\urldef{\adsurl} \url{http://runeberg.org/}
\urldef{\adturl} \url{http://www.osef.org/donations.html}
\urldef{\aduurl} \url{http://linuxmafia.com/faq/Linux_PR/newlug.html}
\urldef{\advurl} \url{http://ale.org/}
\urldef{\adwurl} \url{http://www.blu.org/}
\urldef{\adxurl} \url{http://www.dlug.de/}
\urldef{\adyurl} \url{https://linuxdelhi.org/}
\urldef{\adzurl} \url{http://www.iglu.org.il/}
\urldef{\aeaurl} \url{http://www.lug.or.kr/}
\urldef{\aeburl} \url{http://cofradia.org/}
\urldef{\aecurl} \url{http://www.luga.at/}
\urldef{\aedurl} \url{http://www.lugod.org/}
\urldef{\aeeurl} \url{http://www.lugor.org/}
\urldef{\aefurl} \url{http://www.nllgg.nl/}
\urldef{\aegurl} \url{http://www.ntlug.org/}
\urldef{\aehurl} \url{https://wiki.linux-ottawa.org/doku.php}
\urldef{\aeiurl} \url{http://plugfr.org/}
\urldef{\aejurl} \url{http://www.tlug.jp/}
\urldef{\aekurl} \url{http://www.linux.org.tr/}
\urldef{\aelurl} \url{https://www.alibris.com/The-Non-Lawyers-Non-Profit-Corporation-Kit-Kermit-Burton/book/4711443}
\urldef{\aemurl} \url{https://archive.org/details/alphacorporation00kerm}
\urldef{\aenurl} \url{http://web.archive.org/web/20090818124349/http://www.t-tlaw.com/lr-06.htm}
\urldef{\aeourl} \url{http://www.guidestar.org/news/features/990_myths.jsp}
\urldef{\aepurl} \url{http://epostcard.form990.org/}
\urldef{\aequrl} \url{http://www.congress.gov/cgi-bin/query/C?c105:./temp/~c105ss2v68}
\urldef{\aerurl} \url{http://www.runquist.com/article_vol_protect.htm}
\urldef{\aesurl} \url{mailto:cbbrowne@cbbrowne.com}
\urldef{\aeturl} \url{http://linuxmafia.com/~rick/lexicon.html#bikeshed}
\urldef{\aeuurl} \url{http://creativecommons.org/licenses/by-sa/4.0/}
\urldef{\aevurl} \url{http://linuxmafia.com/lug/}
\urldef{\aewurl} \url{http://www.tldp.org/}
\urldef{\aexurl} \url{mailto:cbbrowne@cbbrowne.com}

\title{راهنمای گروه کاربران لینوکس}
\author{{\textbf{علیرضا ارزه گر}} \\
  {\href{mailto:alirezaarzehgar82@gmail.com}
  {\texttt{<alirezaarzehgar82@gmail.com>}}}}
\date{\today}

\begin{document}
\maketitle
\begin{abstract}
‫این مستند راهنمایی برای تاسیس، نگهداری و بهبود گروه های کاربری گنو/لینوکس 
\lr{(GNU/Linux Users Group)}
یا لاگ 
\lr{(LUG)}
می‌باشد. نسخه انگلیسی این مستند با همکاری
\lr{Kendall Clark} 
و 
\lr{Rick Moen}
تهیه شده است. نسخه انگلیسی آن توسط 
\lr{Rick Moen}
نگهداری شده و نسخه پارسی آن درحال حاضر توسط علیرضا ارزه گر نگهداری میشود. 

%The Linux User Group HOWTO is a guide to founding, maintaining, and
%growing a GNU/Linux user group, co-authored by Kendall Clark and Rick Moen
%(now maintained by Rick Moen).  

\end{abstract}
\tableofcontents

\section{مقدمه}
‍‍‍‍‍	
\subsection{اهداف لاگ}

راهنمای گروه کاربری لینوکس قصد دارد بعنوان مرجعی برای تاسیس، نگهداری و توسعه گروه های گنو/لینوکس استفاده شود.
%The Linux User Group HOWTO is intended to serve as a guide to founding,
%maintaining, and growing a GNU/Linux user group.

گنو/لینوکس یک پیاده سازی بطور آزاد منتشر شده از یونیکس است که برای کامپیوتر های شخصی، سرور ها، کارگاه ها، \lr{PDA} ها و سیستم های نهفته طراحی شده است. گنو/لینوکس در ابتدا بر روی معماری \lr{i386} توسعه یافت و اکنون طیف گسترده ای از پردازنده ها را از کوچک تا بزرگ پشتیبانی می‌کند.
%GNU/Linux is a freely-distributable implementation of Unix for personal
%computers, servers, workstations, PDAs, and embedded systems.  It was
%developed on the i386 and now supports a huge range of processors from 
%tiny to colossal:

\begin{note}
لیست پلتفرم های پشتیبانی شده زیر قابل استنداد نیست و صرفا قصد دارد وسعت لینوکس را نمایش بدهد.
\end{note}

%Note:  The following supported-platforms list is not serious documentation.  
%The point is merely to illustrate the breadth of Linux's reach.  

اگر بطور جدی به پورت های لینوکس علاقه مندید میتوانید دو صفحه
\tturl{http://web.archive.org/web/20070813000855/http://www.itp.uni-hannover.de/ports/linux_ports.html}
و
\tturl{http://web.archive.org/web/20050308130348/http://ngc891.blogdns.net/kernel/docs/arch.txt}
را نیز برسی کنید.  (هردو صفحه مذکور در سال ۲۰۰۵ ناپدید شدند)
چرا که پشتیبانی سخت‌افزاری پیچیده‌تر از عملکرد عمومی \lr{CPU} است و شامل پشتیبانی از انواع مختلف \lr{bus} و مسائل ظریف سخت‌افزاری دیگر می‌شود (به‌ویژه برای پورت‌های لینوکس PDA / embedded / microcontroller / router با اشاره به منبع
\tturl{http://www.linuxfordevices.com/}
).

%If seriously interested in the subject of Linux ports, please see also 

%\emph{Xose Vazquez Perez's Linux ports page} \tturl{http://web.archive.org/web/20070813000855/http://www.itp.uni-hannover.de/ports/linux_ports.html}
%and
%\emph{Jerome Pinot's Linux architectures list} \tturl{http://web.archive.org/web/20050308130348/http://ngc891.blogdns.net/kernel/docs/arch.txt}

% (static mirrors, as both pages vanished in 2005), if only because 
%hardware support is more complex than just generic CPU functionality, 
%encompassing support for myriad bus variations and other subtle hardware
%issues (especially for 

%\href{http://www.linuxfordevices.com/}{Linux PDA / embedded / microcontroller / router ports}
%).  

\begin{latin}
\begin{itemize}
	\item {\bfseries Diverse \emph{PDA / embedded / microcontroller / router} \tturl{http://www.uclinux.org/ports/} devices:} 
	\begin{itemize}
		\item Advanced RISC Machines, Ltd. \emph{ARM} \tturl{http://www.arm.linux.org.uk/} family (StrongARM SA-1110, XScale, ARM6, ARM7, ARM2, ARM250, ARM3i, ARM610, ARM710,ARM7TDMI, ARM720T, and ARM920T, including Sigma Designs DVD systems using ARM cores)
		\item Analog Devices, Inc.'s \emph{Blackfin DSP} \tturl{http://www.linuxfordevices.com/c/a/News/Another-nativeDSP-Linux-port-this-one-to-ADIs-Blackfin/}
		\item Axis Communications \emph{ETRAX series} \tturl{https://en.wikipedia.org/wiki/ETRAX_CRIS}
			("CRIS" = Code Reduced Instruction Set RISC architecture)
		\item Elan SC520 and SC300
		\item FreeScale \emph{MC68EN302} \tturl{http://www.freescale.com/files/netcomm/doc/data_sheet/MC68EN302.pdf}
		\item Fujitsu \emph{FR-V} \tturl{http://ecos.sourceware.org/hardware.html#FR-V}
		\item Hitachi \emph{H8} \tturl{http://www.uclinux.org/pub/uClinux/ports/h8/} series
		\item Intel i960
		\item Intel IA32-compatibles (Cyrix MediaGX, STMicroelectronics \emph{STPC} \tturl{http://web.archive.org/web/20070626190436/http://www.stmcu.com/forums-cat-132-6.html}, ZF Micro ZFx86)
		\item Matsushita \emph{AM3x} \tturl{http://ecos.sourceware.org/hardware.html#MatsushitaAM3x}
		\item MIPS-compatibles (Toshiba TMPRxxxx / TXnnnn, NEC \emph{VR} \tturl{http://www.linux-mips.org/wiki/NEC_VR4100} series, Realtek 8181")
		\item Motorola 680x0-based machines (Motorola VMEbus boards, 
			\emph{ISICAD Prisma} \tturl{http://ds.dial.pipex.com/town/way/fr30/} machines, and Motorola Dragonball \& 
			\emph{ColdFire} \tturl{http://www.uclinux.org/ports/coldfire/} CPUs, and Cisco 2500/3000/4000 series routers)
		\item Motorola embedded \emph{PowerPC} \tturl{http://penguinppc.org/embedded/}
			(including MPC / PowerQUICC I, II, III families)
		\item NEC \emph{V850E} \tturl{http://ecos.sourceware.org/tools/linux-v850-elf.html}
		\item Renesas Technology (formerly Hitachi) SH3/SH4 (\emph{SuperH} \tturl{http://wiki.debian.org/SH4})
		\item Samsung \emph{CalmRISC} \tturl{http://ecos.sourceware.org/hardware.html#CalmRISC}
		\item Texas Instruments's 
		\emph{DM64x} \tturl{http://www.linuxfordevices.com/c/a/News/Worlds-first-nativeDSP-Linux-port/}
			and \emph{C54x DSP} \tturl{http://www.linuxfordevices.com/c/a/News/Embedded-Linux-distro-supports-TI-DSPbased-digital-media-processors/} families
		\item Xilinx \emph{PetaLinux} \tturl{http://www.xilinx.com/tools/petalinux-sdk.htm}
			(formerly SoftBlaze, formerly Microblaze) soft processor implemented on Xilinx FPGAs
	\end{itemize}

	\item {\bfseries Intel \emph{8086 / 80286} \tturl{http://elks.sourceforge.net/}}.
	\item {\bfseries Intel IA32 family:} i386, i486, Pentium, Pentium Pro, 
		Pentium II, Pentium III, Celeron, Xeon, and Pentium IV processors, 
		as well as IA32 clones from AMD (386DX/DXL/SL/SLC/SX, 
		486DX/DX2/DX4/SL/SLC/SLC2/SLC3/SX/SX2, Elan, K5, 
		K6/K6-II/K6-III), Cyrix (386DX/DXL/SL/SLC/SX, 
		486DLC/DLC2/DX/DX2/DX4/SL/SLC/SLC2/SLC3/SX/SX2, Cyrix III), 
		IDT (Winchip, Winchip 2, Winchip 2A/3), 
		IBM (486DX/DX2/DX4/SL/SLC/SLC2/SLC3/SX/SX2),
		NexGen (Nx586), Transmeta (Crusoe), 
		TI (486DLC/DLC2), UMC (486SX-S, U5D/U5S), 
		VIA (C3 Ezra "CentaurHauls", C3-2 "Nehemiah"), 
		and others.
	\item {\bfseries Intel/HP \emph{IA64} \tturl{http://www.ia64-linux.org/}:} Trillian, Itanium, Itanium2/McKinley
	\item {\bfseries x86-64 family} including AMD Hammer/Opteron/K8/Athlon64/Turion/Phenom/Phenom II/FX/Fusion and 
		Intel Prescott/Nocona/Potomac, Core, Atom, Nehalem, Sandy Bridge and Ivy Bridge
	\item {\bfseries Motorola \emph{68020-68040}
		\tturl{http://www.linux-m68k.org/} series (with MMU)}: 
		\emph{m68k Mac} \tturl{http://www.mac.linux-m68k.org/},
		Amiga, Atari ST/TT/Medusa/Falcon, HP/Apollo Domain,
		\emph{HP9000/300} \tturl{http://www.tazenda.demon.co.uk/phil/linux-hp/}, sun3, and 
		\emph{Sinclair Q40} \tturl{http://ftp4.de.freesbie.org/pub/misc/tsx-11/680x0/q40/install/}.
	\item {\bfseries Motorola/IBM PowerPC family:} Most 
		\emph{PowerMac} \tturl{http://penguinppc.org/mac/}
		 (including G3/G4/G5)  / CHRP / PReP / POP, \emph{Amiga PowerUP System}
		 \tturl{http://linux-apus.sourceforge.net/}, and IBM PPC64 (AS/400, RS/6000, iSeries,	pSeries, PowerMac G5).
	\item {\bfseries \emph{MIPS} \tturl{http://www.linux-mips.org/}:} most SGI, Cobalt Qube, 
		\emph{DECStation} \tturl{http://decstation.unix-ag.org/}, Sony PlayStation2, and many others
	\item {\bfseries DEC\emph{Alpha} \tturl{http://www.alphalinux.org/}}
	\item {\bfseries HP \emph{PA-RISC} \tturl{http://www.parisc-linux.org/}}
	\item {\bfseries SPARC International SPARC32 / SPARC64}
	\item {\bfseries Digital \emph{VAX} \tturl{http://vax-linux.org/} minicomputers and MicroVAXen}
	\item {\bfseries Mainframes:} 
	\emph{IBM S/390 models G5 and G6 / zSeries models z800, z890, z900, and z990} \tturl{https://www.ibm.com/developerworks/linux/linux390/}
		 and Fujitsu AP1000+ (SuperSPARC cluster)
\end{itemize}
\end{latin}

توجه کنید که موارد ذکر شد بعضا تنها یک بار فورک شده اند. کوچک یا بزرگ هرچیزی هستند که تا کنون نگهداری شده است. در برخی از معماری های نادر،
\ftnt{NetBSD}{http://www.netbsd.org/}
گاها کارآمد تر خواهد بود. همچنان پورت
\ftnt{Debian GNU/kFreeBSD}{http://www.debian.org/ports/kfreebsd-gnu/}
به اندازه کافی سازگار و مناسب میباشد. این پورت مجهز به کد یوزر اسپیس گنو/لینوکس بر روی کرنل فری بی اس دی با پرفرمنس بالا و پایداری بالا،
\ftnt{OpenIndiana}{https://en.wikipedia.org/wiki/OpenIndiana}
و یا دیگر 
\ftnt{Illumos distribution}{https://en.wikipedia.org/wiki/Illumos#Relatives}
میباشد و میتواند چیز دیگری مشابه کرنل اوپن سولاریس فراهم کند.

%Note that some items listed were probably one-time forks, little or not
%at all maintained since creation.  On some of the rarer architectures, \emph{NetBSD} \texttt{http://www.netbsd.org/} may be more practical.(The \emph{Debian GNU/kFreeBSD} \texttt{http://www.debian.org/ports/kfreebsd-gnu/}
% port should also be solid enough to 
%serve as a compromise option, furnishing GNU/Linux userspace code on the
%high performance / high stability FreeBSD kernel, and 
%\emph{OpenIndiana} \texttt{https://en.wikipedia.org/wiki/OpenIndiana}
% or another 
%\emph{Illumos distribution} \texttt{https://en.wikipedia.org/wiki/Illumos#Relatives}
% can provide something similar on the OpenSolaris kernel.)


\subsection{سایر منابع اطلاعاتی}
%\subsection{Other sources of information}

اگر میخواهید مطالب بیشتری مطالعه کنید، 
\ftnt{Linux Documentation Project}{http://www.tldp.org/}
منبع بهتری برای شروع می‌باشد. برای اطلاعات عمومی راجع به گروه های کاربری کامپیوتر نیز لطفا
\ftnt{Association of PC Users Groups}{http://apcug2.org/}
را بررسی کنید.

%If you want to learn more, the \emph{Linux Documentation Project} \texttt{http://www.tldp.org/} is a good place to start.
%For general information about computer user groups, please see the \emph{Association of PC Users Groups} \texttt{http://apcug2.org/}.


\section{What is a GNU/Linux user group?}

\subsection{What is GNU/Linux?}

To fully appreciate LUGs' (Linux User Groups') role in the GNU/Linux 
movement, it helps to understand what makes GNU/Linux unique.

GNU/Linux as an operating system is powerful -- but GNU/Linux as an
{\itshape {\bfseries idea}\/} about software development is even more so. GNU/Linux
is a {\bfseries free} operating system: It's licensed under the GNU General
Public Licence (and other open source / free software licences -- though 
proprietary application software is sometimes also included in
particular packagings). Thus, source code is freely available in
perpetuity to anyone. It's maintained by a unstructured group of
programmers world-wide, under technical direction from Linus Torvalds
and other key developers. GNU/Linux as a movement has no central
structure, bureaucracy, or other entity to direct its affairs. While
this situation has advantages, it poses challenges for allocation of
human resources, effective advocacy, public relations, user education,
and training.



(This HOWTO credits the Free Software Foundation's 

\emph{GNU Project} \texttt{\abrurl}
 as the crucial motive force behind creating and furthering a free 
aka open source integrated system.  Thus, it refers to "distributions" 
comprising the GNU operating system atop the Linux kernel as "GNU/Linux".
Yes, the term is awkward, and FSF's request for credit isn't widely 
honoured; but the justice of FSF's claim is obvious.)

(This HOWTO's maintainer is also fully aware that the world at large
will never adopt this usage, justice notwithstanding.  If it seems 
mannered, please indulge him, and respect the gesture.)






\subsection{How is GNU/Linux unique?}

GNU/Linux's loose structure is unlikely to change.  That's a good thing:
It works precisely because people are free to come and go as they
please: {\bfseries Free programmers are happy programmers are effective
programmers}.

However, this loose structure can disorient the new user: Whom
does she call for support, training, or education? How does she know
what GNU/Linux is suitable for?

In part, LUGs provide the answers, which is why LUGs have been vital to
the movement: Because your town, village, or metropolis sports no
Linux Corporation "regional office", the LUG takes on many of the same
roles a regional office does for a large multi-national corporation.

GNU/Linux is unusual in neither having nor being burdened by central
structures or bureaucracies to allocate its resources, train its users,
and support its products. These jobs get done through diverse means: the
Internet, consultants, VARs, support companies, colleges, and
universities. However, increasingly, in many places around the globe,
they are done by a LUG.




\subsection{What is a user group?}

Computer user groups are not new. In
fact, they were central to the personal computer's history:
Microcomputers arose in large part to satisfy demand for affordable,
personal access to computing resources from electronics, ham radio, and
other hobbyist user groups.  Giants like IBM eventually discovered the
PC to be a good and profitable thing, but initial impetus came from the
grassroots, leading to groundbreaking efforts like SHARE (1955-present) 
and DECUS (1961-2008).

In the USA, user groups have changed -- many for the worse --
with the times. The financial woes and dissolution of the largest user
group ever, the Boston Computer Society, were well-reported; but, all
over the USA, most PC user groups have seen memberships decline.
American user groups in their heyday produced newsletters, maintained
shareware and diskette libraries, held meetings and social events, and,
sometimes, even ran electronic bulletin board systems (BBSes). With the
advent of the Internet, however, many services that user groups once
provided migrated to things like CompuServe and the Web.

GNU/Linux's rise, however, coincided with and was intensified by the
general public "discovering" the Internet. As the Internet grew more
popular, so did GNU/Linux: The Internet brought new users,
developers, and vendors. So, the same force that sent traditional user
groups into decline propelled GNU/Linux forward, and inspired new groups 
concerned exclusively with it. 

To give just one indication of how LUGs differ from traditional 
user groups: Traditional groups must closely 
monitor what software users redistribute at meetings.
While illegal copying of restricted proprietary software certainly
occurred, it was officially discouraged -- for good reason.
At LUG meetings, however, that entire mindset simply does not apply:
Far from being forbidden, unrestricted copying of GNU/Linux
should be among a LUG's primary goals.  In fact, there is anecdotal
evidence of traditional user groups having difficulty adapting to
GNU/Linux's ability to be lawfully copied at will.

(Caveat:  A few distributions bundle GNU/Linux with proprietary
software packages whose terms don't permit public redistribution.
Check licence terms, if in doubt.  Offers or requests to copy 
distribution-restricted proprietary software of any sort should be
heavily discouraged anywhere in LUGs, and declared off-topic for all 
GNU/Linux user group on-line forums, for legal reasons.)




\subsection{Avoiding Burnout and Decline}

Since around 2003, LUGs in developed countries have seen a decline
similar to that of traditional user groups.  The causes can be debated,
and might include:

\begin{itemize}
\item GNU/Linux being so successful that it's often perceived as
infrastructure rather than as something new and interesting.
\item LUGs getting lost in the noise of social media.
\item Early adopters critical to making LUGs function moving
on to other interests.
\item 
\emph{Meetup.com} \texttt{\absurl}
, with its strong inward-facing focus, sucking
away available talent and energy, and making LUGs less noticeable.
\item GNU/Linux becoming so much easier to install
and use that focus has shifted to more-specialised topics better served
by more-specialised technical communities (DevOps, bioinformatics,
cloud computing, embedded computing, and many others).
\item LUG leaders poorly managing a generational transition, 
leaving nobody ready to take over as they bow out.
\item Greater ubiquity of the Internet generally, and
specifically reputation-based collaborative sites like 

\emph{StackExchange} \texttt{\abturl}
,

\emph{StackOverflow} \texttt{\abuurl}
, 

\emph{Doctype} \texttt{\abvurl}
, 

\emph{Codeproject} \texttt{\abwurl}
, and 

\emph{Serverfault} \texttt{\abxurl}
, not to mention 
users becoming skilled at Web-searching, making LUGs far less
pragmatically necessary, and the main action involving SaaS sites having 
site scale and network effects with which LUGs cannot compete.
\end{itemize}


A few time-tested tips for averting LUG flameout:

\begin{itemize}
\item Automation is your friend.  Any task that can be scripted, 
should be scripted.
\item Check all your LUG's systems, both technical and social,
for single points of failure (SPoFs).  Keep trying to make sure there are
fallbacks if anything or anyone fails.  Do (and test) backups.  Ensure that nothing
important can be done by only one person.
\item Beware of your LUG, or any individual in it, committing 
to carrying out too much work, or with too great frequency.  It's 
better for a LUG to do less, or have its functions occur less often,
than risk people wearing out and leaving.
\item Remember that if people aren't having fun, they won't 
continue for long.  E.g., if your group becomes less technical and 
more social, don't fret.  It's probably a healthy thing.
\item Carefully guard your significant assets, such as domain
ownership, difficult-to-acquire meeting venues, and the names of key
corporate contacts, and keep them away from problematic people sometimes
drawn to LUGs.  Even if you, say, wrestle your domain away from someone
who's suddenly decided to destroy the LUG (which does happen), the
strife will drive away key people.
\end{itemize}







\subsection{Summary}

For the GNU/Linux movement to grow, among other requirements,
LUGs must proliferate and succeed.  Because of GNU/Linux's
unusual nature, LUGs must provide some of the same functions a "regional
office" provides for large computer corporations like IBM, Microsoft,
and Sun. LUGs can and must train, support, and educate users,
coordinate consultants, advocate GNU/Linux as a computing solution,
and even serve as liaison to local news outlets.




\section{What LUGs exist?}

Since this document is meant as a guide not only to maintaining and
growing LUGs but also to founding them, we should, before going further,
discuss what LUGs already exist.




\subsection{LUG lists}

There are several LUG lists on the Web. If you are considering founding a
LUG, your first task should be to find any nearby existing LUGs.
{\itshape Your best bet may be to join a LUG already established in your area,
rather than founding one.\/}

As of 2016, there were LUGs in 43 US states, seven of Canada's ten provinces, 
all six of Australia's states plus the Australian Capital Territory, in 76 
locations in India, and over 100 other countries, including Russia, China, most 
of Western and Eastern Europe, and many parts of Africa, Asia, South America, 
Central America, Oceania, and the Caribbean.  (This does not include Linux 
groups internal to 
\emph{Meetup.com} \texttt{\abyurl}
.)

\begin{itemize}
\item 
\emph{Curlie: Linux User Groups} \texttt{\abzurl}
\item 
\emph{L\'ea-Linux List} \texttt{\acaurl}
  (text is in French)
\item 
\emph{UK Linux User Groups} \texttt{\acburl}
\item 
\emph{Linux Australia} \texttt{\accurl}
\item 
\emph{LUGs List for India and Asia} \texttt{\acdurl}
\item 
\emph{I Linux User Group italiani} \texttt{\aceurl}
\item 
\emph{Linux Links UserGroups} \texttt{\acfurl}
\item 
\emph{LibrePlanet Group List} \texttt{\acgurl}
 (lists FSF affiliates only)
\item 
\emph{LinuxUserGroups.org} \texttt{\achurl}
 (however, a/o 2022, seems neglected since June 2001)
\end{itemize}





\subsection{Solidarity versus convenience}

While (most) LUG lists on the Web are well-maintained, likely they don't
list every LUG. If considering founding a LUG, I suggest, in addition to
consulting these lists, posting a message to 
\emph{comp.os.linux.announce} \texttt{\aciurl}
, 
\emph{comp.os.linux.misc} \texttt{\acjurl}
, or an
appropriate regional Usenet hierarchy, inquiring about nearby LUGs. You
should also lodge a query (mailing list post, comment during a meeting)
at any existing LUG you are aware of anywhere near your area,
about LUGs near you.  If no such (nearby) LUG exists, your postings will
alert potential members to your initiative.

Carefully balance convenience against solidarity:  If a LUG exists in
your metropolitan area but on the other side of the city, starting a new
group may be better for convenience's sake. On the other hand, joining
the other group may be better for reasons of unity and solidarity.
{\bfseries {\itshape Greater numbers almost always means greater power, influence,
and efficiency\/}}. While two groups of 100 members each might be
nice, one with 200 has advantages. Of course, if you live in a small
town or village, any group is better than none.

The point is that starting a LUG is a significant undertaking, which
should be commenced with all relevant facts and some appreciation of the
effect on other groups.




\section{What does a LUG do?}

LUGs' goals are as varied as their locales.  There is no LUG master
plan, nor will this document supply one. Remember: GNU/Linux is free from
bureaucracy and centralised control; so are LUGs.

It is possible, however, to identify a core set of goals for a 
LUG:

\begin{itemize}
\item advocacy
\item education
\item support
\item socialising
\end{itemize}


Each LUG combines these and other goals uniquely, according to its
membership's needs.




\subsection{GNU/Linux advocacy}

The urge to advocate the use of GNU/Linux is widely felt.  When you find
something that works well, you want to tell as many people as you can.
While it is certainly beneficial to the movement, each and every time a
computer journalist writes a positive review of GNU/Linux, it is also
beneficial every time satisfied GNU/Linux users brief their friends,
colleagues, employees, or employers.

There is effective advocacy, and there is ineffective carping: As 
users, we must be constantly vigilant to advocate GNU/Linux in such a way as
to reflect positively on the product, its creators and developers, and
our fellow users.  The 
\emph{Linux Advocacy HOWTO} \texttt{\ackurl}
, available at the 

\emph{Linux Documentation Project} \texttt{\aclurl}
, 
gives some helpful suggestions, as does Don Marti's excellent 

\emph{Linuxmanship} \texttt{\acmurl}
 essay.

Over the long life of this HOWTO, GNU/Linux more or less won the 
day, so the HOWTO maintainer has deleted much of this section, and 
advocacy in his view has become, in his view, overwhelmingly irrelevant.




\subsection{The limits of advocacy}

Advocacy can be mis-aimed; advocacy can go wrong and be
counterproductive; advocacy can be simply inappropriate in the first
place.  The matter merits careful thought, to avoid wasted time or
worse.

Many attempts at advocacy fail ignominiously because the advocate fails
to listen to what the other party feels she wants or needs.  (As Eric
S. Raymond 
\emph{says} \texttt{\acnurl}
, 
"Appeal to the prospect's interests and values, not to
yours.") If that person wants exactly the proprietary-OS setup she
already has, then advocacy wastes your time and hers.  If her
stated requirements equate exactly to MS-Project, MS-Visio, and
Outlook/Exchange groupware, then trying to "sell" her what she doesn't
want will only annoy everyone (regardless of whether her requirements
list is real or artificial).  Save your effort for someone more
receptive.  

Along those lines, bear in mind that, for many people, perhaps most, an
"advocate" is perceived as a salesman, and thus classified as someone to
resist rather than listen to fairly.  They've never heard of someone
urging them to adopt a piece of software without
benefiting materially, so they assume there must be something in
it for you and will push back, and
act as if they're doing you a personal favour to even listen, let alone
try your recommendations.  

I recommend bringing such discussions back to Earth
immediately, by pointing out that software policy should be based in
one's own long-term self interest, that you have zero personal stake in
their choices, and that you have better uses for your time than speaking
to an unreceptive audience. After that, if
they're still interested, at least you won't face the same artificial
obstacle.

At the same time, make sure you don't live up to the stereotype of the
OS advocate, either.  Just proclaiming your views at someone without
invitation is downright rude and offensive.  Moreover, when done
concerning GNU/Linux, it's also pointless:  Unlike the case with proprietary
OSes, our OS will not live or die by the level of its acceptance and
release/maintenance of ported applications.  It and all key applications
are open source: the programmer community that maintains it is
self-supporting, and would keep it advancing and and healthy regardless
of whether the business world and general public uses it with wild
abandon, only a little, or not at all. Because of its open-source
licence terms, source code is permanently available. GNU/Linux cannot be
"withdrawn from the market" on account of insufficient popularity, or at
the whim of some company.  Accordingly, there is simply no point in
arm-twisting OS advocacy -- unlike that of some OS-user communities we
could mention.   (Why not just make information available for those
receptive to it, and stop there?  That meets any reasonable person's
needs.)

Last, understand that the notion of "use value" for software is quite
foreign to most people -- the notion of measuring software's value by
what you can do with it.  The habit of valuing everything at
{\itshape acquisition cost\/} is deeply ingrained.  In 1996, I heard a young
fellow from Caldera Systems speak at a Berkeley, California LUG about
the origins of Caldera Network Desktop (the initial name of their GNU/Linux
distribution) in Novell, Inc.'s "Corsair" desktop-OS project:  In
surveying corporate CEOs and CTOs, they found corporate officers to be
inherently unhappy with anything they could get for free.  So, Caldera
offered them a solution -- by charging money.

Seen from this perspective, being conservative about the costs and
difficulties of GNU/Linux deployments helps make them positively attractive
-- and protects your credibility as a speaker.  Even better would be
to frame the discussion of costs in terms of the cost of functionality
(e.g., 1000-seat Internet-capable company e-mail with offline-user
capability and webmail) as opposed to listing software as a retail-style
line-item with pricing:  After all, any software project has costs,
even if the acquisition price tag is zero, and the real point of open
source isn't initial cost but rather long-term control over IT -- a key
part of one's operations:  With proprietary systems, the user (or
business) has lost control of IT, and is on the wrong side of a monopoly
relationship with one's vendor.  With open source, the user is in
control, and nobody can take that away.  Explained that way (as
opportunity to reduce and control IT risk), people readily understand
the difference -- especially CEOs -- and it's much more significant over
the long term than acquisition cost.




\subsection{GNU/Linux education}

Not only is it the business of a LUG to advocate GNU/Linux usage, but
also to train members, as well as the nearby computing public,
to use our OS and associated components -- a goal that can make a huge
real-world difference in one's local area.  While universities and
colleges are increasingly including GNU/Linux in their curricula, for
sundry reasons, this won't reach some users.  For those, a LUG can
give basic or advanced help in system administration, programming,
Internet and intranet technologies, etc.

In an ironic twist, many LUGs have turned out to be a backbone of
corporate support: Every worker expanding her computer skills through
LUG participation is one fewer the company must train.  Though home
GNU/Linux administration doesn't exactly scale to running corporate data
warehouses, call centres, or similar high-availability facilities, it's
light years better preparation than MS-Windows experience.  As Linux has
advanced into journaling filesystems, high availability, real-time
extensions, and other high-end Unix features, the already blurry line
between GNU/Linux and "real" Unixes vanished entirely.

Not only is such education a form of worker training, but it will also
serve, as information technology becomes increasingly vital to the
global economy, as community service: In the USA's metropolitan areas,
for example, LUGs have taken GNU/Linux into local schools, small businesses,
community and social organisations, and other non-corporate
environments. This accomplishes the goal of advocacy and also
educates the general public.  As more such organisations seek Internet
presence, provide their personnel dial-in access, or other
GNU/Linux-relevant functions, LUGs gain opportunities for community
participation, through awareness and education efforts -- extending to
the community the same generous spirit characteristic of GNU/Linux and the
free software / open source community from its very beginning. Most
users can't program like Torvalds, but we can all give time and
effort to other users, the GNU/Linux community, and the broader
surrounding community.

GNU/Linux is a natural fit for these organisations, because deployments
don't commit them to expensive licence, upgrade, or maintenance fees.
Being technically elegant and economical, it also runs very well on
cast-off corporate hardware that non-profit organisations are only too
happy to use: The unused Pentium Core 2 in the closet can do {\bfseries real
work}, if someone installs GNU/Linux on it.

In addition, education assists other LUG goals over time, in
particular that of support: Better education means better
support, which in turn facilitates education, and eases the 
community's growth.  Thus, education forms the entire effort's keystone:
If only two or three percent of a LUG assume the remainder's support
burden, that LUG's growth will be stifled. One thing you can count on:
{\bfseries {\itshape If new and inexperienced users don't get needed help
from their LUG, they won't participate there for long\/}}.
If a larger percentage of members support the rest, the LUG will not
face that limitation. education -- and, equally, support for
allied projects such as the Apache Web server, X.org, Freedesktop.org, 
TeX, LaTeX, etc.  -- is key to this dynamic: Education turns new users into
experienced ones.

Finally, GNU/Linux is a self-documenting operating environment: In other
words, writing and publicising our community's documentation is up to
us.  Therefore, make sure LUG members know of the 
\emph{Linux Documentation Project} \texttt{\acourl}
 and its worldwide
mirrors.  Consider operating an LDP mirror site.  Also, make sure to
publicise -- through {\ttfamily comp.os.linux.announce}, the LDP, and other
pertinent sources of information -- any relevant documentation the LUG
develops: technical presentations, tutorials, local FAQs, etc.  LUGs'
documentation often fails to benefit the worldwide community for no
better reason than not notifying the outside world.  Don't let that
happen:  It is highly probable that if someone at one LUG had a question
or problem with something, then others elsewhere will have it, too.




\subsection{GNU/Linux support}

Of course, for the {\bfseries newcomer}, the primary role of a
LUG is GNU/Linux support -- but it is a mistake to suppose that 
support means only {\itshape technical\/} support for new users. It
should mean much more.

LUGs have the opportunity to support:

\begin{itemize}
\item users
\item consultants
\item businesses, non-profit organisations, and schools
\item the GNU/Linux movement
\end{itemize}





\subsubsection{Users}

New users' most frequent complaint, once they have GNU/Linux
installed, is the steep learning curve characteristic of all modern 
Unixes. (That sentence was true in 1997 when this HOWTO's first
maintainer wrote it, but happily very little, any more.)  With that learning
curve, however, comes the power and flexibility of a real operating
system. A LUG is often the new user's main resource, to flatten the
learning curve.

During GNU/Linux's first decade, it gained some first-class journalistic 
resources, which should not be neglected:  The main (surviving) monthly 
magazine of longest standing is {\itshape 
\emph{Linux Journal} \texttt{\acpurl}
\/} (USA).
More recently, 
they've been joined by 
{\itshape 
\emph{Linux Format} \texttt{\acqurl}
\/} (UK),
{\itshape 
\emph{Linux Magazine} \texttt{\acrurl}
\/} (German publishing firm; publishes in English, German, Polish, Brazilian Portuguese, and Spanish; North American edition is named {\itshape Linux Pro Magazine\/}),
{\itshape 
\emph{Open Source For You} \texttt{\acsurl}
\/} (India; formerly {\itshape LINUX For You)\/},
{\itshape 
\emph{Full Circle} \texttt{\acturl}
\/} (international; covers Ubuntu family distributions), 
{\itshape 
\emph{Linux Voice} \texttt{\acuurl}
\/} (UK), 
{\itshape 
\emph{easyLinux} \texttt{\acvurl}
\/} (German),
{\itshape 
\emph{LinuxUser} \texttt{\acwurl}
\/} (German), 
{\itshape 
\emph{Ubuntu User} \texttt{\acxurl}
\/} (German
publishing firm; in English), 
{\itshape 
\emph{Free Software Magazine} \texttt{\acyurl}
\/} (formerly {\itshape The Open Voice\/}), and
{\itshape 
\emph{Ubuntu User} \texttt{\aczurl}
\/} (German publishing firm; in English).

Standout on-line magazines and news sites with weekly or better publication 
cycles include {\itshape 
\emph{Linux Weekly News} \texttt{\adaurl}
\/}, 
{\itshape 
\emph{DistroWatch Weekly} \texttt{\adburl}
\/},
{\itshape 
\emph{Linux Today} \texttt{\adcurl}
\/}, and

\emph{FreshNews} \texttt{\addurl}
.

All of these resources have eased LUGs' job of spreading essential
news and information --  about bug fixes, security problems, patches, 
new kernels, etc., but new users must still be made aware of
them, and taught that the newest kernels are always
available from 
\emph{kernel.org} \texttt{\adeurl}
,
that the 
\emph{Linux Documentation Project} \texttt{\adfurl}
 has newer versions of Linux HOWTOs than do DVD/CD-based GNU/Linux
distributions, and so on.

Intermediate and advanced users also benefit from proliferation of
timely and useful tips, facts, and secrets. Because of the GNU/Linux
world's manifold aspects, even advanced users often learn new tricks or
techniques simply by participating in a LUG. Sometimes, they learn of
software packages they didn't know existed; sometimes, they just
remember arcane {\ttfamily vi} command sequences they've not used since
college.




\subsubsection{Consultants}



LUGs can help consultants find their customers and vice-versa,
by providing a forum where they can come together.
Consultants also aid LUGs by providing experienced leadership.
New and inexperienced users gain benefit from both LUGs and
consultants, since their routine or simple requests for support are
handled by LUGs {\itshape gratis\/}, while their complex needs and problems --
the kind requiring paid services -- can be fielded by consultants found 
through the LUG.

The line between support requests needing a consultant and those
that don't is sometimes indistinct; but, in most cases, the difference
is clear. While a LUG doesn't want to gain the reputation for
pawning new users off unnecessarily on consultants -- as this is simply
rude and very anti-GNU/Linux behaviour -- there is no reason for LUGs not to
help broker contacts between users needing consulting services and
professionals offering them.

Caveat:  While "the difference is clear" to intelligent people of goodwill,
the Inevitable Ones are {\itshape also\/} always with us, who act willfully
dense about the limits of free support when they have pushed those
limits too far.  Remember, too, my earlier point about the vast majority
of the population valuing everything at acquisition cost (instead of use
value), {\itshape including what they receive for free\/}.  This leads some,
especially some in the corporate world, to use (and abuse) LUG
technical support with wild abandon, while simultaneously complaining
bitterly of its inadequate detail, insufficient promptness, supposedly
unfair expectations that the user learn and not re-ask minor variations on
the same question endlessly, etc.  In other words, they treat relations
with LUG volunteers the way they would a paid support vendor, but one
they treat with {\itshape zero respect\/} because of its zero acquisition
cost.

In the consulting world, there's a saying about applying "invoice therapy" 
to such behaviour:  Because of the value system alluded to above, if
your consulting advice is poorly heeded and poorly used, it just might
be the case that you need to charge more.  By contrast, the technical
community has often been characterised as a "gift culture", with a
radically different value system: Members gain status through enhanced
reputation among peers, which in turn they improve through visible
participation:  code, documentation, technical assistance to the public,
etc.

Clash between the two very different value-based cultures is inevitable
and can become a bit ugly.  LUG activists should be prepared to intercede
before the ingrate newcomer is handed her head on a platter, and
politely suggest that her needs would be better served by paid
(consultant-based) services.  There will always be judgement calls;
the borderline is inherently debatable and a likely source of
controversy.

Telltale signs that a questioner may need to be transitioned to consulting-based assistance include:

\begin{itemize}
\item An insistence on getting solutions in "recipe" (rote) form, 
with the apparent aim of not needing to learn technological 
fundamentals.
\item Asking the same questions (or ones closely related) repeatedly.
\item Insisting on {\itshape private\/} assistance from helpers active in
{\itshape public\/} (GNU/Linux community) forums.
\item Providing only vague problem descriptions, or ones that change with time.
\item Interrupting answers in order to ask additional questions 
(suggesting lack of attention to the answers).
\item Demanding that answers be recast or delivered more quickly 
(suggesting that the questioner's time and trouble are 
valuable, but that helpers' are not).
\item Asking unusually complex, time-consuming, and/or multipart 
questions.
\end{itemize}


In general, LUG members are especially delighted to help, on a volunteer
basis, members who seem likely to participate in the "gift
culture" by picking up its body of lore and, in turn, perpetuating it
by teaching others in their turn.  Certainly, there's nothing wrong with
having other priorities and values, but such folk may in some cases be
best referred to paid assistance, as a better fit for their needs.

An additional observation that may or may not be useful, at this point:
There are things one may be willing to do for free, to assist others in the
community, that one will refuse to do for money:  Shifting from
assisting someone as a volunteer fundamentally changes the relationship.
A fellow computerist who suddenly becomes a customer is a very different
person; one's responsibilities are quite different, and greater.  You're
advised to be aware, if not wary, of this distinction.






\subsubsection{Businesses, non-profit organisations, and schools}

LUGs also have the opportunity to support local businesses and
organisations. This support has two aspects: First, LUGs can support
businesses and organisations wanting to use our OS (and its 
applications) as a part of their
computing and IT efforts. Second, LUGs can support local businesses
and organisations developing software for GNU/Linux, cater to users,
support or install distributions, etc.

The support LUGs can provide to local businesses wanting to use GNU/Linux as
a part of their computing operations differs little from the help LUGs
give individuals trying GNU/Linux at home. For example, compiling the Linux
kernel doesn't really differ. Supporting businesses, however, may
require supporting proprietary software -- e.g., the Oracle, Sybase,
and DB2 databases (or VMware, CrossOver Linux, and such things).   
Some LUG expertise in these areas may help businesses make the leap
into GNU/Linux deployments.

This leads us directly to the second kind of support a LUG can give to
local businesses: LUGs can serve as a clearinghouse for information
available in few other places. For example:

\begin{itemize}
\item Which local ISP is Linux-friendly?
\item Are there any local hardware vendors building Linux PCs?
\end{itemize}


Maintaining and making this kind of information public not only helps
the LUG members, but also helps friendly businesses and encourages
them to continue to be GNU/Linux-friendly. It may even, in some cases, help
further a competitive environment in which other businesses are
encouraged to follow suit.




\subsubsection{Free / open-source software development}

Finally, LUGs may also support the movement by soliciting and
organising charitable giving. 
\emph{Chris Browne} \texttt{\adgurl}
 has thought about this issue as much as
anyone I know, and he contributes the following:




\paragraph{Chris Browne on free software / open source philanthropy}

 
A further involvement can be to encourage sponsorship of various
GNU/Linux-related organisations in a financial way.  With the 
multiple millions of users, it would be entirely plausible for grateful 
users to individually contribute a little. Given millions of users, and 
the not-unreasonable sum of a hundred dollars of "gratitude" per user (\$100 being
roughly the sum {\itshape not\/} spent this year upgrading a Microsoft OS),
that could add up to {\itshape hundreds of millions\/} of dollars towards
development of improved GNU/Linux tools and applications.



 
A user group can encourage members to contribute to various
"development projects". Having some form of "charitable tax exemption"
status can encourage members to contribute directly to the group,
getting tax deductions as appropriate, with contributions flowing on to
other organisations.



 
It is appropriate, in any case, to encourage LUG members to direct
contributions to organisations with projects and goals they
individually wish to support.



 
This section lists possible candidates. None is being explicitly 
recommended here, but the list represents useful food for
thought.  Many are registered as charities in the USA, thus
making US contributions tax-deductible.



Here are organisations with activities particularly directed towards
development of software working with GNU/Linux:

\begin{itemize}
\item 
\emph{The Linux Foundation} \texttt{\adhurl}
\item 
\emph{Debian / Software In the Public Interest} \texttt{\adiurl}
\item 
\emph{Free Software Foundation} \texttt{\adjurl}
 
\item 
\emph{KDE Project (KDE e.V.)} \texttt{\adkurl}
\item 
\emph{GNOME Foundation} \texttt{\adlurl}
\item 
\emph{Software Freedom Conservancy} \texttt{\admurl}
\item 
\emph{The Mozilla Foundation} \texttt{\adnurl}
\end{itemize}




Contributions to these organisations have the direct effect of
supporting creation of freely redistributable software usable with
GNU/Linux.  Dollar for dollar, such contributions almost certainly yield
greater benefit to the community than any other kind of spending.



 
There are also organisations less directly associated with GNU/Linux, that
may nonetheless be worthy of assistance, such as:

\begin{itemize}
\item The 
\emph{Electronic Frontier Foundation} \texttt{\adourl}
 

Based in San Francisco, EFF is a donor-supported membership organization
working to protect our fundamental rights regardless of technology; to
educate the press, policy-makers, and the general public about civil
liberties issues related to technology; and to act as a defender of 
those liberties. Among our various activities, EFF opposes misguided
legislation, initiates and defends court cases preserving individuals'
rights, launches global public campaigns, introduces leading edge
proposals and papers, hosts frequent educational events, engages the
press regularly, and publishes a comprehensive archive of digital civil
liberties information at one of the most linked-to Web sites in the
world.



\item The LaTeX3 Project Fund 

 
The 
\emph{TeX Users Group (TUG)} \texttt{\adpurl}
 is
working on the "next generation" version of the LaTeX publishing
system, known as LaTeX3.  GNU/Linux is one of the platforms on which TeX
and LaTeX are best supported.

 Donations for the project can be sent to:
\begin{tscreen}
\begin{verbatim}
TeX Users Group
c/o Robin Laakso, executive director
TeX Users Group
PO Box 2311
Portland, OR 97208-2311
\end{verbatim}
\end{tscreen}


Alternatively, donations can be made 

\emph{online} \texttt{\adqurl}
.



\item  
\emph{Project Gutenberg} \texttt{\adrurl}
 

Project Gutenberg's purpose is to make freely available in electronic
form the texts of public-domain books.  This isn't directly a "Linux
thing", but seems fairly worthy, and they actively encourage platform
independence, which means their "products" are quite usable with GNU/Linux.



\item  
\emph{Project Runeberg} \texttt{\adsurl}
 
Project Runeberg is similar to Project Gutenberg, except concentrating
on making editions of classic Nordic (Scandinavian) literature openly
available over the Internet.





\item  
\emph{Open Source Education Foundation} \texttt{\adturl}
 
The Open Source Education Foundation's purpose to enhance K-12 education
through the use of technologies and concepts derived from The Open
Source and Free Software movement.  In conjunction with Tux4Kids, OSEF 
created a bootable distribution of GNU/Linux (Knoppix for Kids) based 
on Klaus Knopper's Knoppix, aimed at kids, parents, teachers, and 
other school officials. OSEF installs and supports school computer labs, 
and has developed a "K12 Box" as a compact Plug and Play workstation 
computer for student computer labs.



\end{itemize}


(Please note that suggested additions to the above list of GNU/Linux-relevant 
charities are most welcome.)






\subsubsection{Linux movement}

I have referred throughout this HOWTO to what I call the {\bfseries GNU/Linux
movement}. There really is no better way to describe the
international GNU/Linux phenomenon: It isn't a bureaucracy, but is
organised. It isn't a corporation, but is important to businesses
everywhere. The best way for a LUG to support the international GNU/Linux
movement is to keep the local community robust, vibrant, and
growing. GNU/Linux is {\itshape developed\/} internationally, which is easy
enough to see by reading the kernel source code's 
\url{MAINTAINERS} file -- but
GNU/Linux is also {\itshape used\/} internationally.  This ever-expanding
user base is key to GNU/Linux's continued success, and is where the LUGs
are vital.

The movement's strength internationally lies in offering
unprecedented computing power and sophistication for its cost and
freedom. The keys are value and independence from proprietary control.
Every time a new person, group, business, or organisation experiences
GNU/Linux's inherent value, the movement grows.  LUGs help that
happen.




\subsection{Linux socialising}

The last goal of a LUG we'll cover is socialising -- in some ways,
the most difficult goal to discuss, because it isn't clear how
many or to what degree LUGs do it. While it would be strange to
have a LUG that didn't engage in the other goals, there may be
LUGs for which socialising isn't a factor.

It seems, however, that whenever two or three GNU/Linux users get together,
fun, hijinks, and, often, beer follow. Linus Torvalds has
always had one enduring goal for Linux: to have more fun. For hackers,
kernel developers, and GNU/Linux users, there's nothing quite like
downloading a new kernel, recompiling an old one, fooling with a
window manager or desktop environment, hacking some code, or experimenting
with an innovative embedded Linux computer. GNU/Linux's sheer fun keeps many
LUGs together, and leads LUGs naturally to socialising.

By "socialising", here I mean primarily sharing experiences, forming
friendships, and mutually-shared admiration and respect. There is
another meaning, however -- one social scientists call
{\itshape acculturation\/}. In any movement, institution, or human
community, there is the need for some process or pattern of events in
and by which, to put it in GNU/Linux terms, newcomers are turned into
hackers. In other words, acculturation turns you from "one of them" to
"one of us".

It is important that new users come to learn GNU/Linux culture,
concepts, traditions, and vocabulary.  GNU/Linux acculturation, unlike "real
world" acculturation, can occur on mailing lists, Web forums, and
Usenet, although the latter's efficacy is challenged by poorly
acculturated users and by spam. LUGs are often much more efficient at
this task than are mailing lists, Web forums, or newsgroups, precisely
because of LUGs' greater interactivity and personal focus.




\section{LUG activities}

In the previous section I focused exclusively on what LUGs do and 
should do. This section's focus shifts to practical strategies.

There are, despite permutations of form, two basic things LUGs do:
First, members meet in physical space; second, they communicate
in cyberspace. Nearly everything LUGs do can be seen in terms of
meetings and online resources.




\subsection{Meetings}

As I said above, physical meetings are synonymous with LUGs (and 
most user groups).  LUGs have these kinds of meetings:

\begin{itemize}
\item social
\item technical presentations
\item informal discussion groups
\item online videoconferencing (Jitsi Meet, Zoom, etc.)
\item user group business
\item GNU/Linux installation
\item configuration and bug-squashing
\end{itemize}


What do LUGs do at these meetings?

\begin{itemize}
\item Install distributions for newcomers and strangers.
\item Teach members about GNU/Linux.
\item Compare GNU/Linux to other operating systems.
\item Teach members about application software.
\item Discuss advocacy.
\item Discuss the free software / open-source movement.
\item Discuss user group business.
\item Eat, drink, and be merry.
\end{itemize}







\subsection{Online resources}

The commercial rise of the Internet coincided roughly with that of
GNU/Linux; the latter owes something to the former. The Net has always been
important to development. LUGs are no different: Most have Web
pages, if not whole Web sites. In fact, I'm not sure how else to find a
LUG, but to check the Web.  

It makes sense, then, for a LUG to make use of whatever Internet
technologies they can: Web sites, Jitsi Meet/Zoom/etc., mailing lists, 
wikis, e-mail, Web discussion forums, netnews, etc. As the world of commerce is
discovering, the Net is an effective way to advertise, inform, educate,
and even sell. The other reason LUGs make extensive use of Internet
technology is that the very essence of GNU/Linux is to {\itshape provide\/} a stable and rich platform to deploy these technologies. So,
not only do LUGs benefit from, say, establishment of a Web site,
because it advertises their existence and helps organise members,
but, in deploying these technologies, LUG members 
learn about them and see GNU/Linux at work.

Arguably, a well-maintained Web site is the one must-have, among those
Internet resources.  My essay

\emph{Recipe for a Successful Linux User Group} \texttt{\aduurl}
, for that reason,
spends considerable time discussing Web issues.  Quoting it (in outline form):

\begin{itemize}
\item You need a Web page.
\item Your Web page needs a reasonable URL.
\item You need a regular meeting location.
\item You need a regular meeting time.
\item You need to avoid meeting-time conflicts.
\item You need to make sure that meetings happen as advertised, without fail.
\item You need a core of several enthusiasts.
\item Your core volunteers need out-of-band methods of communication.
\item You need to get on the main lists of LUGs, and keep your entries accurate.
\item You must have login access to maintain your Web pages, as needed.
\item Design your Web page to be forgiving of deferred maintenance.
\item Always include the day of the week, when you cite event dates. Always check that day of the week, first, using cal.
\item Place time-sensitive and key information prominently near the top of your main Web page.
\item Include maps and directions to your events.
\item Emphasise on your main page that your meeting will be free of charge and open to the public (if it is).
\item You'll want to include an RSVP "mailto" hyperlink, on some events.
\item Use referral pages.
\item Make sure every page has a revision date and maintainer link.
\item Check all links, at intervals.
\item You may want to consider establishing a LUG mailing list.
\item You don't need to be in the Internet Service Provider business.
\item Don't go into any other business, either.
\item Walk the walk. (Do the LUG's computing on GNU/Linux.)
\end{itemize}


That essay partly supplements (and partly overlaps) this HOWTO.

Some LUGs using the Internet effectively:

\begin{itemize}
\item 
\emph{Atlanta Linux Enthusiasts} \texttt{\advurl}
\item 
\emph{Boston Linux and Unix} \texttt{\adwurl}
\item 
\emph{Dusseldorfer Linux Users Group} \texttt{\adxurl}
\item 
\emph{India Linux Users Group Delhi} \texttt{\adyurl}
\item 
\emph{Israeli Group of Linux Users} \texttt{\adzurl}
\item 
\emph{Korean Linux Users Group} \texttt{\aeaurl}
\item 
\emph{Linux Mexico (La Cofradia Digital)} \texttt{\aeburl}
\item 
\emph{Linux User Group Austria} \texttt{\aecurl}
\item 
\emph{Linux User Group of Davis} \texttt{\aedurl}
\item 
\emph{Linux User Group of Rochester} \texttt{\aeeurl}
\item 
\emph{Nederlandse Linux Gebruikers Groep (Netherlands Linux Users Group or NLLGG)} \texttt{\aefurl}
\item 
\emph{North Texas Linux Users Group} \texttt{\aegurl}
\item 
\emph{Ottawa Canada Linux Users Group} \texttt{\aehurl}
\item 
\emph{Provence Linux Users Group} \texttt{\aeiurl}
\item 
\emph{Tokyo Linux Users Group} \texttt{\aejurl}
\item 
\emph{Turkish Linux User Group} \texttt{\aekurl}
\end{itemize}




Please let me know if your LUG uses the Internet in an important or
interesting way; I'd like this list to include your group.




\section{Practical suggestions}

Finally, I want to make some very practical, even mundane, suggestions
for anyone wanting to found, maintain, or grow a LUG.






\subsection{Founding a LUG}



\begin{itemize}
\item Determine the nearest existing LUG.
\item Announce your intentions on {\ttfamily comp.os.linux.announce} and on an appropriate regional hierarchy.
\item Announce your intention wherever computer users are in your area: bookstores, swap meets, cybercafes, colleges corporations, Internet service providers, etc.
\item Find friendly businesses or institutions in your area willing to help you form the LUG.
\item Form a mailing list or some means of communication among the people who express an interest in forming a LUG.
\item Ask key people specifically for help in spreading the word about your intention to form a LUG.
\item Solicit space on a Web server to put a few HTML pages together about the group.
\item Begin looking for a meeting place (in-person and/or videoconferencing).
\item Schedule an initial meeting.
\item Discuss at the initial meeting the goals for the LUG.
\end{itemize}





\subsection{Maintaining and growing a LUG}



\begin{itemize}
\item Make the barriers to LUG membership as low as possible.
\item Make the LUG's Web site a priority: Keep all information current, make it easy to find details about meetings (who, what, and where), and make contact information and feedback mechanisms prominent.
\item Install distributions for anyone who wants it.
\item Post flyers, messages, or handbills wherever computer users are in your area.
\item Secure dedicated leadership.
\item Follow Linus Torvalds's {\itshape benevolent dictator\/} model of leadership.
\item Take the big decisions to the members for a vote.  (Note:  This HOWTO's second maintainer feels volunteers who do needed LUG work deserve significantly greater consideration for their views than do other members.)
\item Start a mailing list devoted to technical support and ask the "gurus" to participate on it.
\item Schedule a mixture of advanced and basic, formal and informal, presentations.
\item Support the software development efforts of your members.
\item Find way to raise money without dues: for instance, selling GNU/Linux merchandise to your members and to others.
\item (Very optional:)  Consider securing formal legal standing for the group, such as incorporation or tax-exempt status.
\item Find out if your meeting place is restricting growth of the LUG.
\item Meet in conjunction with swap meets, computer shows, or other community events where computer users -- i.e., potential GNU/Linux users -- are likely to gather.
\item Elect formal leadership for the LUG as soon as practical: Some helpful officers might include President, Treasurer, Secretary, Meeting Host (general announcements, speaker introductions, opening and closing remarks, etc.), Publicity Coordinator (handles Usenet and e-mail postings, local publicity), and Program Coordinator (organises and schedules speakers at LUG meetings).
\item Provide ways for members and others to give feedback about the direction, goals, and strategies of the LUG.
\item Support GNU/Linux and free software / open source development efforts by donating Web space, or a mailing list.
\item Establish a Web site for relevant software.
\item Archive everything the LUG does for the Web site.
\item Solicit "door prizes" from GNU/Linux vendors, VARs, etc. to give away at meetings.
\item Give credit where due.
\item Submit your LUG's information to all the LUG lists.
\item Publicise your meetings on appropriate Usenet groups and in local computer publications and newspapers.
\item Compose promotional materials, like PostScript files, for instance, members can use to help publicise the LUG at workplaces, bookstores, computer stores, etc.
\item Make sure you know what LUG members want the LUG to do.
\item Release press releases to local media outlets about any unusual LUG events like an Installation Fest, Net Day, etc.
\item Use LUG resources and members to help local non-profit organisations and schools with their Information Technology needs.
\item Advocate the use of our OS enthusiastically but responsibly.
\item Play to LUG members' strengths.
\item Maintain good relations with vendors, VARs, developers, etc.
\item Identify and contact consultants in your area.
\item Network with the leaders of other LUGs in your area, state, region, or country to share experiences, tricks, and resources.
\item Keep LUG members advised on the state of software -- new kernels, bugs, fixes, patches, security advisories -- and the state of the GNU/Linux world at large -- new ports, trademark and licensing issues, where Torvalds is living and working, etc.
\item Notify the Linux Documentation Project -- and other pertinent sources of GNU/Linux information -- about the documentation the LUG produces: technical presentations, tutorials, local HOWTOs, etc.
\end{itemize}





\section{Legal and political issues}






\subsection{Organisational legal issues}

The case for formal LUG organisation can be debated:

{\itshape Pro:\/} Incorporation and recognised tax-exemption limits
liability and helps the group carry insurance.  It aids fundraising.
It avoids claims for tax on group income.

{\itshape Con:\/} Liability shouldn't be a problem for modestly careful
people.  (You're not doing skydiving, after all.)  Also, even
incorporated technical groups seldom carry liability insurance, and that
insurance is typically so narrow in coverage that almost nothing a LUG
does would be covered.  A corporate liability shield is little use for
such needs, either (limiting only the group's potential losses to the
equity stake of the owners, but conferring no immunity to anyone for 
deeds that person carries out).  Fundraising isn't needed for a group whose
activities needn't involve significant expenses.  (Dead-tree newsletters
are so 1980.)  Not needing a treasury, you avoid needing to argue over
it, file reports about it, or fear it being taxed away. Meeting space
can sometimes be gotten for free at ISPs, colleges, pizza parlours,
brewpubs, coffeehouses, computer-training firms, GNU/Linux-oriented
companies, hackerspaces, or other friendly institutions, and can
therefore be free of charge to the public.  No revenues and no expenses
means less need for organisation and concomitant hassles.

For whatever it's worth, this HOWTO's originator and second maintainer lean,
respectively, towards the pro and con sides of the issue -- but choose
your own poison:  If interested in formally organising your LUG, this
section will introduce you to some relevant issues.

{\bfseries Note:} this section should not be construed as competent legal
counsel. These issues require the expertise of competent legal
counsel; you should, before acting on any of the statements made in
this section, consult an attorney.






\subsubsection{Canada}

Thanks to \ifpdf
\href{mailto:cbbrowne@cbbrowne.com}{Chris Browne}%
\else
\onlynameurl{Chris Browne}%
\fi{}
 for the following comments about the Canadian situation.



The Canadian tax environment strongly parallels the US environment (for which,
see below), in that the "charitable organisation" status confers similar tax
advantages for donors over mere "not for profit" status, while
requiring that similar sorts of added paperwork be filed by the
"charity" with the tax authorities in order to attain and maintain
certified charity status.




\subsubsection{Germany}

Correspondent \ifpdf
\href{mailto:Thomas.Kappler@stud.uni-karlsruhe.de}{Thomas Kappler}%
\else
\onlynameurl{Thomas Kappler}%
\fi{}
 warns that the process of founding a non-profit entity in Germany
is a bit complicated, but comprehensively covered at 
http://www.buergergesellschaft.de/?id=106947 (unfortunately not archived; please
advise this HOWTO's maintainer of any current version or successor text).




\subsubsection{Sweden}

In Sweden, LUGs are not required to register, but then are regarded as 
clubs.  Registration with Skatteverket (national tax authority) offers 
two classification options:  non-profit organisation or "economical 
association". The latter is an organisation where the goal is to benefit 
its members economically, and as such is probably unsuitable, being 
traditionally used for collectives of companies, or building societies 
/ co-operative tenant-owners, and such).

Non-profit organisations in Sweden doesn't have specific laws to follow. 
Rather, general Swedish law applies: They can hire people and they can 
make profit. Generally they don't pay tax on their profits. (Profits 
stay in the organisation; unlike the case with "economical associations", 
members don't receive business proceeds.) To be able to do business, you
must register with Skatteverket to get an "organisation number", allowing 
the group to pay and get paid. Otherwise you will probably have to 
arrange business through a member in his/her individual capacity.  
It may then also be possible, after securing an organisation number to 
apply for government financial support.




\subsubsection{United States of America}

There are at least two different legal statuses a LUG in the USA may
attain:

\begin{enumerate}
\item incorporation as a non-profit entity
\item tax-exemption
\end{enumerate}


Although relevant statutes differ among states, most states
allow user groups to incorporate as non-profit entities. Benefits
of incorporation for a LUG include limitations of liability
of LUG members and volunteers ({\itshape but only in their passive roles
as member/shareholders, not as participants\/}), as well as 
limitation or even exemption from state corporate franchise taxes
(which, however, is highly unlikely to be a real concern -- see 
"Common Misconceptions Debunked", below).

While you should consult competent legal counsel before incorporating
your LUG as a non-profit, you can probably reduce your legal
fees by being acquainted with relevant issues before consulting
with an attorney. I recommend Kermit Burton's {\itshape 
\emph{Non-Lawyers' Non-Profit Corporation Kit} \texttt{\aelurl}
\/}.  (Be warned that this work has not been
updated since 1996, but its general guidelines are good.  The prior
1992 edition can be read for free at {\itshape 
\emph{Internet Archive} \texttt{\aemurl}
\/}.)

As for the second status, tax-exemption, this is not a legal status, so
much as an Internal Revenue Service judgement.  It's important to realise
non-profit incorporation {\bfseries does not} ensure that IRS will rule
your LUG tax-exempt.  It is quite possible for a non-profit corporation
to {\bfseries not} be tax-exempt.

IRS has a relatively simple document explaining the criteria
and process for tax-exemption. It is {\bfseries Publication 557:} {\itshape Tax-Exempt Status for Your Organization\/}, available as
an Acrobat file from the IRS's Web site. I strongly recommend
you read this document {\bfseries before} filing for non-profit incorporation.
While becoming a non-profit corporation cannot
ensure your LUG will be declared tax-exempt, some
incorporation methods will {\bfseries prevent} IRS from declaring your
LUG tax-exempt. {\itshape Tax-Exempt Status for Your Organization\/} clearly sets out necessary conditions for your LUG to be declared
tax-exempt.

Finally, there are resources available on the Internet for non-profit
and tax-exempt organisations. Some of the material is probably
relevant to your LUG.

Common Misconceptions Debunked:

\begin{itemize}
\item Incorporation and tax-exempt status are separate issues.  You don't have to be incorporated to get recognition of tax-exempt status (except it's required for one tax-exempt category, 501(c)(3)).  You don't have to be tax-exempt to be incorporated.  (Odds are, you honestly won't want either.  You just probably assume you do.)

\item The "liability shield" one can get from incorporating {\itshape doesn't 
protect volunteers from legal liability\/}.  All it does is prevent any 
plaintiffs from suing individual shareholders (LUG members, in this case) 
for tort damages {\itshape merely because they own the corporation\/}, if the 
corporation itself is alleged to have wronged the plaintiff.  Plaintiff's 
maximum haul in damages from suing the corporation is limited to the 
corporate net assets, in that one case.  However, volunteers are still
fully liable for any personal involvement they're alleged to have had.

\item Umbrella insurance coverage against tort liability (i.e., against civil litigation) for your volunteers almost certainly costs far too much for your group to afford (think \$2,500 each and every year in premium payouts, give or take, to buy \$1M in general liability insurance coverage -- which generally would cover only the corporation as a whole and its directors in the strict performance of their defined duties), if you can find it at all.

\item IRS recognition as a tax-exempt group doesn't mean donations to
your group necessarily become tax-deductible:  Automatic deductibility is
reserved to {\itshape charities only\/}, IRS category 501(c)(3), which must obey 
extremely stifling restrictions on group activities (e.g., it would then 
become illegal to host anti-DMCA events or support any other political
activity), and must meet exacting paperwork and auditing standards.  It's 
difficult to envision 501(c)(3) charity status actually making functional 
sense for any Linux group -- though one continually hears it recommended by
those who imagine being able to tell people their donations will be guaranteed tax deductible must justify any accompanying disadvantages.  Most LUGs would more logically file (if at all) for recognition as a "social and recreation club", category 
\emph{501(c)(7)} \texttt{\aenurl}
.

\item In any event, unless one wishes to become a registered charity to render incoming donations tax-deductible, there is {\itshape literally no point\/} in applying for  IRS  recognition of your small, informal Linux group under any of the Internal Revenue Code section 501(c) tax-exempt statuses, because IRS simply doesn't care about groups with annual gross revenues less than \$25,000, and 
\emph{doesn't want to hear from them} \texttt{\aeourl}
  (2010 update:  IRS now does require a very simple annual 
\emph{e-Postcard} \texttt{\aepurl}
 informational filing from all small non-profits, to keep their 501(c) certifications, but still doesn't want tax from them).

\item The 
\emph{Federal Volunteer Protection Act of 1997} \texttt{\aequrl}
 does 
\emph{not} \texttt{\aerurl}
, in fact, shield volunteers of Internal Revenue Code section 501(c)(3) charities from tort lawsuits.  At most, it furnishes some legal defences that can be raised during (expensive) civil litigation, with a large number of holes and limitations, and in most states will be denied unless the group also carries large amounts of (very expensive -- see above) liability insurance.  Also, unless the volunteer's duties are not very meticulously defined and monitored, and the alleged tort occurs strictly in the scope of those duties, there's no shield at all -- plus the litigated action must not involve a motor vehicle / aircraft / vessel requiring an operator's licence, nor may the volunteer be in violation of any state or Federal law, else again there's no shield at all.  (On the bright side, it's completely false, as often alleged, that the volunteer must be a member of the group, to be covered:  In fact, the Act clearly states that a volunteer may be anyone who performs defined services for a qualifying group and receives no compensation for that labour.)

\end{itemize}


As may be apparent from the above, a number of groups have, in the past, talked themselves into unjustifiable levels of bureaucratic strait-jacketing with no real benefit and serious ongoing disadvantages to their groups, because of misconceptions, careless errors, and tragically bad advice in the above areas.  In general, you should be slow to heed the counsel of amateur financial and tax advisors.  (This HOWTO's maintainer had past experience during his first career as a {\itshape professional\/} finance and tax advisor, but, if you need competent advice tailored to your situation, please have a consultation with someone currently working in that field.)




\subsection{Other legal issues}






\subsubsection{Bootlegging}

As a reminder, it's vital that offers or requests to copy
distribution-restricted proprietary software of any sort be heavily
discouraged anywhere in LUGs, and banned as off-topic from all GNU/Linux user
group on-line forums.  This is not generally even an issue -- much less
so than among proprietary-OS users -- but (e.g.) one LUG of my
acquaintance briefly used a single LUG-owned copy of PowerQuest's
Partition Magic on all NTFS-formatted machines brought to its
installfests for dual-boot OS installations, on a very dubious theory
of legality.

If it smells unlawful, it almost certainly is.  Beware.




\subsubsection{Antitrust}

It's healthy to discuss the consulting business in general in user
group forums, but for antitrust legal reasons it's a bad idea to get into 
"How much do you charge to do {[}foo]?" discussions, there.




\subsection{Software politics}


\emph{Chris Browne} \texttt{\aesurl}
 has the
following to say about the kinds of intra-LUG political dynamics that
often crop up (lightly edited and expanded by the HOWTO maintainer):




\subsubsection{People have different feelings about free / open-source software}

GNU/Linux users are a diverse bunch.  As soon as you try to put a lot of
them together, {\itshape some\/} problem issues can arise. Some, who are
nearly political radicals, believe all software, always, should be
"free".  Because Caldera charges quite a lot of money for its
distribution, and doesn't give all profits over to {\itshape (pick favorite
advocacy organisation)\/}, it must be "evil".  Ditto Red Hat or
SUSE.  Keep in mind that all three of these companies have made and
continue to make significant contributions to free / open-source software.

(HOWTO maintainer's note:  The above was a 1998 note, from before
Caldera Systems exited the GNU/Linux business, renamed itself to The SCO Group,
Inc., and launched a major copyright / contract / patent / trade-secret
lawsuit and PR campaign against GNU/Linux users.  My, those times do change.
Still, we're grateful to the Caldera Systems that {\itshape  was \/}, for
its gracious donation of hardware to help Alan Cox develop SMP kernel
support, for funding the development of RPM, and for its extensive past
kernel source contributions and work to combine the GNU/Linux and historical
Unix codebases.)



 
Others may figure they can find some way to highly exploit the
"freeness" of the GNU/Linux platform for fun and profit. Be aware that many
users of the BSD Unix variants consider {\itshape their\/} licences that
{\itshape do\/} permit companies to build "privatised" custom versions of
their kernels and C libraries preferable to the "enforced permanent
freeness" of the GPL as applied to the Linux kernel and GNU libc.  Do
not presume that all people promoting this sort of view are necessarily
greedy leeches.



 
If/when these people gather, disagreements can occur.



 
Leaders should be clear on the following facts:

\begin{itemize}
\item There are a lot of opinions about the GPL and other open-source
licences and how they work -- mostly misinformed.  It is easy to
misunderstand both the GPL and alternative licensing schemes.  Most
attempts at debating same are, at root,  pointless, ritualised symbolic
warfare among people who should know better.  In the rare event that
participants actually aspire to understand the subject, please direct
them to the OSI's "license-discuss" mailing list and the Debian
Project's "debian-legal" mailing list, where substantive analysis is
possible and encouraged.
\item  GNU/Linux benefits from contributions from many places, including
proprietary-software vendors, e.g., in the Linux kernel, X.org, and
gcc.
\item  Proprietary implies neither better nor horrible.
\end{itemize}




 
The main principle can be extended well beyond this; computer "holy
wars" have long been waged over endless battlegrounds, including 
GNU/Linux vs. other Unix variants vs. Microsoft OSes, the "IBM PC" vs.
sundry Motorola 68000-based systems, the 1970s' varied 8-bit systems 
against each other, KDE versus GNOME....



A wise LUG leader will seek to move past such differences, if only
because they're tedious.  LUG leaders ideally therefore will have thick
skins.




\subsubsection{Non-profit organisations and money don't mix terribly well.}

It is important to be careful with finances in any sort of non-profit.
In businesses, which focus on substantive profit, people are not
typically too worried about minor details such as alleged misspending of
{\itshape immaterial\/} sums.  The same cannot be said of non-profit
organisations.  Some people are involved for reasons of principle, and
devote inordinate attention to otherwise minor issues, an example of C.
Northcote Parkinson's 
\emph{Bike Shed Effect} \texttt{\aeturl}
.  LUG business
meetings' potential for wide participation correspondingly expands the
potential for exactly such inordinate attention.



As a result, it is probably preferable for there to {\itshape not\/} be any
LUG membership fee, as that provides a specific thing for which people
can reasonably demand accountability.  Fees not collected can't be
misused -- or squabbled over.



If there {\itshape is\/} a lot of money and/or other substantive property,
the user group must be accountable to members.



Any vital, growing group should have more than one active person.  In
troubled nonprofits, financial information is often tightly held by
someone who will not willingly relinquish monetary control. Ideally,
there should be {\itshape some\/} LUG duty rotation, including duties
involving financial control.



Regular useful financial reports should be made available to those
who wish them. A LUG maintaining official "charitable status"
for tax purposes must file at least annual financial reports
with the local tax authorities, which would represent a minimum
financial disclosure to members.



With the growth of GNU/Linux-based financial software, regular reports are
now quite practical.  With the growth of the Internet, it should even be
possible to publish these on the World-Wide Web.




\subsection{Elections, democracy, and turnover}

Governing your LUG democratically is absolutely vital -- if and
only if you believe it is.  I intend that remark somewhat less cynically
than it probably sounds, as I shall explain.

Tangible stakes at issue in LUG politics tend to be minuscule to the point of
comic opera:  There are typically no real assets. Differences of view 
can be resolved by either engineering around them with technology (the GNU/Linux-ey
solution) or by letting each camp run efforts in parallel. Moreover, even the
most militantly "democratic" LUGs typically field, like clockwork,
exactly as many candidates as there are offices to be filled -- not a
soul more.

It's tempting to mock such exercises as empty posturing, but such
is not (much) my intent.  Rather, I
mention them to point out something more significant:  Attracting and 
retaining key volunteers is vital to the group's success.  Anything that 
makes that happen is good.  It seems likely that the
"democratic" exercise stressed in some groups, substantive or not,
encourages participation, and gives those elected a sense of status,
legitimacy, and involvement.  Those are Good Things.  

Thus, if elections and formal structure help attract key
participants, use them.  If those deter participants, 
lose them.  If door-prizes and garage sales bring people in, do
door-prizes and garage sales.  Participation, as much as software, is
the lifeblood of your LUG. 

The reason I spoke of "key" volunteers, above, is because, inevitably, a
very few people will do almost all of the needed work.  It's just the
way things go, in volunteer groups. An anecdote may help illustrate my
point:  Towards the end of my long tenure as editor and typesetter of
San Francisco PC User Group's 40-page monthly magazine, I was repeatedly
urged to make magazine management more "democratic".  I finally replied
to the club president, "See that guy over there?  That's Ed, one of my
editorial staff.  Ed just proofread twelve articles for the current
issue.  So, I figure he gets twelve votes."  The president and other
club politicos were dismayed by my work-based recasting of their
democratic ideals: Their notion was that each biped should have an equal
say in editorial policy, regardless of ability to typeset or proofread,
or whether they had ever done {\itshape anything\/} to assist magazine
production. Although he looked quite unhappy about doing so, the
president dropped the subject.  I figured that, when it came right down
to it, he'd decide that the club needed people who got work done more
than they needed his brand of "democracy".

But we weren't quite done:  A month or so later, I was introduced to a 
"Publications Committee", who arrived with the intent of doing nothing but 
vote on matters of newsletter policy (i.e., issue "executive" orders to the 
volunteer production staff).  Their first shock came when I listened politely 
to their advice but then applied my editorial judgement as usual.  Much 
worse, though:  I also assigned them work, as part of my staff.  Almost 
all immediately lost interest. (Bossing around other people seemed likely 
to be fun; doing actual work was not.)

The point is that the widespread urge to vote on everything is at best
orthogonal to any desire to perform needed work; at worst, the former
serves as an excuse to compulsively meddle in others' performance
of the latter.

To sum up:  Have all the "democracy" that makes you happy, but watching after
the well-being of your key volunteers is what matters.  (To quote Candide, 
"We must cultivate our garden.")

Last, plan for your replacement:  If your LUG is a college student
group, and must go through a paperwork deathmarch every year to stay
accredited, make sure that and all other vital processes are documented,
so new LUG officers needn't figure everything out from scratch.  Think
of it as a systems-engineering problem:  You're trying to eliminate
single points of failure.

And what works for the guys in the next town may not work for your crowd:
Surprise!  The keys to this puzzle are still being sought.  So, please
experiment, and let me know what works for you, so I can tell others.
Have fun!




\section{About this document}


\subsection{Terms of use}



Copyright (C) 2003-2022, Rick Moen.  Copyright (C) 1997-1998 by Kendall Grant 
Clark. This document may be distributed under the terms set forth 
in the Creative Commons Attribution-ShareAlike 4.0 International licence at 
\emph{http://creativecommons.org/licenses/by-sa/4.0/} \texttt{\aeuurl}
, or, at your
option, any later version.




\subsection{New versions}

New versions of the Linux User Group HOWTO will be periodically
uploaded to various GNU/Linux Web sites, principally 
\emph{http://linuxmafia.com/lug/} \texttt{\aevurl}
 and
the 
\emph{Linux Documentation Project} \texttt{\aewurl}
.




\subsection{Please contribute to this HOWTO }

 
I welcome questions about and feedback on this document. Please send
them to me at \ifpdf
\href{mailto:rick@linuxmafia.com}{rick@linuxmafia.com}%
\else
\onlynameurl{rick@linuxmafia.com}%
\fi{}
. {\itshape I am especially interested in
hearing from LUG leaders around the world, especially outside the USA\/}. 
Please let me know of innovative or noteworthy things your group does,
meriting description here.




\subsection{Document history}

\begin{itemize}
\item 1.0: Released on 13 July 1997.
\item 1.1: Expanded online resources section.
\item 1.3: Added LUG support organisations and expanded the Legal and Organisational Issues section.
\item 1.3.1: General editing for clarity and conciseness.
\item 1.4: General editing, added new LUG resources.
\item 1.4.1: General editing for clarity.
\item 1.5: Added some resources, some discussion of LUG documentation, also general editing.
\item 1.5.1: Changed Web location for this document and author's e-mail address.
\item 1.5.2: New copyright notice and license.
\item 1.5.3: Miscellaneous edits and minor re-organisations.
\item 1.6: Added Chris Browne's material: GNU/Linux philanthropic
donations and LUG political considerations.
\item 1.6.1: Very minor additions.
\item 1.6.2: Minor corrections.
\item 1.6.3: Maintenance assumed by Rick Moen on 26 July 2003:  General 
initial touch-up, correction of broken URLs, etc.
\item 1.6.4: Further minor fixes and additions.
\item 1.6.5: More-extensive edits, added "Limits of advocacy",
added caveat about conflicting value systems in support contexts.  Added
more news sites, reordered examples of LUGs using Internet well.  General 
tightening of phrasing, greater brevity in places, tempering of the more
egregious boosterism.
\item 1.6.6: More small fixes, added Yahoo LUG list.
\item 1.6.7: Added formal-organisational pros/cons, "Elections,
democracy, and turnover" section, Web site suggestions, and link
to "Recipe for a Successful Linux User Group" essay. Fixed mis-tagged 
sections under "Legal and political issues".
\item 1.6.8: Fixed small glitches.  Rewrote section concerning
GNU/Linux news outlets; parts of sections concerning consultants, businesses,
and elections.
\item 1.6.9: Minor corrections.
\item 1.7.0: Caught up with GLUE membership having become free
of charge.
\item 1.7.1: Added a bunch more newly supported embedded CPUs.
\item 1.7.2: Added more on CPU support / ports (which section was always a bit silly, but I figure it might as well be exhaustive, correct, and {\itshape grandly\/} silly, if present at all); furnished matching URLs.  Added details about GNU/Linux in India, and {\itshape LINUX For You\/} magazine.  Expanded legal issues section.
\item 1.7.3: Added mention of Debian GNU/NetBSD to the CPU ports 
section.  Reorganised and further expanded the latter.  Recorded {\itshape Linux
Gazette\/}'s move to new hosting.  Added {\itshape LinuxFocus\/}.
\item 1.7.4: Added {\itshape LinuxWorld Magazine\/}, fixed URL of Recipe for
a Successful Linux User Group, which I moved.  Added Tux.Org and 
LinuxUserGroups.org as LUG support organisations.
\item 1.7.5: Added several more embedded CPUs to the supported list, implemented licence change (9 Jan 2004) to Creative Commons Attribution ShareAlike 1.0 or later after securing permission from Kendall Clark. (The HOWTO had previously been under the LDPL version posted at LDP's site in 1997, which by 2004 had become not only deprecated but also somewhat indeterminate as to content, because the licence had been edited in place with neither clear versioning nor a distinct URL for each revision.)
\item 1.7.6: Corrected addresses for TeX User Group in USA and
UK.  Added mention of C. Northcote Parkinson's Bike Shed Effect.  Other 
minor corrections.
\item 1.7.7: Added reference to the UK Linux User Groups site.
Added description of PingoS e.V.  Corrected URL for Thomas Kappler's 
e-mail address.  Added Volgograd LUG to Online Resources.
\item 1.7.8: Added Jerome Pinot's Linux architectures list,
used some data from it.  Added "I Linux User Group italiani".  
Corrected capitalisation of PingoS.  After securing permission from
Kendall Clark, added "or any later version" clause to document licence.
\item 1.7.9: Corrected India Linux link and added 
{\itshape LINUX For You\/}, per suggestions from Rohit Kumar.  Added Linux 
Foundation to list of candidates for receiving monetary support.  Made fixes 
to Red Hat LUG list (reincarnated as "Army of Friends" database), as suggested 
by Vincenzo Virgilio. Added LinuxHQ and O'Reilly LUG lists and FSF GNU User 
Groups list.  Added Wikipedia Category:LUGs page.  Dropped material about the 
GLUE site, which SSC, Inc.  tragically deleted in mid-2006 without allowing 
anyone a chance to adopt it.  Added kernel support for two more embedded chip 
families.  Substituted static mirrors for two (vanished) pages listing 
Linux kernel ports.  Dropped {\itshape LinuxWorld Magazine\/} (vanished).  
Removed references to getting help in founding LUGs from Red Hat User 
Group Program and Kara Pritchard's LinuxUserGroups.org (both vanished) 
and from lug.net (deactivated).  Added Swedish tax/regulatory details 
from Martin Karlsson.  Added analysis of issues surrounding incorporation, 
tax-exempt status, and insurance in the USA.  Found new URLs for a vast 
number of links.  Updated licence to Creative Commons BY-SA 3.0, to 
incorporate improvements.  Re-sorted country coverage into alphabetical 
order (a small gesture to further reduce US-centrism).
\item 1.8.0:  Corrected typos.  Improved some markup.  Expanded "Common Misconceptions Debunked" section to address recently popular errors about USA Volunteer Protection Act of 1997, civil liability, and IRS 501(c)(3) tax-exempt status.  Linked directly to the Act and an analysis page.  Furnished links for {\itshape Non-Lawyers' Non-Profit Corporation Kit\/}, for {\itshape DistroWatch Weekly\/}, and for the Raymond quotation.
\item 1.8.1:  Banished more typos.  (I blame society.)
\item 1.8.2:  Added more CPU ports.  APCUG changed Web sites.  Linux India's LUG list, lugww.counter.li.org (formerly Woven Goods's Linux Worldwide LUG list), Red Hat Army of Friends, LUG Webring, O'Reilly LinuxGroups, and {\itshape LINUX For You Magazine\/} (of India) LUG List vanished.  IDG moved the Raymond article (again).  LinuxFocus was revived via a CMS.  NewsForge was shut down by The Company Formerly Known as VA Linux.  Linux International Development Grant Fund program vanished.  Project Gutenberg moved to its own domain.  Colorado Linux Users and Enthusiasts moved.  Added mention of IRS e-Postcard for 501(c) non-profits.  Added Linux Users Group - Delhi.  Added New Zealand Linux Resource.  Added Project Runeberg.
\item 1.8.3:  Replacement ARM Linux Web site.  New URLs for many embedded and other systems.  SoftBlaze renamed to PetaLinux.  LUGs WorldWide Project, Linux Online -- User Groups, LinuxHQ User Groups, and New Zealand Linux Resource vanished.  Free Software Foundation GNU Users Groups got moved/renamed to LibrePlanet Group List.  New URLs for LinuxFormat and LinuxCounter.  PingoS e.V. vanished (though its SelfLinux project for hypertext tutorials in the German language persists).  Linux User Group of Singapore, St. Petersburg Linux User Group, and Svenska Linuxforeningen folded.
\item 1.8.4:  Linux.org (without its former Linux Online branding) has been rebuilt and offers a new LUG directory.
\item 1.8.5:  Replaced defunct NexentaOS with Dyson and other
IllumOS distributions.  Recorded new URL for APCUG.  Rewrote introduction
to list of supported hardware platforms to stress that this part isn't
serious documentation, but just intended to illustrate the breadth of
Linux's reach.  Corrected slightly incorrect statement about licensing of
Linux-based OSes.  Added new section Avoiding Burnout and Decline.
Added hackerspaces to list of possible meeting venues.  Added Lugslist, which
heroically rose in 2015 to explicitly compensate for collapse of the
much-missed lugww.counter.li.org and GLUE LUG lists.  Removed linux.org
LUG list, which Michael McLagen's Linux Online, Inc. deleted without notice.
Removed Yahoo Linux $>$ User Groups, vanished along with all the rest of
dir.yahoo.com.  Removed CLUE: the Canadian Linux Users' Exchange at 
www.linux.ca, which is down for rebuild but is promised to be back Q1
2016.  Corrected URL for Linux Australia's LUG list.  Removed {\itshape Linux
Gazette\/}, folded in 2011.  Removed {\itshape Linux Focus\/}, dormant
since 2010.  Updated name and Web site of {\itshape LINUX for You\/} magazine, which has now become {\itshape OpenSource For You\/}.
Added magazines {\itshape Full Circle\/}, {\itshape Linux Voice\/},
{\itshape easyLinux\/}, {\itshape LinuxUser\/}, and {\itshape Ubuntu User\/}.
Clarified where each magazine originates and detail national versions of
{\itshape Linux Magazine\/}.  Replaced reference to Win4Lin with one to CrossOver
Linux.  Added caution that Linux Consultants Guide is a decade out of
date.  Removed Open Source Applications Foundation, which seems to have
died shortly Mitch Kapor left it in 2008.  Added further clarification 
about limited benefits of incorporation and insurance.  Annotated LibrePlanet 
list as being FSF affiliates only.  Updated claim about how many LUGs 
exist worldwide.  Updated version of CC BY-SA licence applicable to this 
HOWTO from 3.0 to 4.0.  Included nod to realism that, no, the world at large
is never going to adopt the usage "GNU/Linux", but please indulge me anyway.
Linked in two appropriate places to separate Meetup.com rant.
\item 1.8.6:  Fixed new typos and punctuation errors, revised antiquated emphasis on ftp, and averted one quibble about tax-exempt status not requiring incorporation (the 501(c)(3) exception).  Politely disagreed with Kendall's implication that everyone deserves an equal say in "big decisions".
\item 1.8.7:  Added Software Freedom Conservancy and The Mozilla
Foundation.  Added Linux Links UserGroups list.
\item 1.8.8:  Dropped Dyson, a Debian-compatible Illumos
distribution that died in the 2010s.  Substituted OpenIndiana, the
flagship and most general-purpose Illumos distribution.  Among lists of
LUGs, deleted lugslist.com (vanished 2018), dmoz.org (replaced by
curlie.org on Mar. 17, 2018).  Added curlie.org, added back Kara Pritchard"s 
LinuxUserGroups.org (as it's back).  Altered URL for "Linux Links
UserGroups" to a search link.  Deleted antique line that "wide-scale commercial acceptance 
(of Linux) is only newly underway", because it's 2022, and, how quaint.  Truncated most 
of the advocacy section as irrelevant.  Changed Eric Raymond "Effective Advocacy" article 
link to point to Internet Archive.
Deleted {\itshape Linux User and Developer\/} magazine (UK; closed 2018). Amended URL for 
{\itshape Open Source for You\/} magazine.  Added {\itshape Free Software Magazine\/} and 
{\itshape Ubuntu User\/} magazine.  Removed 
Joshua Drake's Linux Consultants Guide that had moved to Command Prompt, Inc. but
has now vanished.  (Its predecessor incarnation, the Consultants HOWTO at tldp.org, 
is too crufty.)  Removed Linux Counter link (retired 2018 after a quarter-century, 
with a final estimate of worldwide Linux users of 91.9 million, as of August 2017).
Amended URL for Mozilla Foundation.  Sprinkled some mentions of Jitsi Meet/Zoom/etc.
Removed Colorado Linux Users and Enthusiasts, who seem to have been inactive since 
about 2014.  Replaced India Linux Users Group - Delhi, which seems to have been abandoned 
to a domain squatter, with India Linux Users Group Delhi.  Amended URL for Ottawa 
Canada Linux Users Group.  Eliminated section about site tux.org, which has been blank
or offline since Oct. 2021.  Delete mention of "ftp", because 2022.  Noted that 
all known copies of the external German text on the process of founding a non-profit have 
vanished.
\end{itemize}









\subsection{Acknowledgements}



I would like to give a big thank-you to Kendall Grant Clark for the 
initial versions of this document in 1997-1998, and for trusting me to take
over and renovate his creation starting in 2003.

Warm regards and thanks to 
\emph{Chris Browne} \texttt{\aexurl}
 for describing the situation with
non-profit and charitable groups in Canada, his thoughts on financial
donations as a way to participate in GNU/Linux and the free software and 
open-source software movements, and his ideas about the kinds of
political issues likely to arise within LUGs.

In addition, the following people have made helpful comments and
suggestions:

\begin{itemize}
\item Matthew Craig
\item Jeff Garvas
\item Greg Hankins
\item James Hertzler
\item Thomas Kappler
\item Martin Karlsson
\item Hugo van der Kooij
\item Rohit Kumar
\item David Lawyer
\item Charles Lindahl
\item Don Marti
\item Vincenzo Virgilio
\end{itemize}






\end{document}
