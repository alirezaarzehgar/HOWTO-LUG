\section{فعالیت‌های لاگ}
%\section{LUG activities}

تمرکز من در بخش قبل این بود که لاگ‌ها چه کاری می‌کنند و چه کاری باید بکنند.
اما در این بخش بر روی استراتژی‌های عملی تمرکز خواهیم کرد. دیگر بحث هدف
و ایده نیست. الزامات و رویدادهایی که برای لاگ الزامی است را شرح خواهیم داد.
%In the previous section I focused exclusively on what LUGs do and 
%should do. This section's focus shifts to practical strategies.

با وجود تغییراتی که در لاگ‌ها رخ داده است، لاگ‌ها دو کار اساسی را انجام می‌دهند:
اول، اعضا در بصورت فیزیکی و در جلسه ای همدیگر را ملاقات می‌کنند. دوم،
از طریق فضای‌مجازی باهم ارتباط می‌گیرند.
تقریبا تمام کارهایی که لاگ‌ها انجام می‌دهند به نوعی در جلسات حضوری و یا منابع آنلاین
قرار خواهند گرفت.
%There are, despite permutations of form, two basic things LUGs do:
%First, members meet in physical space; second, they communicate
%in cyberspace. Nearly everything LUGs do can be seen in terms of
%meetings and online resources.

\subsection{جلسات}
%\subsection{Meetings}

همانطور که قبل‌تر اشاره شد، جلسات حضوری مترادف با لاگ
(و اکثر گروه‌های کاربری دیگر) می‌باشند. لاگ می‌تواند انواع مختلفی از جلسات را داشته باشد.
در زیر برخی از انواع جلسات ذکر شده است:
%As I said above, physical meetings are synonymous with LUGs (and 
%most user groups).  LUGs have these kinds of meetings:

\begin{itemize}
\item احتماعی
%\item social

\item ارائه فنی
%\item technical presentations

\item گروه‌های مباحثه غیر رسمی
%\item informal discussion groups

\item ویدئوکنفرانس آنلاین
%\item online videoconferencing (Jitsi Meet, Zoom, etc.)

\item کسب‌وکار گروه کاربری
%\item user group business

\item نصب گنو/لینوکس
%\item GNU/Linux installation

\item پیکربندی و رفع اشکال
%\item configuration and bug-squashing
\end{itemize}

لاگ‌ها در این جلسات چه می‌کنند؟
%What do LUGs do at these meetings?

\begin{itemize}
\item
نصب توزیع‌های گنو/لینوکس برای تازه‌کاران و نا آشنایان.
%\item Install distributions for newcomers and strangers.

\item
آموزش اعضا راجع به گنو/لینوکس.
%\item Teach members about GNU/Linux.

\item
مقایسه کردن گنو/لینوکس با دیگر سیستم‌عامل‌ها.
%\item Compare GNU/Linux to other operating systems.

\item
آموزش اعضا راجع به نرم‌افزارهای کاربردی.
%\item Teach members about application software.

\item
گفت‌وگو راجع به حمایت و ترویج.
%\item Discuss advocacy.

\item
گفت‌وگو راجع به جنبش نرم‌افزار آزاد/متن‌باز.
%\item Discuss the free software / open-source movement.

\item
گفت‌وگو راجع به کسب‌وکارهای گروه.
%\item Discuss user group business.

\item
دورهمی به‌همراه پذیرایی با غذا یا نوشیدنی، جشن و شادی.
%\item Eat, drink, and be merry.
\end{itemize}

\subsection{منابع آنلاین}
%\subsection{Online resources}

ظهور تجارت اینترنت شدیدا با ظهور گنو/لینوکس همزمان بود. درواقع
گنو/لینوکس مدیون ظهور اینترنت است. اینترنت همیشه برای توسعه امر مهمی بوده است.
لاگ‌ها هم از این امر مستثنا نبودند و برای توسعه خود نیاز به اینترنت داشتند.
اکثر آنها اگر یک وبسایت کامل هم نداشتند، حداقل یک صفحه وب برای معرفی خود داشتند.
درواقع من حتی مطمئن نیستم آیا راه دیگری برای پیدا کردن لاگ به‌غیر از بررسی وبسایت
آن وجود دارد یا خیر.
%The commercial rise of the Internet coincided roughly with that of
%GNU/Linux; the latter owes something to the former. The Net has always been
%important to development. LUGs are no different: Most have Web
%pages, if not whole Web sites. In fact, I'm not sure how else to find a
%LUG, but to check the Web.

پس قابل درک است که لاگ که هر امکاناتی که اینترنت ارائه می‌دهد را استفاده کند.
نمونه آن ایجاد وبسایت‌ها و برگزاری جلسات در جیتسی، زوم و غیره است.
همچنین لاگ‌ها در لیست‌های پستی، ویکی‌ها، ایمیل، انجمن‌های گفت‌وگو تحت وب، نت‌نیوز و غیره
فعالیت دارند.
همانطور که دنیای تجارت متوجه شده است، اینترنت یک راه کارآمد برای تبلیغ، اطلاع‌رسانی،
آموزش و حتی فروش است.
دلیل دیگری که لاگ‌ها به‌طور گسترده از تکنولوژی‌های اینترنتی استفاده می‌کنند این است که
گنو/لینوکس ذاتابرای فراهم‌سازی یک پلفرم پایدار و غنی برای
\tfftnt{استقرار}{deploy}
دقیقا همین تکنولوژی‌های نام‌برده وجود دارد و توسعه می‌یابد.
بنابراین راه اندازی یک وبسایت برای لاگ مزایای زیادی دارد. این‌کار نه‌تنها وجود لاگ را
تبلیغ و اطلاع رسانی کرده و به سازماندهی اعضا کمک می‌کند، بلکه اعضای لاگ به‌واسطه
استقرار این تکنولوژی‌ها در رابطه با آنها مطالبی را می‌آموزند و گنو/لینوکس را در محیط
عملیاتی و کار خواهند دید.
%It makes sense, then, for a LUG to make use of whatever Internet
%technologies they can: Web sites, Jitsi Meet/Zoom/etc., mailing lists, 
%wikis, e-mail, Web discussion forums, netnews, etc. As the world of commerce is
%discovering, the Net is an effective way to advertise, inform, educate,
%and even sell. The other reason LUGs make extensive use of Internet
%technology is that the very essence of GNU/Linux is to {\itshape provide\/} a
%stable and rich platform to deploy these technologies. So,
%not only do LUGs benefit from, say, establishment of a Web site,
%because it advertises their existence and helps organise members,
%but, in deploying these technologies, LUG members 
%learn about them and see GNU/Linux at work.

در میان منابع اینترنتی که ذکر شدند مانند لیست پستی، جیتسی و غیره، یک وبسایت
با نگهداری و پشتیبانی مناسب قطعا از ضروریات و الزامات است. مقاله من با نام
\fftnt{دستورالعمل برای یک گروه کاربری لینوکس موفق}{http://linuxmafia.com/faq/Linux_PR/newlug.html}،
به همین دلیل زمان قابل توجهی را برای بحث راجع به مشکلات وبسایت صرف کرده است.
الزاماتی که در آن مقاله برای ایجاد لاگ نوشته ام در اینجا نقل‌قول می‌کنم.
به‌طور کلی برای راه‌اندازی لاگ شما به موارد زیر نیاز دارید:

%Arguably, a well-maintained Web site is the one must-have, among those
%Internet resources.  My essay
%\emph{Recipe for a Successful Linux User Group} \texttt{http://linuxmafia.com/faq/Linux_PR/newlug.html}
%, for that reason,
%spends considerable time discussing Web issues.  Quoting it (in outline form):

\begin{itemize}
\item
شما به یک صفحه وب نیاز دارید.
%\item You need a Web page.
\item
صفحه وب شما نیاز به یک \lr{URL} قابل‌قبول دارد.
%\item Your Web page needs a reasonable URL.
\item
شما نیاز به محلی برای ملاقات‌ها و جلسات دارید.
%\item You need a regular meeting location.
\item
شما نیاز به زمان زمان برای ملاقات و جلسه دارید.
%\item You need a regular meeting time.
\item
نباید بین زمان جلسات تداخلی وجود داشته باشد.
%\item You need to avoid meeting-time conflicts.
\item
باید مطمئن شوید که جلسات همانطور که برنامه‌ریزی شده اند و بدون هیچ نقصی تشکیل خواهند شد.
%\item You need to make sure that meetings happen as advertised, without fail.
\item
شما نیاز به یک گروه از چندین مشتاق برای پیش‌بردن لاگ نیاز دارید. (تیم اجرایی)
%\item You need a core of several enthusiasts.
\item
تیم اجرایی شما نیاز به راه ارتباطی متفاوتی از دیگر اعضا دارد.
%\item Your core volunteers need out-of-band methods of communication.
\item
شما باید اطلاعات لاگ خود را در لیست اصلی لاگ‌ها اضافه کنید و آنها را به درستی نگه‌دارید.
%\item You need to get on the main lists of LUGs, and keep your entries accurate.
\item
شما باید دسترسی موردنیاز برای نگهداری صفحه وب لاگ راداشته باشید.
%\item You must have login access to maintain your Web pages, as needed.
\item
صفحه وب لاگ را طوری طراحی کنید که کمتر نیاز به نگهداری داشته باشد و حتی در صورتی که
نگهداری آن به تعویق افتاد باز هم عملکرد قابل‌قبولی داشته باشد.
%\item Design your Web page to be forgiving of deferred maintenance.
\item
زمانی که تاریخ رویدادی را اعلام می‌کنید، همیشه مشخص کنید چه روزی در هفته می‌شود.
%\item Always include the day of the week, when you cite event dates. Always check that day of the week, first, using cal.
\item
اطلاعات کلیدی و مبتنی بر زمان (که با گذشت زمان اعتبارشان را از دست می‌دهند)
را به صورت برجسته و در بالای صفحه اصلی وب قرار دهید.
%\item Place time-sensitive and key information prominently near the top of your main Web page.
\item
مپ و لوکیشن محل برگزاری هر رویداد را در صفحه وب مشخص کنید.
%\item Include maps and directions to your events.
\item
در صفحه اصلی تاکید کنید که ملاقات شما رایگان خواهد بود و شرکت عموم مردم آزاد می‌باشد
(اگر اینطور است).
%\item Emphasise on your main page that your meeting will be free of charge and open to the public (if it is).
\item
در صورت نیاز برای بعضی از رویدادها لینک ثبت‌نام بگذارید تا از تعداد شرکت‌کنندگان مطلع شوید.
%\item You'll want to include an RSVP "mailto" hyperlink, on some events.
\item
به صفحه‌های دیگر ارجاع دهید.
%\item Use referral pages.
\item
مطمئن شوید که هر صفحه حاوی تاریخ آخرین ویرایش و شخصی که آن را نگهداری می‌کند است.
%\item Make sure every page has a revision date and maintainer link.
\item
صحت تمام لینک‌ها را در طول زمان بررسی کنید و مطمئن شوید هنوز هم اعتبار دارند.
%\item Check all links, at intervals.
\item
ممکن است درنظر داشته باشید که یک لیست پستی برای لاگ ایجاد بکنید.
%\item You may want to consider establishing a LUG mailing list.
\item
نیازی نیست حتما در زمینه کسب‌وکار ارائه خدمات اینترنتی فعالیتی داشته باشید.
%\item You don't need to be in the Internet Service Provider business.
\item
نیازی نیست کار دیگری خارج از اهداف لاگ انجام بدهید.
%\item Don't go into any other business, either.
\item
ایده‌های خود را بکار بگیرید و در عمل آنها را نشان دهید.
(کارهای لاگ را به‌کمک گنو/لینوکس انجام دهید)
%\item Walk the walk. (Do the LUG's computing on GNU/Linux.)
\end{itemize}

این مقاله تاحدی این راهنما را کامل می‌کند (و حدودا با آن همپوشانی دارد).
%That essay partly supplements (and partly overlaps) this HOWTO.

برخی لاگ‌ها که در زیر فهرست شده‌اند به‌خوبی از اینترنت و تکنولوژی‌های آن استفاده می‌کنند:
%Some LUGs using the Internet effectively:

\begin{itemize}
\item
\fftnt{مشتاقان لینوکس آتلانتا}{http://ale.org/}
%\emph{Atlanta Linux Enthusiasts} \texttt{http://ale.org/}
\item
\fftnt{لینوکس و یونیکس بوستون}{http://www.blu.org/}
%\emph{Boston Linux and Unix} \texttt{http://www.blu.org/}
\item
\fftnt{گروه کاربران لینوکس دوسلدورف}{http://www.dlug.de/}
%\emph{Dusseldorfer Linux Users Group} \texttt{http://www.dlug.de/}
\item
\fftnt{گروه کاربران لینوکس دهلی هند}{https://linuxdelhi.org/}
%\emph{India Linux Users Group Delhi} \texttt{https://linuxdelhi.org/}
\item
\fftnt{گروه کاربران لینوکس اسرائیل}{http://www.iglu.org.il/}
%\emph{Israeli Group of Linux Users} \texttt{http://www.iglu.org.il/}
\item
\fftnt{گروه کاربران لینوکس کره}{http://www.lug.or.kr/}
%\emph{Korean Linux Users Group} \texttt{http://www.lug.or.kr/}
\item
\fftnt{لینوکس مکزیکو}{http://cofradia.org/}
%\emph{Linux Mexico (La Cofradia Digital)} \texttt{http://cofradia.org/}
\item
\fftnt{گروه کاربری لینوکس استرالیا}{http://www.luga.at/}
%\emph{Linux User Group Austria} \texttt{http://www.luga.at/}
\item
\fftnt{گروه کاربری لینوکس دیویس}{http://www.lugod.org/}
%\emph{Linux User Group of Davis} \texttt{http://www.lugod.org/}
\item
\fftnt{گروه کاربری لینوکس روچستر}{http://www.lugor.org/}
%\emph{Linux User Group of Rochester} \texttt{http://www.lugor.org/}
\item
\fftnt{گروه کاربران لینوکس ندرلند یا \lr{NLLGG}}{http://www.nllgg.nl/}
%\emph{Nederlandse Linux Gebruikers Groep (Netherlands Linux Users Group or NLLGG)} \texttt{http://www.nllgg.nl/}
\item
\fftnt{گروه کاربران لینوکس تگزاس شمالی}{http://www.ntlug.org/}
%\emph{North Texas Linux Users Group} \texttt{http://www.ntlug.org/}
\item
\fftnt{گروه کاربران لینوکس اتاوا در کانادا}{https://wiki.linux-ottawa.org/doku.php}
%\emph{Ottawa Canada Linux Users Group} \texttt{https://wiki.linux-ottawa.org/doku.php}
\item
\fftnt{گروه کاربران لینوکس پرووانس}{http://plugfr.org/}
%\emph{Provence Linux Users Group} \texttt{http://plugfr.org/}
\item
\fftnt{گروه کاربران لینوکس توکیو}{http://www.tlug.jp/}
%\emph{Tokyo Linux Users Group} \texttt{http://www.tlug.jp/}
\item
\fftnt{گروه کاربری لینوکس ترکیه}{http://www.linux.org.tr/}
%\emph{Turkish Linux User Group} \texttt{http://www.linux.org.tr/}
\end{itemize}

آیا اینترنت و تکنولوژی‌های مبتنی بر آن در لاگ شما هم اهمیت دارد ؟ اجازه بدهید
من از این موضوع خبر داشته باشم و اگر لاگ شما هم از این تکنولوژی‌ها استفاده می‌کند
مایلیم که آن را به این لیست اضافه کنیم.
%Please let me know if your LUG uses the Internet in an important or
%interesting way; I'd like this list to include your group.

\subsubsection{منابع آنلاین لاگ‌های ایران}
لاگ‌های محلی در ایران اغلب در تلگرام فعالیت دارند و انجمن‌های تلگرامی آنها معمولا
فعال‌تر، پایدارتر و پرطرفدارتر می‌باشند. در این بخش مشخصات لاگ‌های ایران را قرار می‌دهیم.

\begin{itemize}
%name,website,channel,group,git
\irluginfo{لاگ مشهد}{}{https://t.me/mashhadlug}{https://t.me/+8PnN0irrwzk3Zjk8}{https://github.com/mashhadlug}
\irluginfo{لاگ دانشگاه فردوسی مشهد}{}{https://t.me/fum_lug}{https://t.me/ferdowsilug}{}
\irluginfo{لاگ ارومیه}{https://urumlug.ir/?ref=unrivaled.ir}{https://t.me/urumlug}{https://t.me/Urumlug_group}{}
\irluginfo{لاگ البرز}{}{https://t.me/alborzlug}{}{}
\irluginfo{لاگ استهبان}{}{}{https://t.me/EstahbanLug}{}
\irluginfo{لاگ بندرعباس}{}{https://t.me/bndlugg}{https://t.me/bndluggg}{}
\irluginfo{لاگ پرشین}{http://persianlug.com/}{https://t.me/persianlug}{https://t.me/persianlug_chat/125}{}
\irluginfo{لاگ تهران}{https://tehlug.org/?ref=unrivaled.ir}{https://t.me/TehranLUG}{https://t.me/+hSwSZQG0oophYzg0}{https://github.com/tehlug}
\irluginfo{لاگ جامعه لینوکسی شیراز}{https://shirazlinuxcommunity.ir/?ref=unrivaled.ir}{https://t.me/shirazlinuxcommunity}{https://t.me/shirazlinuxG}{}
\irluginfo{دورهمی گنو}{}{https://t.me/gnumeetup}{}{}
\irluginfo{ریلاگ}{}{https://t.me/relug_info}{https://t.me/relugs}{}
\irluginfo{لاگ زنجان}{https://zanjanlug.ir/}{}{https://t.me/zanjanlug}{}
\irluginfo{لاگ شیراز}{https://shirazlug.ir/?ref=unrivaled.ir}{https://t.me/shirazlug}{https://t.me/lugshiraz}{https://framagit.org/shirazlug}
\irluginfo{لاگ کرج}{https://karajlug.org/}{}{https://t.me/+QRReI-w2ZJ8V5r4l}{}
\irluginfo{لاگ قزوین}{}{https://t.me/qazvin_lug}{https://t.me/joinchat/BNxsFT47uBsVdWB5RXr1ug}{}
\irluginfo{لاگ کرمان}{}{}{https://t.me/kermanlinux/1}{}
\irluginfo{لاگ گیلان}{}{}{https://t.me/+P2Pv304Nh19E_uiD}{}
\end{itemize}

