

\section{Practical suggestions}

Finally, I want to make some very practical, even mundane, suggestions
for anyone wanting to found, maintain, or grow a LUG.






\subsection{Founding a LUG}



\begin{itemize}
\item Determine the nearest existing LUG.
\item Announce your intentions on {\ttfamily comp.os.linux.announce} and on an appropriate regional hierarchy.
\item Announce your intention wherever computer users are in your area: bookstores, swap meets, cybercafes, colleges corporations, Internet service providers, etc.
\item Find friendly businesses or institutions in your area willing to help you form the LUG.
\item Form a mailing list or some means of communication among the people who express an interest in forming a LUG.
\item Ask key people specifically for help in spreading the word about your intention to form a LUG.
\item Solicit space on a Web server to put a few HTML pages together about the group.
\item Begin looking for a meeting place (in-person and/or videoconferencing).
\item Schedule an initial meeting.
\item Discuss at the initial meeting the goals for the LUG.
\end{itemize}





\subsection{Maintaining and growing a LUG}



\begin{itemize}
\item Make the barriers to LUG membership as low as possible.
\item Make the LUG's Web site a priority: Keep all information current, make it easy to find details about meetings (who, what, and where), and make contact information and feedback mechanisms prominent.
\item Install distributions for anyone who wants it.
\item Post flyers, messages, or handbills wherever computer users are in your area.
\item Secure dedicated leadership.
\item Follow Linus Torvalds's {\itshape benevolent dictator\/} model of leadership.
\item Take the big decisions to the members for a vote.  (Note:  This HOWTO's second maintainer feels volunteers who do needed LUG work deserve significantly greater consideration for their views than do other members.)
\item Start a mailing list devoted to technical support and ask the "gurus" to participate on it.
\item Schedule a mixture of advanced and basic, formal and informal, presentations.
\item Support the software development efforts of your members.
\item Find way to raise money without dues: for instance, selling GNU/Linux merchandise to your members and to others.
\item (Very optional:)  Consider securing formal legal standing for the group, such as incorporation or tax-exempt status.
\item Find out if your meeting place is restricting growth of the LUG.
\item Meet in conjunction with swap meets, computer shows, or other community events where computer users -- i.e., potential GNU/Linux users -- are likely to gather.
\item Elect formal leadership for the LUG as soon as practical: Some helpful officers might include President, Treasurer, Secretary, Meeting Host (general announcements, speaker introductions, opening and closing remarks, etc.), Publicity Coordinator (handles Usenet and e-mail postings, local publicity), and Program Coordinator (organises and schedules speakers at LUG meetings).
\item Provide ways for members and others to give feedback about the direction, goals, and strategies of the LUG.
\item Support GNU/Linux and free software / open source development efforts by donating Web space, or a mailing list.
\item Establish a Web site for relevant software.
\item Archive everything the LUG does for the Web site.
\item Solicit "door prizes" from GNU/Linux vendors, VARs, etc. to give away at meetings.
\item Give credit where due.
\item Submit your LUG's information to all the LUG lists.
\item Publicise your meetings on appropriate Usenet groups and in local computer publications and newspapers.
\item Compose promotional materials, like PostScript files, for instance, members can use to help publicise the LUG at workplaces, bookstores, computer stores, etc.
\item Make sure you know what LUG members want the LUG to do.
\item Release press releases to local media outlets about any unusual LUG events like an Installation Fest, Net Day, etc.
\item Use LUG resources and members to help local non-profit organisations and schools with their Information Technology needs.
\item Advocate the use of our OS enthusiastically but responsibly.
\item Play to LUG members' strengths.
\item Maintain good relations with vendors, VARs, developers, etc.
\item Identify and contact consultants in your area.
\item Network with the leaders of other LUGs in your area, state, region, or country to share experiences, tricks, and resources.
\item Keep LUG members advised on the state of software -- new kernels, bugs, fixes, patches, security advisories -- and the state of the GNU/Linux world at large -- new ports, trademark and licensing issues, where Torvalds is living and working, etc.
\item Notify the Linux Documentation Project -- and other pertinent sources of GNU/Linux information -- about the documentation the LUG produces: technical presentations, tutorials, local HOWTOs, etc.
\end{itemize}




