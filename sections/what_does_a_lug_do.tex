\section{یک لاگ چه کارهایی انجام می‌دهد؟}
%\section{What does a LUG do?}

به یاد داشته باشید که گنو/لینوکس از بروکراسی و کنترل مرکزی آزاد است.
لاگ‌ها هم به همین شکل هستند. اهداف لاگ به اندازه مکان‌هایشان متنوع است
و هیچ طرح و برنامه کلی برای لاگ وجود ندارد. البته که این مستند هم
قرار نیست چنین طرحی را ایجاد کند.
%LUGs' goals are as varied as their locales. There is no LUG master
%plan, nor will this document supply one. Remember: GNU/Linux is free from
%bureaucracy and centralised control; so are LUGs.

اما میتوان مجموعه ای از هدف های یک لاگ را مشخص کرد:
%It is possible, however, to identify a core set of goals for a 
%LUG:

\begin{itemize}
\item ترویج \lr{(advocacy)}
%\item advocacy
\item آموزش \lr{(education)}
%\item education
\item پشتیبانی \lr{(support)}
%\item support
\item اجتماعی‌شدن \lr{(socialising)}
%\item socialising
\end{itemize}

هر لاگ ترکیبی از این اهداف و دیگر اهداف منحصر به فرد و مرتبط
به نیاز های اعضا می‌باشد.
%Each LUG combines these and other goals uniquely, according to its
%membership's needs.

\subsection{ترویج گنو/لینوکس}
%\subsection{GNU/Linux advocacy}


تمایل به ترویج \lr{(advocacy)}
\footnote{
معنای دقیق \lr{advocacy} ترویج نیست و
معادل های مختلفی برای ترجمه آن می‌توان استفاده کرد که البته
هیچکدام از آنها نمی‌تواند بطور دقیق معنای آنرا برساند. \lr{advocacy}
در لغت می‌تواند به معنای طرفداری، ترویج، مدافعه و وکالت باشد.
بطور کلی \lr{advocacy} به معنای پشتیبانی از یک ایده می‌باشد.
درطول این مستند معادل هایی مانند ترویج یا طرفداری را خواهید دید
و منظور ما از تمام آنها \lr{advocacy} خواهد بود.
}
گنو/لینوکس به‌طور گسترده‌ای احساس می‌شود.
زمانی که شما چیزی را پیدا می‌کنید که به خوبی کار می‌کند،
دوست دارید به هرتعداد آدمی که می‌توانید آن را معرفی کنید.
همانطور که نوشتن یک ریویو مثبت از یک روزنامه‌نگار کامپیوتری
راجع به گنو/لینوکس برای این جنبش مفید است، معرفی گنو/لینوکس
به دوستان، همکاران، کارمندان و حتی کارفرمایان هم مفیداست.
%The urge to advocate the use of GNU/Linux is widely felt. When you find
%something that works well, you want to tell as many people as you can.
%While it is certainly beneficial to the movement, each and every time a
%computer journalist writes a positive review of GNU/Linux, it is also
%beneficial every time satisfied GNU/Linux users brief their friends,
%colleagues, employees, or employers.

بعنوان کاربر همیشه باید هوشیار باشیم که طوری از گنو/لینوکس
طرفداری کنیم و آن را ترویج دهیم که بازتاب مثبت بر محصول،
سازنده ها، توسعه دهنده ها و حتی سایر کاربرها داشته باشد.
توجه داشته باشید که طرفداری هم می‌تواند سازنده باشد و هم می‌تواند
صرفا بهانه‌گیری منفی و غیر سازنده باشد.
\fftnt{راهنمای ترویج لینوکس}{http://www.tldp.org/HOWTO/Advocacy.html}
در
\fftnt{پروژه مستندسازی لینوکس}{http://www.tldp.org/}
موجود می‌باشد و پیشنهادهای بسیار کمک کننده ای ارائه می‌دهد.
پروژه دیگری مشابه پروژه مستندسازی لینوکس، به نام 
\ftnt{Linuxmanship}{http://zgp.org/~dmarti/linuxmanship/}
وجود دارد که کار \lr{Don Marti} می‌باشد.
%There is effective advocacy, and there is ineffective carping: As 
%users, we must be constantly vigilant to advocate GNU/Linux in such a way as
%to reflect positively on the product, its creators and developers, and
%our fellow users. The 
%\emph{Linux Advocacy HOWTO} \texttt{http://www.tldp.org/HOWTO/Advocacy.html}
%, available at the 

%\emph{Linux Documentation Project} \texttt{http://www.tldp.org/}
%, 
%gives some helpful suggestions, as does Don Marti's excellent 

%\emph{Linuxmanship} \texttt{http://zgp.org/~dmarti/linuxmanship/}
% essay.

در طول‌عمر زیاد این مستند گنو/لینوکس تقریبا موفق بوده است و به‌همین دلیل،
نگهدارنده این مستند حجم زیادی از این قسمت مستند را حذف کرده. درواقع
محبوبیت و جایگاه فعلی گنو/لینوکس طوری است که ترویج و طرفداری آن فاقد اهمیت می‌باشد.
%Over the long life of this HOWTO, GNU/Linux more or less won the 
%day, so the HOWTO maintainer has deleted much of this section, and 
%advocacy in his view has become, in his view, overwhelmingly irrelevant.

\subsection{محدودیت های ترویج (\lr{advocacy})}
%\subsection{The limits of advocacy}

طرفداری می‌تواند بدون هدف باشد. طرفداری می‌تواند اشتباه و دارای اثر عکس باشد.
طرفداری می‌تواند در ابتدا به‌سادگی نامناسب باشد. این موضوع مستحق تفکر و بحث دقیق است.
زیرا اگر به ترویج و طرفداری به اشتباه انجام گیرد می‌تواند تنها به معنای اتلاف وقت باشد.
%Advocacy can be mis-aimed; advocacy can go wrong and be
%counterproductive; advocacy can be simply inappropriate in the first
%place.  The matter merits careful thought, to avoid wasted time or
%worse.

خیلی از تلاش ها برای طرفداری به‌طرز وحشتناکی با شکست مواجه می‌شوند
زیرا این طرفداری نتوانسته است به خواسته ها و نیاز های طرف دیگر داستان
گوش بدهد. (همانطور که \lr{Eric S. Raymond}
\fftnt{می‌گوید}{http://web.archive.org/web/20120131000749/http://www.itworld.com/print/36449}
،
\tfftnt{
به منافع و علاقمندی های مشتری خود توجه کنید، نه خودتان.
}{Appeal to the prospect's interests and values, not to yours}
اگر شخصی ستاپ سیستم‌عامل انحصاری که درحال حاضر دارد را می‌پسندد
و آنرا دوست دارد، ترویج تنها وقت شما و او را هدر می‌دهد.
اگر نیازهای او کاملا با \lr{MS-Project}، \lr{MS-Visio} و
\tfftnt{گروه‌افزار اوت‌لوک/استک‌اکسچنج}{Outlook/Exchange groupware}
مطابق است، تلاش برای فروختن چیزی که او واقعا نمی‌خواهد 
تنها باعث آزار او خواهد شد و هیچکس از این رفتار لذت نمی‌برد.
(صرف‌نظر از اینکه آیا واقعا آن نیازها را دارد یا خیر).
تلاش و انرژی خودرا برای فرد دیگری که پذیرای آن باشد نگه‌‌دارید.
%Many attempts at advocacy fail ignominiously because the advocate fails
%to listen to what the other party feels she wants or needs.  (As Eric
%S. Raymond 
%\emph{says} \texttt{http://web.archive.org/web/20120131000749/http://www.itworld.com/print/36449}
%, 
%"Appeal to the prospect's interests and values, not to
%yours.") If that person wants exactly the proprietary-OS setup she
%already has, then advocacy wastes your time and hers.  If her
%stated requirements equate exactly to MS-Project, MS-Visio, and
%Outlook/Exchange groupware, then trying to "sell" her what she doesn't
%want will only annoy everyone (regardless of whether her requirements
%list is real or artificial).  Save your effort for someone more
%receptive.  

در این راستا بخاطر بسپارید که برای خیلی از مردم، شاید بیشترشان
یک "مدافع" درواقع یک فروشنده تصور می‌شود. بنابراین طبیعی است
که بجای عادلانه گوش‌دادن به شما، درمقابل حرف هایتان مقاوت کند.
آنها هرگز کسی را ندیده اند که کسی بدون درنظر داشتن منفعت، به رایگان
نرم‌افزاری را دراختیارشان قرار بدهد. از همین رو فکر می‌کنند این طرفداری
بهرحال باید سودی برای شما داشته باشد. به‌همین خاطر گوش دادن به صحبت های
شما را هم یک لطف شخصی می‌دانند که در حق شما داشتند. چه برسه به توصیه‌های
شما هم عمل کنند.
%Along those lines, bear in mind that, for many people, perhaps most, an
%"advocate" is perceived as a salesman, and thus classified as someone to
%resist rather than listen to fairly. They've never heard of someone
%urging them to adopt a piece of software without
%benefiting materially, so they assume there must be something in
%it for you and will push back, and
%act as if they're doing you a personal favour to even listen, let alone
%try your recommendations.  

پیشنهاد می‌کنم فورا بحث را قابل‌لمس و منطقی کنید. به این اشاره‌کنید
که این سیاست‌های نرم‌افزاری در بلندمدت براساس منفعت افراد خواهند بود.
همچنین شخص شما هیچ نفعی از انتخاب ایشان نخواهید برد. این استفاده
مفید تری از زمانتان است تا آنکه با مخاطبانی صحبت کنید که قصد پذیرش
حرف شما را ندارند. بعد از تمام اینها اگر هنوز به این بحث علاقمند بودند،
حداقل با مانعی مصنوعی روبرو نخواهید شد. این موانع مواردی مانند پیش‌داوری
و تفکر غلط مخاطبان راجع به هدف شما از ترویج گنو/لینوکس می‌باشد.
%I recommend bringing such discussions back to Earth
%immediately, by pointing out that software policy should be based in
%one's own long-term self interest, that you have zero personal stake in
%their choices, and that you have better uses for your time than speaking
%to an unreceptive audience. After that, if
%they're still interested, at least you won't face the same artificial
%obstacle.

در همین‌حال مطمعن باشید که درگیر کلیشه‌های طرفداری بی‌مورد از یک سیستم عامل نشده اید.
صرف اعلام نقطه‌نظر خود به یک شخص، بدون دعوت و اینکه او بخواهد صرفا بی‌ادبانه و توهین‌آمیز
تلقی می‌شود. بعلاوه زمانی‌ که این موضوع به گنو/لینوکس مرتبط است هم این تلاش بی‌فایده خواهد بود.
برخلاف سیستم‌عامل های انحصاری، سیستم‌عامل ما قرار نیست بر اساس مقبولیت و انتشار/نگهداری
برنامه های سازگارشده خود زنده شود یا بمیرد.
گنو/لینوکس و البته تمام اپلیکیشن‌های مهم آن متن‌باز هستن. جامعه برنامه‌نویسانی که آنها
را نگهداری می‌کنند خودکفا بوده و آن را سالم نگه‌داشته و رشد می‌دهدن.
صرف‌نظر از اینکه دنیای بیزینس و عموم مردم عاشق آن هستند و بی‌پروایانه از آن استفاده می‌کنند،
استفاده زیادی دارند و یا حتی اصلا مورد استفاده قرار نخواهند داد.
به‌دلیل قوانین پروانه متن‌باز آنها، کدمنبع به‌طور دائمی در دسترس همگان خواهد بود.
امکان ندارد که گنو/لینوکس بخاطر عدم محبوبیت از سمت یک سری شرکت و کمپانی
"از رده خارج شود". بر این اساس هیچ دلیلی برای پافشاری بر طرفداری از این سیستم‌عامل
وجود ندارد. البته که بعضی از کاربران در انجمن‌هایی این پافشاری و طرفداری بی‌مورد را
انجام می‌دهد که ذکر کردیم. (چطور است اطلاعات را فقط به فردی بدهیم که پذیرای آن است؟
این امر نیاز هرکسی را که شایسته این اطلاعات است برطرف می‌کند.)
%At the same time, make sure you don't live up to the stereotype of the
%OS advocate, either.  Just proclaiming your views at someone without
%invitation is downright rude and offensive. Moreover, when done
%concerning GNU/Linux, it's also pointless:  Unlike the case with proprietary
%OSes, our OS will not live or die by the level of its acceptance and
%release/maintenance of ported applications.  It and all key applications
%are open source: the programmer community that maintains it is
%self-supporting, and would keep it advancing and and healthy regardless
%of whether the business world and general public uses it with wild
%abandon, only a little, or not at all. Because of its open-source
%licence terms, source code is permanently available. GNU/Linux cannot be
%"withdrawn from the market" on account of insufficient popularity, or at
%the whim of some company.  Accordingly, there is simply no point in
%arm-twisting OS advocacy -- unlike that of some OS-user communities we
%could mention.   (Why not just make information available for those
%receptive to it, and stop there?  That meets any reasonable person's
%needs.)

در نهایت درک کنید که
\fftnt{"ارزش استفاده"}{https://en.wikipedia.org/wiki/Use_value}
برای نرم‌افزار برای اکثر مردم غریبه است (مفهوم ارزش‌گذاری نرم‌افزار بر اساس کاربرد آن).
عادت ارزش‌گذاری هرچیز بر اساس
{\itshape هزینه خرید}
قدمت دیرینه دارد.
من در سال ۱۹۹۶ از جوانی در یک لاگ واقع در برکلی، کالیفرنیا راجع به ریشه‌های
\lr{Caldera Network Desktop}
(نام اولیه توزیع گنو/لینوکسی که توسعه داده بود) در پروژه
سیستم‌عامل دسکتاپ "\lr{Corsair}" توسط شرکت \lr{Novell} شنیدم.
پس از بررسی مدیران اجرایی و فنی آنها متوجه شدند که افسران شرکت ها
از هرچه می‌توانند به رایگان دریافت کنند ناراضی هستند. سپس \lr{Caldera}
به آنها راهکاری را با دریافت هزینه ارائه داد.

%Last, understand that the notion of "use value" for software is quite
%foreign to most people -- the notion of measuring software's value by
%what you can do with it.  The habit of valuing everything at
%{\itshape acquisition cost\/} is deeply ingrained.  In 1996, I heard a young
%fellow from Caldera Systems speak at a Berkeley, California LUG about
%the origins of Caldera Network Desktop (the initial name of their GNU/Linux
%distribution) in Novell, Inc.'s "Corsair" desktop-OS project:  In
%surveying corporate CEOs and CTOs, they found corporate officers to be
%inherently unhappy with anything they could get for free.  So, Caldera
%offered them a solution -- by charging money.

از این دیدگاه محتاط بودن درمورد گوشزد کردن هزینه و دشواری های توسعه گنو/لینوکس
به مردم به جذب شدن آنها کمک کرده و اعتبار شما را بعنوان یک سخنران حفظ می‌کند.
حتی بردن بحث به سمت قیمت به معنی قیمت عملکردی می‌تواند بهتر باشد
(مانند ظرفیت ۱۰۰ نفره ایمیل های شرکتی با کاربر های آفلاین و آنلاین).
در مقابل فهرست کردن نرم‌افزار به سبک خرده‌فروشی با قیمت گذاری به اندازه کافی مفید نمی‌باشد.
در نهایت تمام پروژه های نرم‌افزاریهزینه ای دارند حتی اگر برچسب قیمت آنها صفر باشد.
هدف اصلی متن‌باز قیمت اولیه نیست بلکه کنترل بلند مدت بر روی فناوری اطلاعات و تکنولوژی می‌باشد
(که بخش کلیدی برای یک سازمان است). در یک سیستم انحصاری کاربر (یا کسب‌و‌کار) کنترل فناوری اطلاعات
را از دست داده و در رابطه انحصاری نادرست با فروشنده خود قرار می‌گیرد.
کاربر با اوپن سورس می‌تواند کنترل داشته باشد و کسی نمی‌تواند آزادی را از او بگیرد.
همین موضوع توضیح می‌دهد که چرا مردم (به ویژه مدیران جرایی) درحال حاضر تفاوت بین 
اوپن سورس و کالای انحصاری را (بعنوان فرصتی برای کاهش و کنترل ریسک های فناوری اطلاعات)
درک می‌کنند. و این امر در بلندمدت بسیار مهم تر از قیمت خواهد بود.
%Seen from this perspective, being conservative about the costs and
%difficulties of GNU/Linux deployments helps make them positively attractive
%-- and protects your credibility as a speaker.  Even better would be
%to frame the discussion of costs in terms of the cost of functionality
%(e.g., 1000-seat Internet-capable company e-mail with offline-user
%capability and webmail) as opposed to listing software as a retail-style
%line-item with pricing:  After all, any software project has costs,
%even if the acquisition price tag is zero, and the real point of open
%source isn't initial cost but rather long-term control over IT -- a key
%part of one's operations:  With proprietary systems, the user (or
%business) has lost control of IT, and is on the wrong side of a monopoly
%relationship with one's vendor.  With open source, the user is in
%control, and nobody can take that away.  Explained that way (as
%opportunity to reduce and control IT risk), people readily understand
%the difference -- especially CEOs -- and it's much more significant over
%the long term than acquisition cost.

\subsection{آموزش گنو/لینوکس}
%\subsection{GNU/Linux education}

نه تنها لاگ وظیفه ترویج گنو/لینوکس را دارد، بلکه باید اعضا را نیز آموزش دهد
تا از سیستم‌عامل و دیگر بخش‌های مرتبط مرتبط به ما استفاده کنند (هدفی که می‌تواند
تغییر زیادی در منطقه محلی فرد ایجاد کند). 
درحالی که دانشگاه‌ها و کالج ها به‌طور فزاینده ای گنو/لینوکس را
وارد برنامه تحصیلی خود کرده‌اند، با اینحال باز هم اطلاعات لازم به برخی کاربران نخواهد رسید.
یک لاگ می‌تواند کمک‌های ابتدایی و پیشرفته ای در تکنولوژی‌های مدیریت سیستم، برنامه‌نویسی
اینترنت و اینترانت و غیره را انجام بدهد.
%Not only is it the business of a LUG to advocate GNU/Linux usage, but
%also to train members, as well as the nearby computing public,
%to use our OS and associated components -- a goal that can make a huge
%real-world difference in one's local area.  While universities and
%colleges are increasingly including GNU/Linux in their curricula, for
%sundry reasons, this won't reach some users.  For those, a LUG can
%give basic or advanced help in system administration, programming,
%Internet and intranet technologies, etc.

برخلاف انتظار بیشتر لاگ‌ها ستون پشتیبانی از شرکت ها هستند.
به ازای هر نیرویی که با مشارکت در لاگ مهارت های کامپیوتری خود را
افزایش می‌دهد، این مسئولیت از روی دوش شرکت و کمپانی ها برداشته خواهد شد.
اگرچه مدیریت یک سیستم گنو/لینوکس خانگی به دقیقا اندازه مدیریت سیستم های
بزرگی در مقیاس راه‌اندازی یک
\fftnt{پایگاه داده تحلیلی}{https://en.wikipedia.org/wiki/Data_warehouse}،
\fftnt{مراکز تماس}{https://en.wikipedia.org/wiki/Call_centre}،
یا دیگر تاسیسات دسترسی‌بالا نمی‌رسد. اما کاملا بهتر از تجربه کار با
مایکروسافت ویندوز خواهد بود.
با پیشرفت لینوکس به سمت
\fftnt{فایل سیستم های ژورنالی}{https://en.wikipedia.org/wiki/Journaling_file_system}،
\fftnt{دسترسی سطح بالا}{High availability}،
افزونه های بلادرنگ، و دیگر قابلیت های سطح‌بالا لینوکس،
مرز مبهم بین گنو/لینوکس و یونیکس های واقعی کاملا ناپدید شده اند.
%In an ironic twist, many LUGs have turned out to be a backbone of
%corporate support: Every worker expanding her computer skills through
%LUG participation is one fewer the company must train.  Though home
%GNU/Linux administration doesn't exactly scale to running corporate data
%warehouses, call centres, or similar high-availability facilities, it's
%light years better preparation than MS-Windows experience.  As Linux has
%advanced into journaling filesystems, high availability, real-time
%extensions, and other high-end Unix features, the already blurry line
%between GNU/Linux and "real" Unixes vanished entirely.

نه‌تنها چنین آموزش هایی نوعی آموزش به کارمندان است، بلکه با افزایش
اهمیت فناوری تکنولوژی برای اقتصاد جهانی، خدماتی به جامعه نیز ارائه می‌کند.
برای مثال در کلان‌شهر های آمریکا لاگ‌ها گنو/لینوکس را به مدارس، کسب‌و‌کار های کوچک،
جامعه،  ارگان های عمومی و سایر محیط‌های غیر شرکتی نیز آورده اند.
این امر باعث می‌شود اهداف مربوط به ترویج و همچنان آموزش همگانی برای مردم براورده شود.
با افزایش بیشتر حضور شرکت ها در اینترنت و نیاز به فراهم کردن دسترسی برای پرسنل خود،
یا دیگر عملکرد های مربوط به گنو/لینوکس، لاگ‌ها فرصت هایی برای مشارکت جمعی خود را کسب می‌کنند.
این مشارکت از طریق تلاش‌هایی برای آگاه سازی و آموزش انجام می‌گیرد.
این اتفاق روح سخاوتمند ذات جامعه گنو/لینوکس و نرم‌افزار‌ آزاد/متن‌باز را به انجمن منتقل می‌کند.
اکثر کاربرها نمی‌توانند مانند لینوس توروالدز برنامه‌نویسی کنند اما همه ما می‌توانیم تلاش و زمان خود را
برای کاربران دیگر، جامعه گنو/لینوکس و البته محیط گسترده‌تر اطراف خود صرف کنیم.
%Not only is such education a form of worker training, but it will also
%serve, as information technology becomes increasingly vital to the
%global economy, as community service: In the USA's metropolitan areas,
%for example, LUGs have taken GNU/Linux into local schools, small businesses,
%community and social organisations, and other non-corporate
%environments. This accomplishes the goal of advocacy and also
%educates the general public.  As more such organisations seek Internet
%presence, provide their personnel dial-in access, or other
%GNU/Linux-relevant functions, LUGs gain opportunities for community
%participation, through awareness and education efforts -- extending to
%the community the same generous spirit characteristic of GNU/Linux and the
%free software / open source community from its very beginning. Most
%users can't program like Torvalds, but we can all give time and
%effort to other users, the GNU/Linux community, and the broader
%surrounding community.

گنو/لینوکس برای این سازمان‌ها انتخاب مناسبی است. زیرا استقرارها
آنها را متعهد به پروانه‌ها، بروزرسانی‌ها و یا نگهداری های گران قیمت
نخواهد کرد. اگر فنی، خوش سلیقه و اقتصادی باشید، گنو/لینوکس همچنان بر روی
سخت افزار هایی که از طرف شرکت‌های تولید سخت افزار رها شده اند نیز به خوبی
کار می‌کند. معمولا سازمان های غیرانتفایی/ناسودبر مایل به استفاده از این
سخت‌افزار های ترد شده هستند. درواقع یک سیستم با مشخصات پنتیوم ۲ هسته ای
که در کمد اتاقتان وجود دارد، می‌تواند "واقعا کار کند" اگر کسی روی آن
گنو/لینوکس نصب کند.

%GNU/Linux is a natural fit for these organisations, because deployments
%don't commit them to expensive licence, upgrade, or maintenance fees.
%Being technically elegant and economical, it also runs very well on
%cast-off corporate hardware that non-profit organisations are only too
%happy to use: The unused Pentium Core 2 in the closet can do {\bfseries real
%work}, if someone installs GNU/Linux on it.

علاوه بر این آموزش در طول زمان به تحقق اهداف دیگر لاگ نیز کمک خواهد کرد.
مخصوصا حمایت و پشتیبانی. آموزش بهتر به معنای حمایت بهتر است.
این امر باعث سهولت در آموزش و رشد آسان تر انجمن خواهد شد.
بنابراین آموزش سنگ‌بنای تمام تلاش‌ها را شکل می‌دهد.
اگر فقط ۲ یا ۳ درصد جمعیت لاگ، بار دیگر اعضای لاگ را به‌دوش بکشد،
رشد آن لاگ متوقف خواهد شد. مطمعن باشید
{\bfseries {\itshape
اگر اعضای جدید و بی‌تجربه کمک‌های لازم را از لاگ دریافت نکنند، مدت زیادی مشارکت نخواهد کرد.
}}
اما اگر درصد بیشتر اعضای لاگ دیگر اعضا را حمایت کنند، لاگ با چنین محدودیت هایی روبرو نخواهد شد.
آموزش (و البته به همان اندازه حمایت برای پروژه های متحدی مانند وب‌سرور آپاچی،
 \lr{X.org, Freedesktop.org, Tex, LaTeX}
 غیره) کلید این تکاپو خواهد بود. آموزش کاربران جدید را به متخصصان آینده تبدیل می‌کند.
%In addition, education assists other LUG goals over time, in
%particular that of support: Better education means better
%support, which in turn facilitates education, and eases the 
%community's growth.  Thus, education forms the entire effort's keystone:
%If only two or three percent of a LUG assume the remainder's support
%burden, that LUG's growth will be stifled. One thing you can count on:
%{\bfseries {\itshape If new and inexperienced users don't get needed help
%from their LUG, they won't participate there for long\/}}.
%If a larger percentage of members support the rest, the LUG will not
%face that limitation. education -- and, equally, support for
%allied projects such as the Apache Web server, X.org, Freedesktop.org, 
%TeX, LaTeX, etc.  -- is key to this dynamic: Education turns new users into
%experienced ones.

نهایتا گنو/لینوکس یک محیط عملیاتی خود-مستندساز است.
به معنای دیگر، نوشتن و انتشار مستندات انجمن ما به خودمان مربوط است.
از همین رو اطمینان حاصل کنید که ممبر های لاگ
\tfftnt{پروژه مستند سازی لینوکس}{Linux Documentation Project}
و میرورهای آن را می‌شناسند.
به راه‌اندازی یک سایت میرور از \lr{LDP} نیز فکر کنید.
مطمعن شوید که هر مستندی که می‌تواند به رشد لاگ کمک کند را
بصورت عمومی منتشر می‌کنید. این مستندات شامل ارائه های فنی، آموزش ها،
پرسش و پاسخ های محلی و غیره باشند. برای عمومی کردن این مستندها می‌توانید
از طریق
{\ttfamily comp.os.linux.announce}, \lr{LDP}
و دیگر منابع مربوطه اقدام کنید.
معمولا مستندات لاگ نمی‌توانند در سطح جهانی سود برسانند. دلیل این اتفاق
تنها عدم اطلاع رسانی صحیح می‌باشد. اجازه ندهید این اتفاق بیوفتد.
بسیار محتمل است که فردی در یک لاگ سوال یا مشکلی با چیزی داشته باشد
و شخصی در جای دیگری همان سوال یا مشکل را در سر داشته باشد.
 
%Finally, GNU/Linux is a self-documenting operating environment: In other
%words, writing and publicising our community's documentation is up to
%us.  Therefore, make sure LUG members know of the 
%\emph{Linux Documentation Project} \texttt{\acourl}
% and its worldwide mirrors.
%Consider operating an LDP mirror site.  Also, make sure to
%publicise -- through {\ttfamily comp.os.linux.announce}, the LDP, and other
%pertinent sources of information -- any relevant documentation the LUG
%develops: technical presentations, tutorials, local FAQs, etc.  LUGs'
%documentation often fails to benefit the worldwide community for no
%better reason than not notifying the outside world.  Don't let that
%happen:  It is highly probable that if someone at one LUG had a question
%or problem with something, then others elsewhere will have it, too.

\subsection{پشتیبانی از گنو/لینوکس \lr{(support)}}
%\subsection{GNU/Linux support}

Of course, for the {\bfseries newcomer}, the primary role of a
LUG is GNU/Linux support -- but it is a mistake to suppose that 
support means only {\itshape technical\/} support for new users. It
should mean much more.

LUGs have the opportunity to support:

\begin{itemize}
\item users
\item consultants
\item businesses, non-profit organisations, and schools
\item the GNU/Linux movement
\end{itemize}

\subsubsection{Users}

New users' most frequent complaint, once they have GNU/Linux
installed, is the steep learning curve characteristic of all modern 
Unixes. (That sentence was true in 1997 when this HOWTO's first
maintainer wrote it, but happily very little, any more.)  With that learning
curve, however, comes the power and flexibility of a real operating
system. A LUG is often the new user's main resource, to flatten the
learning curve.

During GNU/Linux's first decade, it gained some first-class journalistic 
resources, which should not be neglected:  The main (surviving) monthly 
magazine of longest standing is {\itshape 
\emph{Linux Journal} \texttt{\acpurl}
\/} (USA).
More recently, 
they've been joined by 
{\itshape 
\emph{Linux Format} \texttt{\acqurl}
\/} (UK),
{\itshape 
\emph{Linux Magazine} \texttt{\acrurl}
\/} (German publishing firm; publishes in English, German, Polish, Brazilian Portuguese, and Spanish; North American edition is named {\itshape Linux Pro Magazine\/}),
{\itshape 
\emph{Open Source For You} \texttt{\acsurl}
\/} (India; formerly {\itshape LINUX For You)\/},
{\itshape 
\emph{Full Circle} \texttt{\acturl}
\/} (international; covers Ubuntu family distributions), 
{\itshape 
\emph{Linux Voice} \texttt{\acuurl}
\/} (UK), 
{\itshape 
\emph{easyLinux} \texttt{\acvurl}
\/} (German),
{\itshape 
\emph{LinuxUser} \texttt{\acwurl}
\/} (German), 
{\itshape 
\emph{Ubuntu User} \texttt{\acxurl}
\/} (German
publishing firm; in English), 
{\itshape 
\emph{Free Software Magazine} \texttt{\acyurl}
\/} (formerly {\itshape The Open Voice\/}), and
{\itshape 
\emph{Ubuntu User} \texttt{\aczurl}
\/} (German publishing firm; in English).

Standout on-line magazines and news sites with weekly or better publication 
cycles include {\itshape 
\emph{Linux Weekly News} \texttt{\adaurl}
\/}, 
{\itshape 
\emph{DistroWatch Weekly} \texttt{\adburl}
\/},
{\itshape 
\emph{Linux Today} \texttt{\adcurl}
\/}, and

\emph{FreshNews} \texttt{\addurl}
.

All of these resources have eased LUGs' job of spreading essential
news and information --  about bug fixes, security problems, patches, 
new kernels, etc., but new users must still be made aware of
them, and taught that the newest kernels are always
available from 
\emph{kernel.org} \texttt{\adeurl}
,
that the 
\emph{Linux Documentation Project} \texttt{\adfurl}
 has newer versions of Linux HOWTOs than do DVD/CD-based GNU/Linux
distributions, and so on.

Intermediate and advanced users also benefit from proliferation of
timely and useful tips, facts, and secrets. Because of the GNU/Linux
world's manifold aspects, even advanced users often learn new tricks or
techniques simply by participating in a LUG. Sometimes, they learn of
software packages they didn't know existed; sometimes, they just
remember arcane {\ttfamily vi} command sequences they've not used since
college.




\subsubsection{Consultants}



LUGs can help consultants find their customers and vice-versa,
by providing a forum where they can come together.
Consultants also aid LUGs by providing experienced leadership.
New and inexperienced users gain benefit from both LUGs and
consultants, since their routine or simple requests for support are
handled by LUGs {\itshape gratis\/}, while their complex needs and problems --
the kind requiring paid services -- can be fielded by consultants found 
through the LUG.

The line between support requests needing a consultant and those
that don't is sometimes indistinct; but, in most cases, the difference
is clear. While a LUG doesn't want to gain the reputation for
pawning new users off unnecessarily on consultants -- as this is simply
rude and very anti-GNU/Linux behaviour -- there is no reason for LUGs not to
help broker contacts between users needing consulting services and
professionals offering them.

Caveat:  While "the difference is clear" to intelligent people of goodwill,
the Inevitable Ones are {\itshape also\/} always with us, who act willfully
dense about the limits of free support when they have pushed those
limits too far.  Remember, too, my earlier point about the vast majority
of the population valuing everything at acquisition cost (instead of use
value), {\itshape including what they receive for free\/}.  This leads some,
especially some in the corporate world, to use (and abuse) LUG
technical support with wild abandon, while simultaneously complaining
bitterly of its inadequate detail, insufficient promptness, supposedly
unfair expectations that the user learn and not re-ask minor variations on
the same question endlessly, etc.  In other words, they treat relations
with LUG volunteers the way they would a paid support vendor, but one
they treat with {\itshape zero respect\/} because of its zero acquisition
cost.

In the consulting world, there's a saying about applying "invoice therapy" 
to such behaviour:  Because of the value system alluded to above, if
your consulting advice is poorly heeded and poorly used, it just might
be the case that you need to charge more.  By contrast, the technical
community has often been characterised as a "gift culture", with a
radically different value system: Members gain status through enhanced
reputation among peers, which in turn they improve through visible
participation:  code, documentation, technical assistance to the public,
etc.

Clash between the two very different value-based cultures is inevitable
and can become a bit ugly.  LUG activists should be prepared to intercede
before the ingrate newcomer is handed her head on a platter, and
politely suggest that her needs would be better served by paid
(consultant-based) services.  There will always be judgement calls;
the borderline is inherently debatable and a likely source of
controversy.

Telltale signs that a questioner may need to be transitioned to consulting-based assistance include:

\begin{itemize}
\item An insistence on getting solutions in "recipe" (rote) form, 
with the apparent aim of not needing to learn technological 
fundamentals.
\item Asking the same questions (or ones closely related) repeatedly.
\item Insisting on {\itshape private\/} assistance from helpers active in
{\itshape public\/} (GNU/Linux community) forums.
\item Providing only vague problem descriptions, or ones that change with time.
\item Interrupting answers in order to ask additional questions 
(suggesting lack of attention to the answers).
\item Demanding that answers be recast or delivered more quickly 
(suggesting that the questioner's time and trouble are 
valuable, but that helpers' are not).
\item Asking unusually complex, time-consuming, and/or multipart 
questions.
\end{itemize}


In general, LUG members are especially delighted to help, on a volunteer
basis, members who seem likely to participate in the "gift
culture" by picking up its body of lore and, in turn, perpetuating it
by teaching others in their turn.  Certainly, there's nothing wrong with
having other priorities and values, but such folk may in some cases be
best referred to paid assistance, as a better fit for their needs.

An additional observation that may or may not be useful, at this point:
There are things one may be willing to do for free, to assist others in the
community, that one will refuse to do for money:  Shifting from
assisting someone as a volunteer fundamentally changes the relationship.
A fellow computerist who suddenly becomes a customer is a very different
person; one's responsibilities are quite different, and greater.  You're
advised to be aware, if not wary, of this distinction.






\subsubsection{Businesses, non-profit organisations, and schools}

LUGs also have the opportunity to support local businesses and
organisations. This support has two aspects: First, LUGs can support
businesses and organisations wanting to use our OS (and its 
applications) as a part of their
computing and IT efforts. Second, LUGs can support local businesses
and organisations developing software for GNU/Linux, cater to users,
support or install distributions, etc.

The support LUGs can provide to local businesses wanting to use GNU/Linux as
a part of their computing operations differs little from the help LUGs
give individuals trying GNU/Linux at home. For example, compiling the Linux
kernel doesn't really differ. Supporting businesses, however, may
require supporting proprietary software -- e.g., the Oracle, Sybase,
and DB2 databases (or VMware, CrossOver Linux, and such things).   
Some LUG expertise in these areas may help businesses make the leap
into GNU/Linux deployments.

This leads us directly to the second kind of support a LUG can give to
local businesses: LUGs can serve as a clearinghouse for information
available in few other places. For example:

\begin{itemize}
\item Which local ISP is Linux-friendly?
\item Are there any local hardware vendors building Linux PCs?
\end{itemize}


Maintaining and making this kind of information public not only helps
the LUG members, but also helps friendly businesses and encourages
them to continue to be GNU/Linux-friendly. It may even, in some cases, help
further a competitive environment in which other businesses are
encouraged to follow suit.




\subsubsection{Free / open-source software development}

Finally, LUGs may also support the movement by soliciting and
organising charitable giving. 
\emph{Chris Browne} \texttt{\adgurl}
 has thought about this issue as much as
anyone I know, and he contributes the following:




\paragraph{Chris Browne on free software / open source philanthropy}

 
A further involvement can be to encourage sponsorship of various
GNU/Linux-related organisations in a financial way.  With the 
multiple millions of users, it would be entirely plausible for grateful 
users to individually contribute a little. Given millions of users, and 
the not-unreasonable sum of a hundred dollars of "gratitude" per user (\$100 being
roughly the sum {\itshape not\/} spent this year upgrading a Microsoft OS),
that could add up to {\itshape hundreds of millions\/} of dollars towards
development of improved GNU/Linux tools and applications.



 
A user group can encourage members to contribute to various
"development projects". Having some form of "charitable tax exemption"
status can encourage members to contribute directly to the group,
getting tax deductions as appropriate, with contributions flowing on to
other organisations.



 
It is appropriate, in any case, to encourage LUG members to direct
contributions to organisations with projects and goals they
individually wish to support.



 
This section lists possible candidates. None is being explicitly 
recommended here, but the list represents useful food for
thought.  Many are registered as charities in the USA, thus
making US contributions tax-deductible.



Here are organisations with activities particularly directed towards
development of software working with GNU/Linux:

\begin{itemize}
\item 
\emph{The Linux Foundation} \texttt{\adhurl}
\item 
\emph{Debian / Software In the Public Interest} \texttt{\adiurl}
\item 
\emph{Free Software Foundation} \texttt{\adjurl}
 
\item 
\emph{KDE Project (KDE e.V.)} \texttt{\adkurl}
\item 
\emph{GNOME Foundation} \texttt{\adlurl}
\item 
\emph{Software Freedom Conservancy} \texttt{\admurl}
\item 
\emph{The Mozilla Foundation} \texttt{\adnurl}
\end{itemize}




Contributions to these organisations have the direct effect of
supporting creation of freely redistributable software usable with
GNU/Linux.  Dollar for dollar, such contributions almost certainly yield
greater benefit to the community than any other kind of spending.



 
There are also organisations less directly associated with GNU/Linux, that
may nonetheless be worthy of assistance, such as:

\begin{itemize}
\item The 
\emph{Electronic Frontier Foundation} \texttt{\adourl}
 

Based in San Francisco, EFF is a donor-supported membership organization
working to protect our fundamental rights regardless of technology; to
educate the press, policy-makers, and the general public about civil
liberties issues related to technology; and to act as a defender of 
those liberties. Among our various activities, EFF opposes misguided
legislation, initiates and defends court cases preserving individuals'
rights, launches global public campaigns, introduces leading edge
proposals and papers, hosts frequent educational events, engages the
press regularly, and publishes a comprehensive archive of digital civil
liberties information at one of the most linked-to Web sites in the
world.



\item The LaTeX3 Project Fund 

 
The 
\emph{TeX Users Group (TUG)} \texttt{\adpurl}
 is
working on the "next generation" version of the LaTeX publishing
system, known as LaTeX3.  GNU/Linux is one of the platforms on which TeX
and LaTeX are best supported.

 Donations for the project can be sent to:
\begin{tscreen}
\begin{verbatim}
TeX Users Group
c/o Robin Laakso, executive director
TeX Users Group
PO Box 2311
Portland, OR 97208-2311
\end{verbatim}
\end{tscreen}


Alternatively, donations can be made 

\emph{online} \texttt{\adqurl}
.



\item  
\emph{Project Gutenberg} \texttt{\adrurl}
 

Project Gutenberg's purpose is to make freely available in electronic
form the texts of public-domain books.  This isn't directly a "Linux
thing", but seems fairly worthy, and they actively encourage platform
independence, which means their "products" are quite usable with GNU/Linux.



\item  
\emph{Project Runeberg} \texttt{\adsurl}
 
Project Runeberg is similar to Project Gutenberg, except concentrating
on making editions of classic Nordic (Scandinavian) literature openly
available over the Internet.





\item  
\emph{Open Source Education Foundation} \texttt{\adturl}
 
The Open Source Education Foundation's purpose to enhance K-12 education
through the use of technologies and concepts derived from The Open
Source and Free Software movement.  In conjunction with Tux4Kids, OSEF 
created a bootable distribution of GNU/Linux (Knoppix for Kids) based 
on Klaus Knopper's Knoppix, aimed at kids, parents, teachers, and 
other school officials. OSEF installs and supports school computer labs, 
and has developed a "K12 Box" as a compact Plug and Play workstation 
computer for student computer labs.



\end{itemize}


(Please note that suggested additions to the above list of GNU/Linux-relevant 
charities are most welcome.)






\subsubsection{Linux movement}

I have referred throughout this HOWTO to what I call the {\bfseries GNU/Linux
movement}. There really is no better way to describe the
international GNU/Linux phenomenon: It isn't a bureaucracy, but is
organised. It isn't a corporation, but is important to businesses
everywhere. The best way for a LUG to support the international GNU/Linux
movement is to keep the local community robust, vibrant, and
growing. GNU/Linux is {\itshape developed\/} internationally, which is easy
enough to see by reading the kernel source code's 
\url{MAINTAINERS} file -- but
GNU/Linux is also {\itshape used\/} internationally.  This ever-expanding
user base is key to GNU/Linux's continued success, and is where the LUGs
are vital.

The movement's strength internationally lies in offering
unprecedented computing power and sophistication for its cost and
freedom. The keys are value and independence from proprietary control.
Every time a new person, group, business, or organisation experiences
GNU/Linux's inherent value, the movement grows.  LUGs help that
happen.




\subsection{Linux socialising}

The last goal of a LUG we'll cover is socialising -- in some ways,
the most difficult goal to discuss, because it isn't clear how
many or to what degree LUGs do it. While it would be strange to
have a LUG that didn't engage in the other goals, there may be
LUGs for which socialising isn't a factor.

It seems, however, that whenever two or three GNU/Linux users get together,
fun, hijinks, and, often, beer follow. Linus Torvalds has
always had one enduring goal for Linux: to have more fun. For hackers,
kernel developers, and GNU/Linux users, there's nothing quite like
downloading a new kernel, recompiling an old one, fooling with a
window manager or desktop environment, hacking some code, or experimenting
with an innovative embedded Linux computer. GNU/Linux's sheer fun keeps many
LUGs together, and leads LUGs naturally to socialising.

By "socialising", here I mean primarily sharing experiences, forming
friendships, and mutually-shared admiration and respect. There is
another meaning, however -- one social scientists call
{\itshape acculturation\/}. In any movement, institution, or human
community, there is the need for some process or pattern of events in
and by which, to put it in GNU/Linux terms, newcomers are turned into
hackers. In other words, acculturation turns you from "one of them" to
"one of us".

It is important that new users come to learn GNU/Linux culture,
concepts, traditions, and vocabulary.  GNU/Linux acculturation, unlike "real
world" acculturation, can occur on mailing lists, Web forums, and
Usenet, although the latter's efficacy is challenged by poorly
acculturated users and by spam. LUGs are often much more efficient at
this task than are mailing lists, Web forums, or newsgroups, precisely
because of LUGs' greater interactivity and personal focus.



