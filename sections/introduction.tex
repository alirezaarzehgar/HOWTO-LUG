\section{مقدمه}
%\section{Introduction}

\subsection{این راهنما چه هدفی دارد؟}
%\subsection{Purpose}

ای راهنما قصد دارد بعنوان مرجعی برای تاسیس، نگهداری و توسعه گروه های گنو/لینوکس استفاده شود.
تلاش بر این است که علاقمندان با خواندن این راهنما لاگ را به خوبی شناخته و بتوانند به‌درستی در آن
مشارکت کنند. از طرفی این مستند به مدیران و داوطلبان فعال‌تر لاگ یاد می‌دهد که چطور لاگ را به درستی
مدیریت کنند و چه قوانینی را بهتر است در لاگ اعمال کنند.
%The Linux User Group HOWTO is intended to serve as a guide to founding,
%maintaining, and growing a GNU/Linux user group.

گنو/لینوکس یک پیاده سازی بطور آزاد منتشر شده از یونیکس است که برای کامپیوترهای شخصی،
سرورها، کارگاه ها، \lr{PDA} ها و سیستم های نهفته طراحی شده است.
گنو/لینوکس در ابتدا بر روی معماری \lr{i386} توسعه یافت و اکنون طیف گسترده ای از پردازنده‌ها
را از ریز تا درشت پشتیبانی می‌کند. در این راهنما قصد داریم با معرفی لاگ به گنو/لینوکس و جنبش
آزادی نرم‌افزار کمک کنیم.
%GNU/Linux is a freely-distributable implementation of Unix for personal
%computers, servers, workstations, PDAs, and embedded systems.  It was
%developed on the i386 and now supports a huge range of processors from 
%tiny to colossal:

%%%%%%%%%%%%%%%%%%%%%%%%%%%%%%%
% This information does not help readers
%
\begin{note}
لیست پلتفرم های پشتیبانی شده زیر قابل استنداد نیست و صرفا قصد دارد وسعت لینوکس را نمایش بدهد.
\end{note}

%Note:  The following supported-platforms list is not serious documentation.  
%The point is merely to illustrate the breadth of Linux's reach.  

اگر بطور جدی به پورت های لینوکس علاقه مندید میتوانید دو صفحه
\ftnt{Xose Vazquez Perez's Linux ports page}{http://web.archive.org/web/20070813000855/http://www.itp.uni-hannover.de/ports/linux_ports.html}
و
\ftnt{Jerome Pinot's Linux architectures list}{http://web.archive.org/web/20050308130348/http://ngc891.blogdns.net/kernel/docs/arch.txt}
را نیز برسی کنید.  (هردو صفحه مذکور در سال ۲۰۰۵ ناپدید شدند)
چرا که پشتیبانی سخت‌افزاری پیچیده‌تر از عملکرد عمومی \lr{CPU} است و شامل پشتیبانی از
انواع مختلف \lr{bus} و مسائل ظریف سخت‌افزاری دیگر می‌شود (به‌ویژه برای پورت‌های لینوکس
\ftnt{PDA / embedded / microcontroller / router}{http://www.linuxfordevices.com/}
).

%If seriously interested in the subject of Linux ports, please see also 

%\emph{Xose Vazquez Perez's Linux ports page} \tturl{http://web.archive.org/web/20070813000855/http://www.itp.uni-hannover.de/ports/linux_ports.html}
%and
%\emph{Jerome Pinot's Linux architectures list} \tturl{http://web.archive.org/web/20050308130348/http://ngc891.blogdns.net/kernel/docs/arch.txt}

% (static mirrors, as both pages vanished in 2005), if only because 
%hardware support is more complex than just generic CPU functionality, 
%encompassing support for myriad bus variations and other subtle hardware
%issues (especially for 

%\href{http://www.linuxfordevices.com/}{Linux PDA / embedded / microcontroller / router ports}
%).  

\begin{latin}
\begin{itemize}
	\item {\bfseries Diverse \ftnt{PDA / embedded / microcontroller / router}{http://www.uclinux.org/ports/} devices:}
	\begin{itemize}
		\item Advanced RISC Machines, Ltd. \ftnt{ARM}{http://www.arm.linux.org.uk/} family (StrongARM SA-1110, XScale, ARM6, ARM7, ARM2, ARM250, ARM3i, ARM610, ARM710,ARM7TDMI, ARM720T, and ARM920T, including Sigma Designs DVD systems using ARM cores)
		\item Analog Devices, Inc.'s \ftnt{Blackfin DSP}{http://www.linuxfordevices.com/c/a/News/Another-nativeDSP-Linux-port-this-one-to-ADIs-Blackfin/}
		\item Axis Communications \ftnt{ETRAX series}{https://en.wikipedia.org/wiki/ETRAX_CRIS}
			("CRIS" = Code Reduced Instruction Set RISC architecture)
		\item Elan SC520 and SC300
		\item FreeScale \ftnt{MC68EN302}{http://www.freescale.com/files/netcomm/doc/data_sheet/MC68EN302.pdf}
		\item Fujitsu \ftnt{FR-V}{http://ecos.sourceware.org/hardware.html#FR-V}
		\item Hitachi \ftnt{H8}{http://www.uclinux.org/pub/uClinux/ports/h8/} series
		\item Intel i960
		\item Intel IA32-compatibles (Cyrix MediaGX, STMicroelectronics \ftnt{STPC}{http://web.archive.org/web/20070626190436/http://www.stmcu.com/forums-cat-132-6.html}, ZF Micro ZFx86)
		\item Matsushita \ftnt{AM3x}{http://ecos.sourceware.org/hardware.html#MatsushitaAM3x}
		\item MIPS-compatibles (Toshiba TMPRxxxx / TXnnnn, NEC \ftnt{VR}{http://www.linux-mips.org/wiki/NEC_VR4100} series, Realtek 8181")
		\item Motorola 680x0-based machines (Motorola VMEbus boards, 
			\ftnt{ISICAD Prisma}{http://ds.dial.pipex.com/town/way/fr30/} machines, and Motorola Dragonball \& 
			\ftnt{ColdFire}{http://www.uclinux.org/ports/coldfire/} CPUs, and Cisco 2500/3000/4000 series routers)
		\item Motorola embedded \ftnt{PowerPC}{http://penguinppc.org/embedded/}
			(including MPC / PowerQUICC I, II, III families)
		\item NEC \ftnt{V850E}{http://ecos.sourceware.org/tools/linux-v850-elf.html}
		\item Renesas Technology (formerly Hitachi) SH3/SH4 (\ftnt{SuperH}{http://wiki.debian.org/SH4})
		\item Samsung \ftnt{CalmRISC}{http://ecos.sourceware.org/hardware.html#CalmRISC}
		\item Texas Instruments's 
		\ftnt{DM64x}{http://www.linuxfordevices.com/c/a/News/Worlds-first-nativeDSP-Linux-port/}
			and \ftnt{C54x DSP}{http://www.linuxfordevices.com/c/a/News/Embedded-Linux-distro-supports-TI-DSPbased-digital-media-processors/} families
		\item Xilinx \ftnt{PetaLinux}{http://www.xilinx.com/tools/petalinux-sdk.htm}
			(formerly SoftBlaze, formerly Microblaze) soft processor implemented on Xilinx FPGAs
	\end{itemize}

	\item {\bfseries Intel \ftnt{8086 / 80286}{http://elks.sourceforge.net/}}.
	\item {\bfseries Intel IA32 family:} i386, i486, Pentium, Pentium Pro,
		Pentium II, Pentium III, Celeron, Xeon, and Pentium IV processors,
		as well as IA32 clones from AMD (386DX/DXL/SL/SLC/SX,
		486DX/DX2/DX4/SL/SLC/SLC2/SLC3/SX/SX2, Elan, K5,
		K6/K6-II/K6-III), Cyrix (386DX/DXL/SL/SLC/SX,
		486DLC/DLC2/DX/DX2/DX4/SL/SLC/SLC2/SLC3/SX/SX2, Cyrix III),
		IDT (Winchip, Winchip 2, Winchip 2A/3),
		IBM (486DX/DX2/DX4/SL/SLC/SLC2/SLC3/SX/SX2),
		NexGen (Nx586), Transmeta (Crusoe),
		TI (486DLC/DLC2), UMC (486SX-S, U5D/U5S),
		VIA (C3 Ezra "CentaurHauls", C3-2 "Nehemiah"),
		and others.
	\item {\bfseries Intel/HP \ftnt{IA64}{http://www.ia64-linux.org/}:} Trillian, Itanium, Itanium2/McKinley
	\item {\bfseries x86-64 family} including AMD Hammer/Opteron/K8/Athlon64/Turion/Phenom/Phenom II/FX/Fusion and
		Intel Prescott/Nocona/Potomac, Core, Atom, Nehalem, Sandy Bridge and Ivy Bridge
	\item {\bfseries Motorola \ftnt{68020-68040}{http://www.linux-m68k.org/} series (with MMU)}:
		\ftnt{m68k Mac}{http://www.mac.linux-m68k.org/},
		Amiga, Atari ST/TT/Medusa/Falcon, HP/Apollo Domain,
		\ftnt{HP9000/300}{http://www.tazenda.demon.co.uk/phil/linux-hp/}, sun3, and
		\ftnt{Sinclair Q40}{http://ftp4.de.freesbie.org/pub/misc/tsx-11/680x0/q40/install/}.
	\item {\bfseries Motorola/IBM PowerPC family:} Most
		\ftnt{PowerMac}{http://penguinppc.org/mac/}
		 (including G3/G4/G5)  / CHRP / PReP / POP, \ftnt{Amiga PowerUP System}{http://linux-apus.sourceforge.net/}, and IBM PPC64 (AS/400, RS/6000, iSeries,	pSeries, PowerMac G5).
	\item {\bfseries \ftnt{MIPS}{http://www.linux-mips.org/}:} most SGI, Cobalt Qube,
		\ftnt{DECStation}{http://decstation.unix-ag.org/}, Sony PlayStation2, and many others
	\item {\bfseries DEC\ftnt{Alpha}{http://www.alphalinux.org/}}
	\item {\bfseries HP \ftnt{PA-RISC}{http://www.parisc-linux.org/}}
	\item {\bfseries SPARC International SPARC32 / SPARC64}
	\item {\bfseries Digital \ftnt{VAX}{http://vax-linux.org/} minicomputers and MicroVAXen}
	\item {\bfseries Mainframes:} 
	\ftnt{IBM S/390 models G5 and G6 / zSeries models z800, z890, z900, and z990}{https://www.ibm.com/developerworks/linux/linux390/}
		 and Fujitsu AP1000+ (SuperSPARC cluster)
\end{itemize}
\end{latin}

توجه کنید که موارد ذکر شد بعضا تنها یک بار فورک شده اند. در این لیست ریز تا درشت
پورت‌های مختلف لینوکس معرفی شده. در برخی از معماری های نادر،
\ftnt{NetBSD}{http://www.netbsd.org/}
گاها کارآمد تر خواهد بود. همچنان پورت
\ftnt{Debian GNU/kFreeBSD}{http://www.debian.org/ports/kfreebsd-gnu/}
به اندازه کافی سازگار و مناسب میباشد. این پورت مجهز به کد یوزر اسپیس گنو/لینوکس بر روی کرنل فری بی اس دی با پرفرمنس بالا و پایداری بالا،
\ftnt{OpenIndiana}{https://en.wikipedia.org/wiki/OpenIndiana}
و یا دیگر 
\ftnt{Illumos distribution}{https://en.wikipedia.org/wiki/Illumos#Relatives}
میباشد و میتواند چیز دیگری مشابه کرنل اوپن سولاریس فراهم کند.

%Note that some items listed were probably one-time forks, little or not
%at all maintained since creation.  On some of the rarer architectures, \emph{NetBSD} \texttt{http://www.netbsd.org/} may be more practical.(The \emph{Debian GNU/kFreeBSD} \texttt{http://www.debian.org/ports/kfreebsd-gnu/}
% port should also be solid enough to 
%serve as a compromise option, furnishing GNU/Linux userspace code on the
%high performance / high stability FreeBSD kernel, and
%\emph{OpenIndiana} \texttt{https://en.wikipedia.org/wiki/OpenIndiana}
% or another 
%\emph{Illumos distribution} \texttt{https://en.wikipedia.org/wiki/Illumos#Relatives}
% can provide something similar on the OpenSolaris kernel.)


%%%%%%%%%%%%%%%%%%%%%%%%%%%%%%%

\subsection{منابع بیشتر راجع به لاگ}
%\subsection{Other sources of information}

اگر میخواهید مطالب بیشتری مطالعه کنید،
\fftnt{پروژه مستندسازی لینوکس}{http://www.tldp.org/}
منبع بهتری برای شروع می‌باشد. برای اطلاعات عمومی راجع به گروه های کاربری کامپیوتر نیز لطفا
\fftnt{انجمن گروه‌های کاربران کامپیوتر}{http://apcug2.org/}
را بررسی کنید.

%If you want to learn more, the \emph{Linux Documentation Project} \texttt{http://www.tldp.org/} is a good place to start.
%For general information about computer user groups, please see the \emph{Association of PC Users Groups} \texttt{http://apcug2.org/}.

