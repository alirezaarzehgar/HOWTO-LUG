\section{مسائل قانونی و سیاسی}
%\section{Legal and political issues}

\subsection{مسائل قانونی سازمانی}
%\subsection{Organisational legal issues}

می‌توانیم مسائل قانونی سازمانی را برای لاگ‌های رسمی تشریح کنیم و راجع
به آنها بحث و گفت‌وگو کنیم.
%The case for formal LUG organisation can be debated:

{\itshape مزایا:}
ایجاد یک شخصیت حقوقی و معاف مالیاتی شدن مسئولیت‌های لاگ
را به نسبت دیگر شرکت‌ها که مشمول مالیات هستند کاهش می‌دهد و به گروه
کمک می‌کند تا راحت تر بتواند بیمه شود. این امر باعث می‌شود لاگ آسان‌تر
کمک مالی دریافت کند و همچنین مالیاتی در ازای درامدی که گروه دارد
پرداخت نخواهد کرد.
%{\itshape Pro:\/} Incorporation and recognised tax-exemption limits
%liability and helps the group carry insurance. It aids fundraising.
%It avoids claims for tax on group income.

{\itshape معایب:}
برای انسان‌های متواضع و مراقب مسئولیت یک عیب و ایراد نیست که نیاز باشد
با رسمی شدن از آن فرار کند و یا آن‌را کاهش دهد.
(نهایتا شما دوست ندارید با ترس‌های خود روبرو شوید.)
البته حتی با وجود اینکه گروه‌های رسمی گاها از این روش مسئولیت‌های بیمه
را انجام می‌دهند، اما این بیمه‌ها به حدی محدود است که هیچ یک از فعالیت‌های
لاگ را پوشش نخواهند داد.
سپر حفاظتی مسئولیتی سازمان‌ها کار زیادی انجام نمی‌دهند. این سپر صرفا
یک سری خسارات را محدود می‌کنند و به مساوات بین سهامداران پخش می‌کنند.
اما این سپر مسئولیتی نمیتواند اشتباهات افراد را تصحیح کند و اجازه دهد
افراد خطا کنند.
از طرفی جمع‌آوری کمک‌های مالی آنقدر ها هم برای گروهی که فعالیت هایش وابسته به
مخارج سنگین نیست لازم نخواهد بود. دیگر مانند سال ۱۹۸۰ نیازی نیست برای تبلیغات
در روزنامه و مجله آگهی بگذاریم. درواقع هزینه‌ها کاهش یافته است.
وقتی صندوق‌دار ندارید یعنی نیازی نیست که راجع به گزارش‌های مالی کاری کنید یا
از مالیات بترسید.
گاها فضای جلسات بواسطه \lr{ISP} ها، کالج‌ها، رستوران، بار، کافی‌شاپ، موسسات آموزش کامپیوتر،
سازمان‌های مرتبط با گنو/لینوکس، هکراسپیس‌ها و یا دیگر موسسات فراهم می‌شود و به رایگان
فضا مناسب را در اختیار لاگ خواهند داد.
زمانی که نه درامدی وجود دارد و نه خرجی، یعنی نیاز کمتری به سازمان دهی و دست‌وپنجه
نرم کردن با مشکلات اقتصادی است.
از همین رو رسمی شدن و ثبت شدن لاگ به‌عنوان یک سازمان، به‌خودی خود دردسر هایی دارد
و در صورتی که لاگ جایی ثبت نشود این مشکلات هم نخواهد بود.

%{\itshape Con:\/} Liability shouldn't be a problem for modestly careful
%people. (You're not doing skydiving, after all.)  Also, even
%incorporated technical groups seldom carry liability insurance, and that
%insurance is typically so narrow in coverage that almost nothing a LUG
%does would be covered. A corporate liability shield is little use for
%such needs, either (limiting only the group's potential losses to the
%equity stake of the owners, but conferring no immunity to anyone for 
%deeds that person carries out).  Fundraising isn't needed for a group whose
%activities needn't involve significant expenses.  (Dead-tree newsletters
%are so 1980.)  Not needing a treasury, you avoid needing to argue over
%it, file reports about it, or fear it being taxed away. Meeting space
%can sometimes be gotten for free at ISPs, colleges, pizza parlours,
%brewpubs, coffeehouses, computer-training firms, GNU/Linux-oriented
%companies, hackerspaces, or other friendly institutions, and can
%therefore be free of charge to the public.  No revenues and no expenses
%means less need for organisation and concomitant hassles.

اولین نویسنده این راهنما بیشتر به مزایای رسمی ساختن لاگ اشاره کرده است
و مدیر و حافظ دوم این راهنما بیشتر معایب این موضوع و مشکلات آن را
بررسی کرده است. اما شما تکلیف خود را روشن کنید. اگر علاقه دارید به‌صورت
قانونی لاگ خود را سازماندهی کنید، این بخش از راهنما مشکلات مرتبط به
این موضوع را به شما معرفی می‌کند.

البته که این قوانین و مقوله ثبت گروه کاربری لاگ بعنوان یک سازمان غیر انتفاعی
برای کشور ایران مطرح نخواهد بود و معنایی نخواهد داشت. در ایران لاگ‌ها بعنوان شرکت
ثبت نمی‌شوند و درامد، مالیات یا بیمه برای آنها مطرح نیست.
(در صورتی که اطلاعات بیشتری دارید لطفا این راهنما را تکمیل کنید)
از همین رو اگر مایلید قوانین مختلف قانون گذاری برای ثبت لاگ‌ها را بدانید، می‌توانید
ادامه این بخش را مطالعه کنید.
%For whatever it's worth, this HOWTO's originator and second maintainer lean,
%respectively, towards the pro and con sides of the issue -- but choose
%your own poison:  If interested in formally organising your LUG, this
%section will introduce you to some relevant issues.

{\bfseries نکته:}
تصور نکنید که این بخش می‌تواند مشاوره حقوقی کافی و حرفه‌ای به شما بدهد.
این مسئله نیاز به تخصص یک مشاور حقوقی حاذق خواهد داشت. قبل از انجام هر کاری
بر اساس گزاره‌های این بخش، باید با وکیل مشورت کنید.
%{\bfseries Note:} this section should not be construed as competent legal
%counsel. These issues require the expertise of competent legal
%counsel; you should, before acting on any of the statements made in
%this section, consult an attorney.

از آنجا که قوانین هر کشور متفاوت است، مورد به مورد بررسی خواهیم کرد.
\subsubsection{کانادا}
%\subsubsection{Canada}
از
\fftnt{کریس برواون}{mailto:cbbrowne@cbbrowne.com}
بابت نظرات و مطالبی که راجع به وضعیت کانادا با ما در اختیار گذاشت ممنون هستیم.

%Thanks to \ifpdf
%\href{mailto:cbbrowne@cbbrowne.com}{Chris Browne}%
%\else
%\onlynameurl{Chris Browne}%
%\fi{}
% for the following comments about the Canadian situation.

قوانین مالیاتی کانادا شباهت بسیاری به آمریکا دارد که در ادامه خواهید دید.
در کانادا مزایای مالیاتی مشابهی برای "سازمان‌های خیریه" و "سازمان‌های ناسودبر"
وجود دارد. با اینحال برای بدست آوردن و نگهداشتن وضعیت مجوز خیریه نیاز دارید
یک سری گزارشات و مدارک به مقامات مالیاتی ارسال کنید.
%The Canadian tax environment strongly parallels the US environment (for which,
%see below), in that the "charitable organisation" status confers similar tax
%advantages for donors over mere "not for profit" status, while
%requiring that similar sorts of added paperwork be filed by the
%"charity" with the tax authorities in order to attain and maintain
%certified charity status.

\subsubsection{آلمان}
%\subsubsection{Germany}

خبرنگاری به نام
\fftnt{توماس کپلر}{mailto:Thomas.Kappler@stud.uni-karlsruhe.de}
هشدار داد که فرایند تاسیس یک موجودیت غیرانتفاعی در آلمان، مقداری سخت است.
%Correspondent \ifpdf
%\href{mailto:Thomas.Kappler@stud.uni-karlsruhe.de}{Thomas Kappler}%
%\else
%\onlynameurl{Thomas Kappler}%
%\fi{}
% warns that the process of founding a non-profit entity in Germany
%is a bit complicated, but comprehensively covered at 
%http://www.buergergesellschaft.de/?id=106947 (unfortunately not archived; please
%advise this HOWTO's maintainer of any current version or successor text).

\subsubsection{سوئد}
%\subsubsection{Sweden}
در سوئد نیازی نیست که لاگ‌ها ثبت بشوند اما آنها به‌عنوان کلاب
درنظر گرفته می‌شوند. اداره مالیات سوئد دو گزینه را ارائه می‌دهد:
سازمان ناسودبر یا
"\tfftnt{انجمن اقتصادی}{Economical Association}".
مورد دوم مربوط به سازمان‌هایی
است که هدف در آن نفع اقتصادی اعضامی‌باشد و چنین نوع سازمانی مناسب
یک لاگ نیست. به‌طور سنتی کمپانی‌ها، موسسات اعتباری/تعاونی و مانند آنها
از نوع "انجمن اقتصادی" به‌حساب می‌آیند.
%In Sweden, LUGs are not required to register, but then are regarded as
%clubs. Registration with Skatteverket (national tax authority) offers
%two classification options: non-profit organisation or "economical
%association". The latter is an organisation where the goal is to benefit
%its members economically, and as such is probably unsuitable, being
%traditionally used for collectives of companies, or building societies
%/ co-operative tenant-owners, and such).

سازمان‌های ناسودبر در سوئد قانون مشخصی ندارند که بتوان آن را دنبال کرد.
لبته قوانین عمومی سوئد اعمال خواهند شد. سازمان‌های ناسودبر می‌توانند نیرو
استخدام کنند و حتی می‌توانند کسب‌درامد کنند و عموما در قبال درآمدشان مالیات
پرداخت نخواهند کرد. توجه کنید که در سازمان‌های ناسودبر، برخلاف انجمن اقتصادی،
اعضا چیزی از درامد کسب‌وکار عایدشان نمی‌شود و درامد یک سازمان ناسودبر برای همان
سازمان و اهدافش خرج خواهد شد.
برای ایجاد کسب‌وکار ابتدا باید از طریق اداره مالیات سوئد ثبت شوید تا
شماره سازمانی دریافت کنید. این کار به گروه اجازه می‌دهد که پرداختی های خود را
قانونا انجام دهد.در غیر این صورت مجبورید امور تجاری و مالی را به دوش یک شخص
بسپارید.
البته این هم ممکن است که پس از دریافت یک شماره سازمانی، از دولت برای سازمان
ناسودبر خود درخواست حمایت مالی کنید.
%Non-profit organisations in Sweden doesn't have specific laws to follow.
%Rather, general Swedish law applies: They can hire people and they can
%make profit. Generally they don't pay tax on their profits. (Profits
%stay in the organisation; unlike the case with "economical associations",
%members don't receive business proceeds.) To be able to do business, you
%must register with Skatteverket to get an "organisation number", allowing
%the group to pay and get paid. Otherwise you will probably have to
%arrange business through a member in his/her individual capacity.
%It may then also be possible, after securing an organisation number to
%apply for government financial support.

\subsubsection{ایالات متحده آمریکا}
%\subsubsection{United States of America}
یک لاگ در آمریکا حداقل دو وضعیت قانونی را می‌تواند دریافت کند:
%There are at least two different legal statuses a LUG in the USA may
%attain:

\begin{enumerate}
\item
ثبت رسمی شدن به‌عنوان یک موجودیت ناسودبر و غیرانتفاعی
%\item incorporation as a non-profit entity
\item
معاف از مالیات
%\item tax-exemption
\end{enumerate}

با اینکه در ایالت‌های مختلف قوانین متفاوت هستند اما در اکثر ایالت‌ها
گروه‌های کاربری می‌توانند بعنوان یک سازمان غیرانتفاعی و ناسودبر ثبت شوند.
مزایای ثبت شرکتی لاگ شامل کم شدن مسئولیت‌های منفعل اعضا و داوطلبان لاگ
و همچنین معافیت از مالیات‌های فرنچایز شرکتی می‌باشد.
(البته این موارد مشکلات واقعی و دغدغه های اصلی لاگ نیست.
بخش "باورهای رایج رد شده" را ببینید.)
%Although relevant statutes differ among states, most states
%allow user groups to incorporate as non-profit entities.
%Benefits of incorporation for a LUG include limitations of liability
%of LUG members and volunteers ({\itshape but only in their passive roles
%as member/shareholders, not as participants\/}), as well as 
%limitation or even exemption from state corporate franchise taxes
%(which, however, is highly unlikely to be a real concern -- see 
%"Common Misconceptions Debunked", below).
پیش از اقدام برای ثبت لاگ بعنوان یک سازمان غیرانتفاعی حتما
باید با یک مشاور حقوقی حاذق مشورت کنید. با اینحال می‌توانید با مطلع شدن
از یکی سری مسائل در این مورد، هزینه این مشورت را کاهش بدهید.
پیشنهاد من کتاب
\ftnt{Non-Lawyers Non-Profit Corporation}{https://www.alibris.com/The-Non-Lawyers-Non-Profit-Corporation-Kit-Kermit-Burton/book/4711443}
از کرمیت برتون است. (البته این کتاب از سال ۱۹۹۶ ویرایش نشده است.
اما دستورالعمل‌های کلی آن مفید هستند. مخصوصا ویرایش سال ۱۹۹۲ آن
را می‌توانید رایگان از
\fftnt{اینترنت آرشیو}{https://archive.org/details/alphacorporation00kerm}
دریافت کنید.)
%While you should consult competent legal counsel before incorporating
%your LUG as a non-profit, you can probably reduce your legal
%fees by being acquainted with relevant issues before consulting
%with an attorney. I recommend Kermit Burton's {\itshape 
%\emph{Non-Lawyers' Non-Profit Corporation Kit} \texttt{https://www.alibris.com/The-Non-Lawyers-Non-Profit-Corporation-Kit-Kermit-Burton/book/4711443}
%\/}.  (Be warned that this work has not been
%updated since 1996, but its general guidelines are good.  The prior
%1992 edition can be read for free at {\itshape 
%\emph{Internet Archive} \texttt{https://archive.org/details/alphacorporation00kerm}
%\/}.)

حالت دیگری نیز وجود دارد که در آن اداره مالیات داخلی \lr{IRS} برای لاگ
معافیت مالیاتی درنظر نمی‌گیرد. گاها سازمان های غیرانتفاعی هم مشمول مالیات
خواهند شد و ناسودبر بودن یک تشکیلات الزاملا تضمین نمی‌کند که لاگ شما از مالیات
معاف خواهد بود. کاملا طبیعیست اگر یک سازمان ناسودبر از مالیت معاف نشود.
%As for the second status, tax-exemption, this is not a legal status, so
%much as an Internal Revenue Service judgement. It's important to realise
%non-profit incorporation {\bfseries does not} ensure that IRS will rule
%your LUG tax-exempt.  It is quite possible for a non-profit corporation
%to {\bfseries not} be tax-exempt.

\lr{IRS}
مستنداتی دارد که به سادگی توضیح می‌دهد معیار و روندی برای
معاف مالیاتی وجود دارد.
نشریه ۵۵۷ با نام
{\itshape وضعیت معافیت مالیاتی برای سازمان‌ها}
به صورت فایل آکروبات از وبسایت \lr{IRS} قابل دسترس است.
شدیدا توصیه می‌کنم قبل از اقدام برای ثبت یک شرکت غیرانتفاعی
این مستند را بخوانید.
درحالی که یک سازمان غیرانتفاعی بودن نمی‌تواند معاف مالیاتی شدن
شما را تضمین کند، اما برخی از روش‌های ثبت شرکت باعث می‌شود که
قطعا مشمول مالیات شوید. در نشریه مذکور به وضوح شروط لازم برای
مشمول معاف مالیاتی شدن شما قید شده است.

%IRS has a relatively simple document explaining the criteria
%and process for tax-exemption.
%It is {\bfseries Publication 557:}
%{\itshape Tax-Exempt Status for Your Organization\/}, available as
%an Acrobat file from the IRS's Web site. I strongly recommend
%you read this document {\bfseries before} filing for non-profit incorporation.
%While becoming a non-profit corporation cannot
%ensure your LUG will be declared tax-exempt, some
%incorporation methods will {\bfseries prevent} IRS from declaring your
%LUG tax-exempt. {\itshape Tax-Exempt Status for Your Organization\/}
%clearly sets out necessary conditions for your LUG to be declared
%tax-exempt.

نهایتا منابعی در اینترنت وجود دارند که به سازمان‌های ناسودبر و معافیت مالیاتی
اشاره دارند. برخی از آنها می‌توانند به لاگ هم مربوط باشند.
%Finally, there are resources available on the Internet for non-profit
%and tax-exempt organisations. Some of the material is probably
%relevant to your LUG.

یک سری باورهای اشتباه راجع به مالیات و قوانین مرتبط به سازمان‌های ناسودبر
وجود دارد که قصد داریم آنهارا تکذیب کنیم و حقیقت را شرح دهیم.

%Common Misconceptions Debunked:

\begin{itemize}
\item
فرایند ایجاد شخصیت حقوقی برای یک شرکت و مقوله معاف مالیاتی
دو مسئله جدا هستند. الزاما هردو باهم نخواهند بود. برای معاف از مالیات
بودن نیازی نیست الزاما شما شرکتی را ثبت کنید. (البته استثنا وجود دارد.
برای مثال
\fftnt{۵۰۱(سی)(۳)}{https://en.wikipedia.org/wiki/501(c)(3)_organization})
و همچنین برای ثبت شرکت نیازی نیست حتما معاف مالیاتی شوید.
(البته قطعا شما می‌خواهید معاف از مالیات شوید.)
%\item Incorporation and tax-exempt status are separate issues.
%You don't have to be incorporated to get recognition of tax-exempt status
%(except it's required for one tax-exempt category, 501(c)(3)).
%You don't have to be tax-exempt to be incorporated.
%(Odds are, you honestly won't want either.  You just probably assume you do.)

\item
"حفاظ مسئولیتی" که ثبت رسمی شرکت می‌تواند ایجاد کند، داوطلبان را از
مسئولیت‌های قانونی آنها محافظت نمی‌کند. این حفاظ مسئولیتی از شکایت و
بروز مشکل برای سهامداران (که در مورد ما منظور اعضای لاگ هستند) جلوگیری
می‌کند. در صورتی که شرکت خسارتی ایجاد کرده و شکایاتی برای آن ثبت شده است،
سهامداران آن از این شکایت در امان هستند. شکایات و جریمه‌ها به دارایی همان
سازمان محدود می‌شود و جریمه سنگین تر از دارایی سازمان به آنها تعلق نخواهد گرفت.
البته که داوطلبان همچنان مسئولیت‌های قانونی خود را دارند و نباید مرتکب به
انجام خطایی بشوند. این حفاظ قانونی صرفا داوطلبان و اعضای لاگ را از مسائل
حقوقی مرتبط به لاگ جدا می‌کند. جرم لاگ، الزاما ارتباطی با اعضای آن ندارد.
%\item The "liability shield" one can get from incorporating {\itshape doesn't 
%protect volunteers from legal liability\/}. All it does is prevent any 
%plaintiffs from suing individual shareholders (LUG members, in this case)
%for tort damages {\itshape merely because they own the corporation\/}, if the 
%corporation itself is alleged to have wronged the plaintiff.  Plaintiff's 
%maximum haul in damages from suing the corporation is limited to the 
%corporate net assets, in that one case.  However, volunteers are still
%fully liable for any personal involvement they're alleged to have had.

\item

\fftnt{پوشش بیمه طرح دعوی در دادگاه یا چتر بیمه}{https://en.wikipedia.org/wiki/Umbrella_insurance}
بسیار گران است. لاگ نمی‌تواند داوطلبان را تحت پوشش این بیمه قرار دهد و
از پس هزینه‌های سنگین آن بر نخواهد آمد. حتی اگر بتواند آن را اخذ کند.
برای اینکه یک میلیون دلار بیمه پوشش مسئولیت خریداری کنید، سالیانه باید
۲۵۰۰ دلار پرداخت کنید. البته با این مبلغ تنها رهبران و مدیران اصلی را تحت
پوشش قرار می‌دهد و کاملا وظایف محدودی را انجام خواهد داد.
%\item Umbrella insurance coverage against tort liability (i.e., against civil litigation)
%for your volunteers almost certainly costs far too much for your group to afford
%(think \$2,500 each and every year in premium payouts, give or take,
%to buy \$1M in general liability insurance coverage -- which generally would cover
%only the corporation as a whole and its directors in the strict performance of
%their defined duties), if you can find it at all.

\item
حتی اگر \lr{IRS} یک گروه راه رسما معاف از مالیات کند، الزاما به این معنی نیست
که حمایت‌های مالی هم بدون مالیات هستند. برا اساس مفاد سازمان ۵۰۱ (سی)(۳) کسرپذیری
خودکار مالیات فقط به سازمان‌هایی که فقط ماهیت خیریه دارند تعلق می‌گیرد که این امر
بسیار فعالیت‌های لاگ را محدود می‌کند (برای مثال نمی‌توانید رویدادهای ناقض حقوق کپی‌رایت
برگزارکنید یا از برخی جنبش‌های سیاسی حمایت کنید). در این حالت لاگ بسیار باید مطیع قوانینی
خاص باشد. درواقع وضعیت خیریه ۵۰۱ (سی)(۳) اصلا برای لاگ نمیتواند مفید باشد.
همراه با ذکر این موضوع که کمک‌های مالی می‌تواند کسرپذیر باشند، معایب این قوانین نیز
باید ذکر شود. اکثر لاگ‌ها ترجیح می‌دهند که بعنوان یک "باشگاه اجتماعی و تفریحی" شناخته شوند.
این مورد در دسته‌بندی
\fftnt{۵۰۱ (سی)(۷)}{http://web.archive.org/web/20090818124349/http://www.t-tlaw.com/lr-06.htm}
قید شده است.
%\item IRS recognition as a tax-exempt group doesn't mean donations to
%your group necessarily become tax-deductible:  Automatic deductibility is
%reserved to {\itshape charities only\/}, IRS category 501(c)(3), which must obey 
%extremely stifling restrictions on group activities (e.g., it would then 
%become illegal to host anti-DMCA events or support any other political
%activity), and must meet exacting paperwork and auditing standards.  It's 
%difficult to envision 501(c)(3) charity status actually making functional 
%sense for any Linux group -- though one continually hears it recommended by
%those who imagine being able to tell people their donations will be guaranteed
%tax deductible must justify any accompanying disadvantages.
% Most LUGs would more logically file (if at all) for recognition as a "social and recreation club", category 
%\emph{501(c)(7)} \texttt{http://web.archive.org/web/20090818124349/http://www.t-tlaw.com/lr-06.htm}
%.

\item
ثبت رسمی شدن به‌عنوان یک سازمان خیریه آن هم برای یک گروه کاربری لینوکس غیر رسمی و کوچک
هیچ توجیه و دلیلی ندارد. \lr{IRS} اهمیتی به سازمان‌هایی که درآمد سالانه زیر ۲۵۰۰۰ دلار دارند نمی‌دهد
و آنها مشمول مالیات نخواهند شد. از همین رو برای یک گروه کاربری کوچک هیچ نیازی به ثبت رسمی نخواهد بود.
\LTRfootnote{\url{http://www.guidestar.org/news/features/990_myths.jsp}}
%\item In any event, unless one wishes to become a registered charity to render incoming donations tax-deductible, there is {\itshape literally no point\/} in applying for  IRS  recognition of your small, informal Linux group under any of the Internal Revenue Code section 501(c) tax-exempt statuses, because IRS simply doesn't care about groups with annual gross revenues less than \$25,000, and 
%\emph{doesn't want to hear from them} \texttt{http://www.guidestar.org/news/features/990_myths.jsp}
%  (2010 update:  IRS now does require a very simple annual 
%\emph{e-Postcard} \texttt{http://epostcard.form990.org/}
% informational filing from all small non-profits, to keep their 501(c) certifications, but still doesn't want tax from them).

\item
قانون حمایت از داوطلبان فدرال در سال 1997 به طور کلی از داوطلبان سازمان‌های خیریه 501(c)(3) در برابر دعاوی حقوقی محافظت نمی‌کند. این قانون فقط برخی دفاع‌های قانونی را در طول دعاوی مدنی (که می‌تواند هزینه‌بر باشد) فراهم می‌کند و دارای محدودیت‌ها و نقاط ضعفی است. در بیشتر ایالت‌ها، این حمایت تنها در صورتی ارائه می‌شود که گروه مورد نظر بیمه مسئولیت زیادی داشته باشد. همچنین، اگر وظایف داوطلب به دقت تعریف و نظارت نشود و ادعای خسارت در محدوده این وظایف نباشد، هیچ حمایتی وجود ندارد. علاوه بر این، اگر حادثه شامل وسیله نقلیه موتوری، هواپیما یا کشتی باشد که نیاز به مجوز دارد، یا داوطلب قوانین ایالتی یا فدرال را نقض کند، باز هم هیچ حمایتی وجود نخواهد داشت. نکته مثبت این است که داوطلب لازم نیست حتماً عضو گروه باشد تا تحت پوشش قرار گیرد؛ قانون به وضوح بیان می‌کند که هر کسی که خدمات مشخصی را برای یک گروه واجد شرایط انجام دهد و بابت آن دستمزد نگیرد، می‌تواند داوطلب باشد.

\fftnt{قانون حمایت از داوطلبان فدرال در سال 1997}{http://www.congress.gov/cgi-bin/query/C?c105:./temp/~c105ss2v68}
داوطلبان سازمان‌های خیریه ۵۰۱(سی)(۳) را در برابر دعاوی حقوقی محافظت نمی‌کند.
\LTRfootnote{\url{http://www.runquist.com/article_vol_protect.htm}}
این قانون فقط برخی دفاع‌های قانونی محدود را در حین دادخواهی مدنی البته با ضعف‌های بسیار
فراهم می‌کند که می‌تواند هزینه‌بر باشد. در بیشتر ایالت‌ها فقط در صورتی این قانون به شما کمک می‌کند
که هزینه کلانی برای بیمه مسئولیت آن پرداخت کرده باشید که قبل تر مشاهده کردید. 
علاوه بر این تا زمانی که وظایف داوطلبان به دقت مشخص و نظارت نشوند و ادعای خسارت دقیقا مرتبط
با همان شرح وظایف نباشد، این قانون هیچ حمایتی از داوطلب نخواهد کرد.  اگر حادثه با موتور سیکلت،
هواپیما یا کشتی باشد نیاز به مجوز دارد. همچنین اگر داوطلب قوانین ایالتی یا فدرال را نقض کند نیز
حمایتی از طرف قانون دریافت نخواهد کرد.
البته نکته مثبت این قوانین این است که برای اینکه یک فرد مشمول پوشش حمایت این قوانین بشود
نیازی نیست حتما عضو گروهی باشد. کافیست بدون دریافت پول کاری را برای این گروه‌ها انجام بدهد
و از همین رو داوطلب خواهد بود و مشمول پوشش این قوانین می‌باشد.
%\item The
%\emph{Federal Volunteer Protection Act of 1997} \texttt{http://www.congress.gov/cgi-bin/query/C?c105:./temp/~c105ss2v68}
%does
%\emph{not} \texttt{http://www.runquist.com/article_vol_protect.htm}
%, in fact, shield volunteers of Internal Revenue Code section 501(c)(3) charities from tort lawsuits.
%At most, it furnishes some legal defences that can be raised during (expensive) civil litigation,
%with a large number of holes and limitations, and in most states will be denied unless the group also
%carries large amounts of (very expensive -- see above) liability insurance.  Also, unless the volunteer's duties
%are not very meticulously defined and monitored, and the alleged tort occurs strictly in the scope of those duties,
%there's no shield at all -- plus the litigated action must not involve a motor vehicle / aircraft / vessel requiring
%an operator's licence, nor may the volunteer be in violation of any state or Federal law, else again there's no shield at all.
%(On the bright side, it's completely false, as often alleged, that the volunteer must be a member of the group, to be covered:
%In fact, the Act clearly states that a volunteer may be anyone who performs defined services
%for a qualifying group and receives no compensation for that labour.)
\end{itemize}

طبق گفته تعدای از لاگ‌های ثبت رسمی لاگ و حقوقی ساختن ماهیت آن، نه‌تنها مزیتی ندارد بلکه با مشکلات
و معایب بسیار سخت و محدود کننده ای همراه است. درگیر شدن با این قوانین محدودیت‌ها و بروکراسی
شدیدی را به لاگ تحمیل خواهد کرد. به‌طور کلی در انتخاب مشاور مالی و مالیاتی دقت کنید و مطمعن باشید
از متخصصین و مشاورات حقوقی حاذق مشورت خواسته ایند. (یکی از نویسندگان و مدیران این راهنما در گذشته
بعنوان شغل اول خود یک مشاور حرفه‌ای مالیت و امور مالی بوده است. اما اگر نیاز به مشاوره دارید، لطفا
از فردی مشورت بگیرید که درحال حاظر در همین حیطه فعال است.)
%As may be apparent from the above, a number of groups have, in the past, talked themselves into
%unjustifiable levels of bureaucratic strait-jacketing with no real benefit and serious ongoing disadvantages
%to their groups, because of misconceptions, careless errors, and tragically bad advice in the above areas.
%In general, you should be slow to heed the counsel of amateur financial and tax advisors.
%(This HOWTO's maintainer had past experience during his first career as a {\itshape professional\/}
%finance and tax advisor, but, if you need competent advice tailored to your situation, please have a consultation
%with someone currently working in that field.)

\subsection{مسائل قانونی دیگر}
%\subsection{Other legal issues}

\subsubsection{قاچاق نرم‌افزار}
%\subsubsection{Bootlegging}
همانطور که به یاد دارید در جلسات لاگ و انجمن‌های آنلاین گنو/لینوکس باید از پخش و استفاده غیر قانونی
از نرم‌افزارهای و سیستم‌عامل‌های انحصاری شدیدا جلوگیری شود. نقض قانون باید در لاگ
\fftnt{خارج از موضوع}{Off-topic}
باشد. البته که چنین مشکلی به‌طور کلی زیاد پیش نمی‌آید (حتی در میان کاربران سیستم‌عامل‌های انحصاری
نیز این مشکل کمتر هم است) اما برای مثال در یکی از لاگ‌هایی که می‌شناختم، برای نصب دوال بوت سیستم‌عامل
در یکی از جشن های نصب، لاگ از یک کپی از نرم‌افزار 
\fftnt{پارتیشن مجیک}{https://en.wikipedia.org/wiki/PartitionMagic}
برای تمام سیستم‌ها استفاده کرد. این امر بسیار از نظر قانونی مشکوک است و بنظر غیرقانونی است.

%As a reminder, it's vital that offers or requests to copy
%distribution-restricted proprietary software of any sort be heavily
%discouraged anywhere in LUGs, and banned as off-topic from all GNU/Linux user
%group on-line forums. This is not generally even an issue -- much less
%so than among proprietary-OS users -- but (e.g.) one LUG of my
%acquaintance briefly used a single LUG-owned copy of PowerQuest's
%Partition Magic on all NTFS-formatted machines brought to its
%installfests for dual-boot OS installations, on a very dubious theory
%of legality.

اگر احساس می‌کنید چیزی غیر قانونی است، مطمعن باشید همینطور است. مراقب باشید مرتکب اشتباه نشوید.
%If it smells unlawful, it almost certainly is. Beware.

\subsubsection{قوانین ضد انحصار}
صحبت راجع به کسب‌وکار در لاگ مانعی ندارد. اما از نظر قوانین ضد انحصار و مبارزه با انحصارطلبی، صحیح نیست
که در لاگ راجع به اینکه "چقدر برای انجام فولان کار دریافت می‌کنید؟" گفت‌وگو کنید.
%It's healthy to discuss the consulting business in general in user
%group forums, but for antitrust legal reasons it's a bad idea to get into 
%"How much do you charge to do {[}foo]?" discussions, there.

\subsection{سیاست‌های نرم‌افزاری}
%\subsection{Software politics}
\fftnt{کریس براون}{mailto:cbbrowne@cbbrowne.com}
در این بخش راجع به سیاست‌های پویای درون لاگ صحبت کرده که اغلب پیش می‌آیند.
(نگهدارنده این راهنما نیز تغییرات کوچکی را در این بخش اعمال کرده است):

%\emph{Chris Browne} \texttt{mailto:cbbrowne@cbbrowne.com}
% has the
%following to say about the kinds of intra-LUG political dynamics that
%often crop up (lightly edited and expanded by the HOWTO maintainer):

\subsubsection{مردم احساسات متفاوتی راجع به نرم‌افزار آزاد/متن‌باز دارند}
%\subsubsection{People have different feelings about free / open-source software}
کاربران گنو/لینوکس آنقدر متفاوت هستند که وقتی تلاش می‌کنید آنها را درکنار هم جمع کنید، یک سری
مشکلات رخ می‌دهند. برخی از آنها عقاید فراطی دارند. معتقدند تمام نرم‌افزارها همیشه باید "آزاد" باشند.
باور دارند چون سازمان‌هایی مانند
\tfftnt{کلدرا}{Caldera}، \tfftnt{رد هت}{Red Hat} و \tfftnt{سوزه}{SUSE}
مبلغ زیادی بابت توزیع‌های خود دریافت می‌کنند و تمام سود خود را صرف حمایت از نرم‌افزار آزاد نمی‌کنند،
پس سازمان‌هایی شیطانی هستند.
به‌یاد داشته باشید که این سه کمپانی تاثیر بسیار قابل توجهی در مشارکت در نرم‌افزار آزاد/متن‌باز دارند.
%GNU/Linux users are a diverse bunch.  As soon as you try to put a lot of
%them together, {\itshape some\/} problem issues can arise. Some, who are
%nearly political radicals, believe all software, always, should be
%"free".  Because Caldera charges quite a lot of money for its
%distribution, and doesn't give all profits over to {\itshape (pick favorite
%advocacy organisation)\/}, it must be "evil".  Ditto Red Hat or
%SUSE.  Keep in mind that all three of these companies have made and
%continue to make significant contributions to free / open-source software.

(نکاتی اضافی: مطالب بالا مربوط به سال ۱۹۹۸ هستند. پس از این تاریخ سیستم کلدرا کسب‌وکار گنو/لینوکس
خود را پایان داد و نام خود را به گروه \lr{SCO} تغییر داد. پس از آن یک دعوی حقوقی بزرگ در رابطه با
کپی‌رایت، قراردادها، ثبت اختراع‌ها و اسرار تجرای و همچنین یک کمپین روابط عمومی
\footnote{
یک کمپین روابط عمومی می‌تواند تاثیرات طولانی مدتی بر شهرت و تصویر عمومی کسب‌وکارها داشته باشد؛
اما باید به‌خوبی و به‌صورت حرفه‌ای برنامه‌ریزی شود. کمپین‌های PR چیزی بیشتر از انتشار یک بیانیه مطبوعاتی
در مورد یک محصول جدید هستند و اگر به‌درستی انجام شوند می‌توانند توجه رسانه‌های زیادی را جلب کند،
فروش را افزایش دهد و یک ارتباط مثبت بین برند و مخاطب ایجاد کند. ما در این مقاله، معنی کمپین روابط عمومی،
انواع کمپین‌ها، نکات اجرایی مربوط به آن و چند نمونه موفق از کمپین‌های PR را بررسی کرده‌ایم.
}
در مقابل کاربران گنو/لینوکس راه‌اندازی کرد. گذر زمان تغییر ایجاد می‌کند.
البته هنوز هم از کلدرا بابت اهدای سخت‌افزار برای کمک به توسعه پشتیبانی کرنل از \lr{SMP} توسط
\tfftnt{آلن کاکس}{Alan Cox}،
کمک‌های مالی به توسعه \lr{RPM} ، مشارکت‌های گسترده قدیمی او در سورس کرنل و تلاش‌های او
برای ترکیب گنو/لینوکس با کدبیس قدیمی یونیسک متشکریم.
)
%(HOWTO maintainer's note:  The above was a 1998 note, from before
%Caldera Systems exited the GNU/Linux business, renamed itself to The SCO Group,
%Inc., and launched a major copyright / contract / patent / trade-secret
%lawsuit and PR campaign against GNU/Linux users.  My, those times do change.
%Still, we're grateful to the Caldera Systems that {\itshape  was \/}, for
%its gracious donation of hardware to help Alan Cox develop SMP kernel
%support, for funding the development of RPM, and for its extensive past
%kernel source contributions and work to combine the GNU/Linux and historical
%Unix codebases.)
برخی تصور می‌کنند که می‌توانند از "آزادی" گنو/لینوکس برای تفریح یا سود سو استفاده کنند.
لایسنس  \lr{BSD} به کمپانی‌ها اجازه می‌دهد که کرنل و کتابخانه‌های سی را برای خود توسعه
داده و عمومی هم نکنند. برخی از کاربران این لاینسن را به لایسنس \lr{GPL} ترجیح می‌دهند.
لایسنس \lr{GPL} باعث می‌شود بالجبار نرم‌افزار برای همیشه آزاد بماند همانطور که این مسئله
برای کرنل و کتابخانه سی گنو صادق است. البته افرادی که بر این باورند الزاما طماع نیستند.


%Others may figure they can find some way to highly exploit the
%"freeness" of the GNU/Linux platform for fun and profit. Be aware that many
%users of the BSD Unix variants consider {\itshape their\/} licences that
%{\itshape do\/} permit companies to build "privatised" custom versions of
%their kernels and C libraries preferable to the "enforced permanent
%freeness" of the GPL as applied to the Linux kernel and GNU libc.  Do
%not presume that all people promoting this sort of view are necessarily
%greedy leeches.

همانطور که مشاهده کردید، دسته‌های مختفلی وجود دارند که مخالف یکدیگر هستند.
درصورتی که این افراد درکنار همدیگر جمع شوند ممکن است اختلاف نظرهایی نیز به‌وجود بیاید.
%If/when these people gather, disagreements can occur.

رهبران باید راجع به حقایق زیر شفاف باشند:
%%Leaders should be clear on the following facts:

\begin{itemize}
\item 
نظرات بسیاری راجع به لایسنس \lr{GPL} و حتی لایسنس‌های متن‌باز دیگر و کارکرد آنها وجود دارد.
(البته عمدتا این نظرات نادرست هستند و مردم درک درستی از \lr{GPL} و جایگزین‌های آن ندارند.
این بحث‌ها برای افرادی که دانش کافی ندارند واقعا بی‌فایده هستند و نیازی نیست به آنها پرداخته شود.
اما اکر افرادی بودند که واقعا مایل بودند این مطالب رو بیاموزند، لطفا آنها را به لیست پستی "\lr{license-discuss}" 
در \lr{OSI} و همچنین "\lr{debian-legal}" از پروژه دبیان ارجاع بدهید. آنها در این دو لیست پستی
می‌توانند مطالب و تحلیل‌های اساسی را ببینند.
)
%\item There are a lot of opinions about the GPL and other open-source
%licences and how they work -- mostly misinformed.  It is easy to
%misunderstand both the GPL and alternative licensing schemes.  Most
%attempts at debating same are, at root,  pointless, ritualised symbolic
%warfare among people who should know better.  In the rare event that
%participants actually aspire to understand the subject, please direct
%them to the OSI's "license-discuss" mailing list and the Debian
%Project's "debian-legal" mailing list, where substantive analysis is
%possible and encouraged.

\item 
گنو/لینوکس فقط از یک فرد یا دسته نفع نبرده است. درواقع گنو/لینوکس از افراد متنوعی
سود برده که شرکت‌های نرم‌افزاری انحصاری نیز از عضو آنها می‌باشند. شرکت های انحصاری
هم در کرنل لینوکس، \lr{X.org} و \lr{gcc} مشارکت کرده اند و از همین رو نمی‌توان آنها
را الزاما بد دانست.
%\item  GNU/Linux benefits from contributions from many places, including
%proprietary-software vendors, e.g., in the Linux kernel, X.org, and
%gcc.

\item 
انحصاری بودن یک چیز به معنای بهتر بودن یا وحشتناک بودن آن نیست.
%\item  Proprietary implies neither better nor horrible.
\end{itemize}

مدت‌هاست که یک جنگ بی پایان کامپیوتری شروع شده است. در این میدان نبرد
گنو/لینوکس، یونیکس‌های دیگر، سیستم عامل های مایکروسافت، کامپیوترهای آی بی ام،
سیستم‌های مختلف بر پایه موتورولا ۶۸۰۰۰، سیستم‌های ۸ بیتی سال ۱۹۷۰ در مقابل یکدیگر
می‌جنگند. این جنگ مقدس در دنیای نرم افزار رسوخ کرده است. به‌طوری که در
همین میدان \lr{KDE} در مقابل \lr{GOME} و هزاران نرم‌افزار دیگر رقابت می‌کنند.
%The main principle can be extended well beyond this; computer "holy
%wars" have long been waged over endless battlegrounds, including 
%GNU/Linux vs. other Unix variants vs. Microsoft OSes, the "IBM PC" vs.
%sundry Motorola 68000-based systems, the 1970s' varied 8-bit systems 
%against each other, KDE versus GNOME....

یک رهبر فرزانه در لاگ باید از این همه تفاوت در عقیده و سلیقه گذر کند چرا که بحث و جدل
بر سر این اختلافات واقعا خسته‌کننده و بی‌فایده است. رهبران لاگ از پس مدیریت این بحث ها 
بر می‌آیند. چرا که آنها بسیار پوست کلفت تر از این حرف ها هستند.
%A wise LUG leader will seek to move past such differences, if only
%because they're tedious.  LUG leaders ideally therefore will have thick skins.

\subsubsection{سازمان‌های ناسودبر و پول ترکیب خوبی باهم نمی‌سازند}
%\subsubsection{Non-profit organisations and money don't mix terribly well.}

مسائل مالی در یک سازمان ناسودبر بسیار مهم و زیر ذره‌بین هستند. مردم اهمیتی نمی‌دهد
که یک کسب‌وکار چقدر پول هدر داده است. اما برای سازمان‌های غیر انتفاعی این موضوع مهم می‌شود.
از نظر مردم مهم است که در یک سازمان ناسودبر مانند خیریه بودجه در چه راهی خرج شده است.
آیا واقعا در راستای همان ایدئولوژی که در ابتدا مشخص شده است خرج شده یا خیر؟
همانطور که سیریل نورث‌کوت پارکینسون می‌گوید
\LTRfootnote{\url{https://en.wikipedia.org/wiki/Law_of_triviality}}
اکثر مردم به‌جای مسائل مهم توجه و زمان خود را بر سر مسائل کم اهمیت می‌گذارند.
جلسات لاگ بسیار پتانسیل این اتلاف وقت و هزینه بی‌جا و اشتباه را دارند. درواقع بسیار محتمل است
که در لاگ، بودجه صرف مسائل بی اهمیت شود. این مسئله از دید مردم توجیهی ندارد.
%It is important to be careful with finances in any sort of non-profit.
%In businesses, which focus on substantive profit, people are not
%typically too worried about minor details such as alleged misspending of
%{\itshape immaterial\/} sums.  The same cannot be said of non-profit
%organisations.  Some people are involved for reasons of principle, and
%devote inordinate attention to otherwise minor issues, an example of C.
%Northcote Parkinson's 
%\emph{Bike Shed Effect} \texttt{http://linuxmafia.com/~rick/lexicon.html#bikeshed}
%. LUG business
%meetings' potential for wide participation correspondingly expands the
%potential for exactly such inordinate attention.

اکثر اوقات ترجیح بر این است که لاگ، هیچ هزینه ای را بابت حق عضویت از اعضای لاگ و داوطلبان دریافت نکند.
چرا که هزینه جمع‌آوری نشده هرگز مورد سواستفاده قرار نمی‌گیرد یا هدر نخواهد شد.
%As a result, it is probably preferable for there to {\itshape not\/} be any
%LUG membership fee, as that provides a specific thing for which people
%can reasonably demand accountability. Fees not collected can't be
%misused -- or squabbled over.

اگر پول یا دارایی زیادی در لاگ وجود دارد، لاگ موضف است در قبال آن  جواب‌گوی اعضا باشد.
%If there {\itshape is\/} a lot of money and/or other substantive property,
%the user group must be accountable to members.

یکی از لازمه‌های حیات یک گروه این است که افراد فعال زیادی را داشته باشد. اما در سازمان‌های غیرانتفاعی
مورد دار و مشکل‌دار، معمولا قدرت کنترل مالی آن سازمان به دست یک نفر است و این فرد هم به راحتی کنترل
خود را واگذار نخواهد کرد. وظایف تعریف شده در لاگ نباید همیشه به عهده افراد ثابتی باشد. ایده‌آل است که
وظایف مشخص شده بین افراد فعال و مدیران لاگ به‌صورت چرخشی سپرده شود.
این موضوع شامل کنترل مالی لاگ نیز می‌باشد. درواقع کنترل مالی لاگ نباید برای همیشه دست یک فرد باشد.
%Any vital, growing group should have more than one active person. In
%troubled nonprofits, financial information is often tightly held by
%someone who will not willingly relinquish monetary control. Ideally,
%there should be {\itshape some\/} LUG duty rotation, including duties
%involving financial control.

هر ساله لاگ باید یک گزارش مالی برای اداره مالیت آماده کند. مقامات مالیاتی با وجود اینکه لاگ را رسما
یک سازمان خیریه می‌شناسند، اما بازهم باید گزارشات مالی آن را بررسی کنند. شایان است کمی از این
گزارشات را برای اعضای لاگ نمایش دهیم تا آنها نیز از فعالیت‌های مالی لاگ باخبر شوند.
%Regular useful financial reports should be made available to those
%who wish them. A LUG maintaining official "charitable status"
%for tax purposes must file at least annual financial reports
%with the local tax authorities, which would represent a minimum
%financial disclosure to members.

با پیشرفت نرم‌افزارهای مالی در گنو/لینوکس، تهیه گزارشات مالی بسیار آسان و عملی خواهند بود. همچنین
با پیشرفت اینترنت انتشار آن در وب ممکن می‌باشد.
%With the growth of GNU/Linux-based financial software, regular reports are
%now quite practical. With the growth of the Internet, it should even be
%possible to publish these on the World-Wide Web.

\subsection{انتخابات، دموکراسی و جابجایی}
%\subsection{Elections, democracy, and turnover}

دموکراسی برای لاگ بسیار حیاتی است. البته اگر که به این موضوع باور داشته باشید. جلوتر به این موضوع
خواهیم پرداخت.
%Governing your LUG democratically is absolutely vital -- if and
%only if you believe it is.  I intend that remark somewhat less cynically
%than it probably sounds, as I shall explain.

اعضای لاگ می‌توانند اختلافاتی داشته باشند. برای مثال این اختلاف می‌تواند در برگزاری جلسه لاگ و یا توسعه یک نرم‌افزار
باشد. اما این مشکلات در لاگ بسیار بی اهمیت هستند. نیازی وجود ندارد که حتما مشکلات با یک روش رفع بشوند.
با رعایت دموکراسی در لاگ و حق انتخاب دادن به اعضای لاگ می‌توان از این مشکلات چشم‌پوشی کرد.
عده ای می‌توانند با روش خود مشکلات موجود را حل کنند. نیازی نیست که حتما یک راهکار برای یک مشکل ارائه شود.
برای مثال یک نرم‌افزار می‌تواند نسخه‌های متفاوتی داشته باشد و اعضا به دلخواه راهکارهای خود را در هر نسخه انجام دهند.
همینطور اگر دو گروه در لاگ در موردی باهم اختلاف‌نظر دارند، می‌توانند به‌صورت همزمان و مستقل خود روی ایده‌هایی
که تصور می‌کنند صحیح است زمان بگذارند. نیازی نیست حتما به توافقی برسند.
دموکراسی در لاگ‌ها به حدی شدید است که موقع انتخابات برای کاندیداهای راهبران لاگ و تقسیم وظایف سعی بر این است
که به تعداد هر داوطلب یا راهبر وظیفه ایجاد بشود. این امر باعث می‌شود که کنترل لاگ تقسیم شده و دموکراسی رعایت شود.
دموکراسی شدید لاگ باعث می‌شود که رقابت کمتر شده و ذاتا اعضای لاگ آن را کنترل کنند.

%Tangible stakes at issue in LUG politics tend to be minuscule to the point of
%comic opera:  There are typically no real assets. Differences of view 
%can be resolved by either engineering around them with technology (the GNU/Linux-ey
%solution) or by letting each camp run efforts in parallel. Moreover, even the
%most militantly "democratic" LUGs typically field, like clockwork,
%exactly as many candidates as there are offices to be filled -- not a
%soul more.

تا این حد رعایت دموکراسی و تمرین دموکراسی می‌تواند مضحک باشد و مسخره نکردن آن گاها سخت خواهد بود.
اما من قصد ندارم این دموکراسی شدید را مسخره کنم.
همانطور که اشاره کردم جذب و نگهداشتن داوطلبان کلیدی برای موفقیت لاگ حیاتی است و هرکاری
که سبب شود این اتفاق رخ دهد مطلوب است. بنظر می‌آید این دموکراسی به برخی گروه‌ها
شوک وارد می‌کند و ذاتا مشارکت کنندگان را تشویق می‌کند و به افرادی که برگزیده شده اند احساس
مشروعیت و مسئولیت می‌دهد و باعث می‌شود افراد بیشتر در لاگ مشارکت کنند. این بسیار خوب و مطلوب است.
%It's tempting to mock such exercises as empty posturing, but such
%is not (much) my intent.  Rather, I
%mention them to point out something more significant:  Attracting and 
%retaining key volunteers is vital to the group's success.  Anything that 
%makes that happen is good.  It seems likely that the
%"democratic" exercise stressed in some groups, substantive or not,
%encourages participation, and gives those elected a sense of status,
%legitimacy, and involvement.  Those are Good Things.  

به‌هرحال اگر انتخابات و ساختار رسمی در لاگ به جذب مشارکت‌کننده‌های کلیدی کمک می‌کند، از آن استفاده کنید.
اما اگر باعث می‌شود مشارکت‌کننده‌ها کمتر شوند بدیهی است  که از آن صرف‌نظر کنید. این مسئله برای مواردی
مانند جایزه‌هایی که به قید قرعه به شرکت‌کننده‌ها اعطا می‌شود
\LTRfootnote{\url{https://en.wikipedia.org/wiki/Door_prize}}
و بازارچه‌ها هم صادق است. مشارکت عضو حیاتی لاگ شما خواهد بود. هرکاری باعث مشارکت حداکثری شود
برای لاگ مفید و موجه خواهد بود.
%Thus, if elections and formal structure help attract key
%participants, use them.  If those deter participants, 
%lose them.  If door-prizes and garage sales bring people in, do
%door-prizes and garage sales.  Participation, as much as software, is
%the lifeblood of your LUG. 

بارها از واژه "داوطلبان کلیدی" استفاده کردیم. مگر تفاوتی بین داوطلبان وجود دارد؟ بله. نمی‌توانیم از این موضوع فرار
کنیم. تمام کارهای موجود و مورد نیاز به دست عده کمی انجام می‌شود. در گروه‌های داوطلبانه کارها اینگونه
پیش می‌رود. شاید ذکر یک داستان برای منتقل کردن حرفم به شما کمک کند:
من برای زمان طولانی حروف‌چین و ویرایشگر مجله گروه کاربری کامپیوتر سان فرانسیسکو بودم. هر ماه، ما یک مجله
۴۰ صفحه‌ای منتشر می‌کردیم. من در اواخر سابقه کاری خود برای دموکراتیک تر کردن فرایند مدیریت مجله تحت فشار بودم.
در واقع از سمت رئیس گفته می‌شد که باید دموکراسی را در مدیریت خود شدیدا رعایت کنم و این مرا تحت فشار قرار می‌داد.
نهایتا به رئیس باشگاه گفتم: "آن مرد آنجا را می‌بینی؟ او ادوارد، یکی از ویراستاران من است.
او ۱۲ مقاله با مشکلات مرتبط فعلی ما را ویرایش کرده است. بنابراین من تصور می‌کنم او باید ۱۲ رای در اداره این
مجله داشته باشد." ایده من این بود که هر شخص به اندازه کاری که انجام می‌دهد باید حق رای و تصمیم‌گیری
در مجله را داشته باشد. دلیلی وجود ندارد که تمام نیروها یک حق رای داشته باشند.
این روند ارزش‌گذاری کارمحور من ایده‌آل‌های دموکراتیک رئیس و دیگر مدیران را نقض می‌کرد. زیرا آنها معتقد بودند
هر دوپایی صرف‌نظر از مهارت‌‌های او باید قدرت و حق رای یکسانی داشته باشد.
در نهایت من تصور می‌کردم روزی خواهد رسید که رئیس تصمیم بگیرد که باشگاه به افرادی نیاز دارد که کار کنند و
این موضوع بیش از رعایت دموکراسی مهم هست. همیشه این دموکراسی نیست که کار را راه می اندازد.

%The reason I spoke of "key" volunteers, above, is because, inevitably, a
%very few people will do almost all of the needed work.  It's just the
%way things go, in volunteer groups. An anecdote may help illustrate my
%point:  Towards the end of my long tenure as editor and typesetter of
%San Francisco PC User Group's 40-page monthly magazine, I was repeatedly
%urged to make magazine management more "democratic".  I finally replied
%to the club president, "See that guy over there?  That's Ed, one of my
%editorial staff.  Ed just proofread twelve articles for the current
%issue.  So, I figure he gets twelve votes."  The president and other
%club politicos were dismayed by my work-based recasting of their
%democratic ideals: Their notion was that each biped should have an equal
%say in editorial policy, regardless of ability to typeset or proofread,
%or whether they had ever done {\itshape anything\/} to assist magazine
%production. Although he looked quite unhappy about doing so, the
%president dropped the subject.  I figured that, when it came right down
%to it, he'd decide that the club needed people who got work done more
%than they needed his brand of "democracy".

اما هنوز داستان تمام نشده بود: حدود یک ماه بعد من با یک کمیته انتشارات آشنا شدم.
تمام قصد و نیت آن خبرنامه این بود که برای سیاست‌های خبرنامه رای گیری کنند. درواقع
آنها آمده بودند که صرفا به داوطلبان دستورات اجرایی بدهند و خود کاری از پیش نبرند.
من در ابتدا با نهایت احترام به حرف آنها گوش کردم و بعد مثل همیشه انتقادهای فنی خودم راجع به
ویراستاری و مهارت خودم را وارد کردم. آنها از این مسئله بسیار متعجب شدند که من از پیشنهادات دستوری
آنها صرف‌نظر کردم و حرف خودم را زده ام. اما بد تر از آن، همچنان من به آنها کار محول کردم و از آنها
خواستم که خود نیز بخشی از کار های مجله را انجام بدهند. دقیقا مانند کارکنان دیگر خودم.
البته که آنها پذیرای نظرات بنده نبودند. این موضوع بسیار مشخص است. ریاست کردن و دستور دادن به دیگران
بسیار دلپذیر تر و جذاب تر از انجام کار واقعی است. چیزی که گروه نیاز دارد کار است، نه توصیه و مشاوره برای
بهبود قوانین دموکراسی.
%But we weren't quite done:  A month or so later, I was introduced to a 
%"Publications Committee", who arrived with the intent of doing nothing but 
%vote on matters of newsletter policy (i.e., issue "executive" orders to the 
%volunteer production staff).  Their first shock came when I listened politely 
%to their advice but then applied my editorial judgement as usual.  Much 
%worse, though:  I also assigned them work, as part of my staff.  Almost 
%all immediately lost interest. (Bossing around other people seemed likely 
%to be fun; doing actual work was not.)

هدف از این داستان این بود که بگویم تقاضای گسترده برای رای گیری راجع به هر چیز در بهترین حالت
لزوما برای انجام کار ها مفید نیست و در بد ترین حالت بهانه‌ای برای انجام ندادن کار و دخالت در تصمیمات
افرادی که دارند تلاش می‌کنند است. لزا دموکراسی و رای گیری در هر چیز، همیشه موجه نخواهد بود. گاها
کاملا اشتباه و باعث دور شدن از مسیری که باید خواهد بود.
%The point is that the widespread urge to vote on everything is at best
%orthogonal to any desire to perform needed work; at worst, the former
%serves as an excuse to compulsively meddle in others' performance
%of the latter.

برای جمع بندی باید بگوییم که هرچقدر که صلاح می‌دانید دموکراسی داشته باشید. این هیچ عیبی ندارد.
اما بدانید که مراقبت از داوطلبان کلیدی لاگ بسیار از هرچیز، حتی دموکراسی مهم تر است.
به‌قول آقای کاندید، فقط لازم است باید باغ خودمان را رشد دهیم.
%To sum up:  Have all the "democracy" that makes you happy, but watching after
%the well-being of your key volunteers is what matters.  (To quote Candide, 
%"We must cultivate our garden.")

با تمام این صحبت‌ها متوجه می‌شودیم که یک سری کارها را باید به صورت دیکتاتوری و شخصا بعنوان یک
داوطلب کلیدی انجام دهیم. اما روزی خواهد رسید که شما نخواهید بود و باید برای آن روز برنامه داشته باشید.

نهایتا برای جایگزین کردن خودتان برنامه بریزید: اگر شما یک گروه دانش‌آموزی در دانشگاه هستید، و برای
حفظ اعتبار خود و گروه، هرسال نیازمند دردسرهای اداری و کاغذبازی در دانشگاه خواهید بود، مطمعن شوید
که تمام این فرایندهای مهم برای نگهداشتن گروه مستند خواهد شد. اگر تمام فرایندهای لازم برای نگهداری
لاگ را مستند کنید، مدیران جدید لاگ نیازی ندارند که از ابتدا همه چیز را ایجاد کنند و می‌توانند به‌راحتی
راه شما را ادامه بدهند.
فرض کنید می‌خواهید یک سیستم طراحی کنید و به این موضوع مانند یک مسئله در مهندسی سیستم‌ها نگاه کنید.
شما تملاش می‌کنید که تمام راه‌ها برای شکست خوردن را ببندید و سعی کنید به مسیری با موفقیت تضمین شده برسید.
%Last, plan for your replacement:  If your LUG is a college student
%group, and must go through a paperwork deathmarch every year to stay
%accredited, make sure that and all other vital processes are documented,
%so new LUG officers needn't figure everything out from scratch.  Think
%of it as a systems-engineering problem:  You're trying to eliminate
%single points of failure.

هیچ راه تضمین شده و مشخصی وجود ندارد. راهکاری که برای عده‌ای کار خواهد کرد، برای شما ممکن است
اصلا جوابگو نباشد. درواقع هنوز عده ای از تکه‌های این پازل پیدا نشده است! از همین رو باید امتحان کنیم.
لطفا شما هم روش خود را امتحان کنید و ما را نیز در جریان بگذارید تا به دیگران آن را اطلاع بدهیم.
خوش بگذره!
\footnote{
اگر اطلاعاتی راجع به رعایت دموکراسی یا شیوه مدیریت لاگ محلی خود دارید لطفا به این مستند اضافه کنید.
}

%And what works for the guys in the next town may not work for your crowd:
%Surprise!  The keys to this puzzle are still being sought.  So, please
%experiment, and let me know what works for you, so I can tell others.
%Have fun!

