

\section{Legal and political issues}






\subsection{Organisational legal issues}

The case for formal LUG organisation can be debated:

{\itshape Pro:\/} Incorporation and recognised tax-exemption limits
liability and helps the group carry insurance.  It aids fundraising.
It avoids claims for tax on group income.

{\itshape Con:\/} Liability shouldn't be a problem for modestly careful
people.  (You're not doing skydiving, after all.)  Also, even
incorporated technical groups seldom carry liability insurance, and that
insurance is typically so narrow in coverage that almost nothing a LUG
does would be covered.  A corporate liability shield is little use for
such needs, either (limiting only the group's potential losses to the
equity stake of the owners, but conferring no immunity to anyone for 
deeds that person carries out).  Fundraising isn't needed for a group whose
activities needn't involve significant expenses.  (Dead-tree newsletters
are so 1980.)  Not needing a treasury, you avoid needing to argue over
it, file reports about it, or fear it being taxed away. Meeting space
can sometimes be gotten for free at ISPs, colleges, pizza parlours,
brewpubs, coffeehouses, computer-training firms, GNU/Linux-oriented
companies, hackerspaces, or other friendly institutions, and can
therefore be free of charge to the public.  No revenues and no expenses
means less need for organisation and concomitant hassles.

For whatever it's worth, this HOWTO's originator and second maintainer lean,
respectively, towards the pro and con sides of the issue -- but choose
your own poison:  If interested in formally organising your LUG, this
section will introduce you to some relevant issues.

{\bfseries Note:} this section should not be construed as competent legal
counsel. These issues require the expertise of competent legal
counsel; you should, before acting on any of the statements made in
this section, consult an attorney.






\subsubsection{Canada}

Thanks to \ifpdf
\href{mailto:cbbrowne@cbbrowne.com}{Chris Browne}%
\else
\onlynameurl{Chris Browne}%
\fi{}
 for the following comments about the Canadian situation.



The Canadian tax environment strongly parallels the US environment (for which,
see below), in that the "charitable organisation" status confers similar tax
advantages for donors over mere "not for profit" status, while
requiring that similar sorts of added paperwork be filed by the
"charity" with the tax authorities in order to attain and maintain
certified charity status.




\subsubsection{Germany}

Correspondent \ifpdf
\href{mailto:Thomas.Kappler@stud.uni-karlsruhe.de}{Thomas Kappler}%
\else
\onlynameurl{Thomas Kappler}%
\fi{}
 warns that the process of founding a non-profit entity in Germany
is a bit complicated, but comprehensively covered at 
http://www.buergergesellschaft.de/?id=106947 (unfortunately not archived; please
advise this HOWTO's maintainer of any current version or successor text).




\subsubsection{Sweden}

In Sweden, LUGs are not required to register, but then are regarded as 
clubs.  Registration with Skatteverket (national tax authority) offers 
two classification options:  non-profit organisation or "economical 
association". The latter is an organisation where the goal is to benefit 
its members economically, and as such is probably unsuitable, being 
traditionally used for collectives of companies, or building societies 
/ co-operative tenant-owners, and such).

Non-profit organisations in Sweden doesn't have specific laws to follow. 
Rather, general Swedish law applies: They can hire people and they can 
make profit. Generally they don't pay tax on their profits. (Profits 
stay in the organisation; unlike the case with "economical associations", 
members don't receive business proceeds.) To be able to do business, you
must register with Skatteverket to get an "organisation number", allowing 
the group to pay and get paid. Otherwise you will probably have to 
arrange business through a member in his/her individual capacity.  
It may then also be possible, after securing an organisation number to 
apply for government financial support.




\subsubsection{United States of America}

There are at least two different legal statuses a LUG in the USA may
attain:

\begin{enumerate}
\item incorporation as a non-profit entity
\item tax-exemption
\end{enumerate}


Although relevant statutes differ among states, most states
allow user groups to incorporate as non-profit entities. Benefits
of incorporation for a LUG include limitations of liability
of LUG members and volunteers ({\itshape but only in their passive roles
as member/shareholders, not as participants\/}), as well as 
limitation or even exemption from state corporate franchise taxes
(which, however, is highly unlikely to be a real concern -- see 
"Common Misconceptions Debunked", below).

While you should consult competent legal counsel before incorporating
your LUG as a non-profit, you can probably reduce your legal
fees by being acquainted with relevant issues before consulting
with an attorney. I recommend Kermit Burton's {\itshape 
\emph{Non-Lawyers' Non-Profit Corporation Kit} \texttt{\aelurl}
\/}.  (Be warned that this work has not been
updated since 1996, but its general guidelines are good.  The prior
1992 edition can be read for free at {\itshape 
\emph{Internet Archive} \texttt{\aemurl}
\/}.)

As for the second status, tax-exemption, this is not a legal status, so
much as an Internal Revenue Service judgement.  It's important to realise
non-profit incorporation {\bfseries does not} ensure that IRS will rule
your LUG tax-exempt.  It is quite possible for a non-profit corporation
to {\bfseries not} be tax-exempt.

IRS has a relatively simple document explaining the criteria
and process for tax-exemption. It is {\bfseries Publication 557:} {\itshape Tax-Exempt Status for Your Organization\/}, available as
an Acrobat file from the IRS's Web site. I strongly recommend
you read this document {\bfseries before} filing for non-profit incorporation.
While becoming a non-profit corporation cannot
ensure your LUG will be declared tax-exempt, some
incorporation methods will {\bfseries prevent} IRS from declaring your
LUG tax-exempt. {\itshape Tax-Exempt Status for Your Organization\/} clearly sets out necessary conditions for your LUG to be declared
tax-exempt.

Finally, there are resources available on the Internet for non-profit
and tax-exempt organisations. Some of the material is probably
relevant to your LUG.

Common Misconceptions Debunked:

\begin{itemize}
\item Incorporation and tax-exempt status are separate issues.  You don't have to be incorporated to get recognition of tax-exempt status (except it's required for one tax-exempt category, 501(c)(3)).  You don't have to be tax-exempt to be incorporated.  (Odds are, you honestly won't want either.  You just probably assume you do.)

\item The "liability shield" one can get from incorporating {\itshape doesn't 
protect volunteers from legal liability\/}.  All it does is prevent any 
plaintiffs from suing individual shareholders (LUG members, in this case) 
for tort damages {\itshape merely because they own the corporation\/}, if the 
corporation itself is alleged to have wronged the plaintiff.  Plaintiff's 
maximum haul in damages from suing the corporation is limited to the 
corporate net assets, in that one case.  However, volunteers are still
fully liable for any personal involvement they're alleged to have had.

\item Umbrella insurance coverage against tort liability (i.e., against civil litigation) for your volunteers almost certainly costs far too much for your group to afford (think \$2,500 each and every year in premium payouts, give or take, to buy \$1M in general liability insurance coverage -- which generally would cover only the corporation as a whole and its directors in the strict performance of their defined duties), if you can find it at all.

\item IRS recognition as a tax-exempt group doesn't mean donations to
your group necessarily become tax-deductible:  Automatic deductibility is
reserved to {\itshape charities only\/}, IRS category 501(c)(3), which must obey 
extremely stifling restrictions on group activities (e.g., it would then 
become illegal to host anti-DMCA events or support any other political
activity), and must meet exacting paperwork and auditing standards.  It's 
difficult to envision 501(c)(3) charity status actually making functional 
sense for any Linux group -- though one continually hears it recommended by
those who imagine being able to tell people their donations will be guaranteed tax deductible must justify any accompanying disadvantages.  Most LUGs would more logically file (if at all) for recognition as a "social and recreation club", category 
\emph{501(c)(7)} \texttt{\aenurl}
.

\item In any event, unless one wishes to become a registered charity to render incoming donations tax-deductible, there is {\itshape literally no point\/} in applying for  IRS  recognition of your small, informal Linux group under any of the Internal Revenue Code section 501(c) tax-exempt statuses, because IRS simply doesn't care about groups with annual gross revenues less than \$25,000, and 
\emph{doesn't want to hear from them} \texttt{\aeourl}
  (2010 update:  IRS now does require a very simple annual 
\emph{e-Postcard} \texttt{\aepurl}
 informational filing from all small non-profits, to keep their 501(c) certifications, but still doesn't want tax from them).

\item The 
\emph{Federal Volunteer Protection Act of 1997} \texttt{\aequrl}
 does 
\emph{not} \texttt{\aerurl}
, in fact, shield volunteers of Internal Revenue Code section 501(c)(3) charities from tort lawsuits.  At most, it furnishes some legal defences that can be raised during (expensive) civil litigation, with a large number of holes and limitations, and in most states will be denied unless the group also carries large amounts of (very expensive -- see above) liability insurance.  Also, unless the volunteer's duties are not very meticulously defined and monitored, and the alleged tort occurs strictly in the scope of those duties, there's no shield at all -- plus the litigated action must not involve a motor vehicle / aircraft / vessel requiring an operator's licence, nor may the volunteer be in violation of any state or Federal law, else again there's no shield at all.  (On the bright side, it's completely false, as often alleged, that the volunteer must be a member of the group, to be covered:  In fact, the Act clearly states that a volunteer may be anyone who performs defined services for a qualifying group and receives no compensation for that labour.)

\end{itemize}


As may be apparent from the above, a number of groups have, in the past, talked themselves into unjustifiable levels of bureaucratic strait-jacketing with no real benefit and serious ongoing disadvantages to their groups, because of misconceptions, careless errors, and tragically bad advice in the above areas.  In general, you should be slow to heed the counsel of amateur financial and tax advisors.  (This HOWTO's maintainer had past experience during his first career as a {\itshape professional\/} finance and tax advisor, but, if you need competent advice tailored to your situation, please have a consultation with someone currently working in that field.)




\subsection{Other legal issues}






\subsubsection{Bootlegging}

As a reminder, it's vital that offers or requests to copy
distribution-restricted proprietary software of any sort be heavily
discouraged anywhere in LUGs, and banned as off-topic from all GNU/Linux user
group on-line forums.  This is not generally even an issue -- much less
so than among proprietary-OS users -- but (e.g.) one LUG of my
acquaintance briefly used a single LUG-owned copy of PowerQuest's
Partition Magic on all NTFS-formatted machines brought to its
installfests for dual-boot OS installations, on a very dubious theory
of legality.

If it smells unlawful, it almost certainly is.  Beware.




\subsubsection{Antitrust}

It's healthy to discuss the consulting business in general in user
group forums, but for antitrust legal reasons it's a bad idea to get into 
"How much do you charge to do {[}foo]?" discussions, there.




\subsection{Software politics}


\emph{Chris Browne} \texttt{\aesurl}
 has the
following to say about the kinds of intra-LUG political dynamics that
often crop up (lightly edited and expanded by the HOWTO maintainer):




\subsubsection{People have different feelings about free / open-source software}

GNU/Linux users are a diverse bunch.  As soon as you try to put a lot of
them together, {\itshape some\/} problem issues can arise. Some, who are
nearly political radicals, believe all software, always, should be
"free".  Because Caldera charges quite a lot of money for its
distribution, and doesn't give all profits over to {\itshape (pick favorite
advocacy organisation)\/}, it must be "evil".  Ditto Red Hat or
SUSE.  Keep in mind that all three of these companies have made and
continue to make significant contributions to free / open-source software.

(HOWTO maintainer's note:  The above was a 1998 note, from before
Caldera Systems exited the GNU/Linux business, renamed itself to The SCO Group,
Inc., and launched a major copyright / contract / patent / trade-secret
lawsuit and PR campaign against GNU/Linux users.  My, those times do change.
Still, we're grateful to the Caldera Systems that {\itshape  was \/}, for
its gracious donation of hardware to help Alan Cox develop SMP kernel
support, for funding the development of RPM, and for its extensive past
kernel source contributions and work to combine the GNU/Linux and historical
Unix codebases.)



 
Others may figure they can find some way to highly exploit the
"freeness" of the GNU/Linux platform for fun and profit. Be aware that many
users of the BSD Unix variants consider {\itshape their\/} licences that
{\itshape do\/} permit companies to build "privatised" custom versions of
their kernels and C libraries preferable to the "enforced permanent
freeness" of the GPL as applied to the Linux kernel and GNU libc.  Do
not presume that all people promoting this sort of view are necessarily
greedy leeches.



 
If/when these people gather, disagreements can occur.



 
Leaders should be clear on the following facts:

\begin{itemize}
\item There are a lot of opinions about the GPL and other open-source
licences and how they work -- mostly misinformed.  It is easy to
misunderstand both the GPL and alternative licensing schemes.  Most
attempts at debating same are, at root,  pointless, ritualised symbolic
warfare among people who should know better.  In the rare event that
participants actually aspire to understand the subject, please direct
them to the OSI's "license-discuss" mailing list and the Debian
Project's "debian-legal" mailing list, where substantive analysis is
possible and encouraged.
\item  GNU/Linux benefits from contributions from many places, including
proprietary-software vendors, e.g., in the Linux kernel, X.org, and
gcc.
\item  Proprietary implies neither better nor horrible.
\end{itemize}




 
The main principle can be extended well beyond this; computer "holy
wars" have long been waged over endless battlegrounds, including 
GNU/Linux vs. other Unix variants vs. Microsoft OSes, the "IBM PC" vs.
sundry Motorola 68000-based systems, the 1970s' varied 8-bit systems 
against each other, KDE versus GNOME....



A wise LUG leader will seek to move past such differences, if only
because they're tedious.  LUG leaders ideally therefore will have thick
skins.




\subsubsection{Non-profit organisations and money don't mix terribly well.}

It is important to be careful with finances in any sort of non-profit.
In businesses, which focus on substantive profit, people are not
typically too worried about minor details such as alleged misspending of
{\itshape immaterial\/} sums.  The same cannot be said of non-profit
organisations.  Some people are involved for reasons of principle, and
devote inordinate attention to otherwise minor issues, an example of C.
Northcote Parkinson's 
\emph{Bike Shed Effect} \texttt{\aeturl}
.  LUG business
meetings' potential for wide participation correspondingly expands the
potential for exactly such inordinate attention.



As a result, it is probably preferable for there to {\itshape not\/} be any
LUG membership fee, as that provides a specific thing for which people
can reasonably demand accountability.  Fees not collected can't be
misused -- or squabbled over.



If there {\itshape is\/} a lot of money and/or other substantive property,
the user group must be accountable to members.



Any vital, growing group should have more than one active person.  In
troubled nonprofits, financial information is often tightly held by
someone who will not willingly relinquish monetary control. Ideally,
there should be {\itshape some\/} LUG duty rotation, including duties
involving financial control.



Regular useful financial reports should be made available to those
who wish them. A LUG maintaining official "charitable status"
for tax purposes must file at least annual financial reports
with the local tax authorities, which would represent a minimum
financial disclosure to members.



With the growth of GNU/Linux-based financial software, regular reports are
now quite practical.  With the growth of the Internet, it should even be
possible to publish these on the World-Wide Web.




\subsection{Elections, democracy, and turnover}

Governing your LUG democratically is absolutely vital -- if and
only if you believe it is.  I intend that remark somewhat less cynically
than it probably sounds, as I shall explain.

Tangible stakes at issue in LUG politics tend to be minuscule to the point of
comic opera:  There are typically no real assets. Differences of view 
can be resolved by either engineering around them with technology (the GNU/Linux-ey
solution) or by letting each camp run efforts in parallel. Moreover, even the
most militantly "democratic" LUGs typically field, like clockwork,
exactly as many candidates as there are offices to be filled -- not a
soul more.

It's tempting to mock such exercises as empty posturing, but such
is not (much) my intent.  Rather, I
mention them to point out something more significant:  Attracting and 
retaining key volunteers is vital to the group's success.  Anything that 
makes that happen is good.  It seems likely that the
"democratic" exercise stressed in some groups, substantive or not,
encourages participation, and gives those elected a sense of status,
legitimacy, and involvement.  Those are Good Things.  

Thus, if elections and formal structure help attract key
participants, use them.  If those deter participants, 
lose them.  If door-prizes and garage sales bring people in, do
door-prizes and garage sales.  Participation, as much as software, is
the lifeblood of your LUG. 

The reason I spoke of "key" volunteers, above, is because, inevitably, a
very few people will do almost all of the needed work.  It's just the
way things go, in volunteer groups. An anecdote may help illustrate my
point:  Towards the end of my long tenure as editor and typesetter of
San Francisco PC User Group's 40-page monthly magazine, I was repeatedly
urged to make magazine management more "democratic".  I finally replied
to the club president, "See that guy over there?  That's Ed, one of my
editorial staff.  Ed just proofread twelve articles for the current
issue.  So, I figure he gets twelve votes."  The president and other
club politicos were dismayed by my work-based recasting of their
democratic ideals: Their notion was that each biped should have an equal
say in editorial policy, regardless of ability to typeset or proofread,
or whether they had ever done {\itshape anything\/} to assist magazine
production. Although he looked quite unhappy about doing so, the
president dropped the subject.  I figured that, when it came right down
to it, he'd decide that the club needed people who got work done more
than they needed his brand of "democracy".

But we weren't quite done:  A month or so later, I was introduced to a 
"Publications Committee", who arrived with the intent of doing nothing but 
vote on matters of newsletter policy (i.e., issue "executive" orders to the 
volunteer production staff).  Their first shock came when I listened politely 
to their advice but then applied my editorial judgement as usual.  Much 
worse, though:  I also assigned them work, as part of my staff.  Almost 
all immediately lost interest. (Bossing around other people seemed likely 
to be fun; doing actual work was not.)

The point is that the widespread urge to vote on everything is at best
orthogonal to any desire to perform needed work; at worst, the former
serves as an excuse to compulsively meddle in others' performance
of the latter.

To sum up:  Have all the "democracy" that makes you happy, but watching after
the well-being of your key volunteers is what matters.  (To quote Candide, 
"We must cultivate our garden.")

Last, plan for your replacement:  If your LUG is a college student
group, and must go through a paperwork deathmarch every year to stay
accredited, make sure that and all other vital processes are documented,
so new LUG officers needn't figure everything out from scratch.  Think
of it as a systems-engineering problem:  You're trying to eliminate
single points of failure.

And what works for the guys in the next town may not work for your crowd:
Surprise!  The keys to this puzzle are still being sought.  So, please
experiment, and let me know what works for you, so I can tell others.
Have fun!



