\section{گروه کاربران گنو/لینوکس چیست؟}
%\section{What is a GNU/Linux user group?}
\tfftnt{گروه کاربران لینوکس}{Linux Users Group (LUG)}
یک سازمان خصوصی و اغلب غیرانتفاعی است که با هدف پشتیبانی و آموزش کاربران گنو/لینوکس
به‌خصوص کاربران تازه‌کار تشکیل می‌شود. البته در این انجمن لزوما فعالیت‌ها و اهداف به همین ختم نمی‌شود.
در لاگ علاقمندان به گنو/لینوکس و نرم‌افزار آزاد با همدیگر آشنا می‌شوند و می‌توانند با اهداف مختلفی از جمله
اشتراک تجربه و یا حتی خوشگذرانی باهمدیگر وقت بگذرانند. لاگ شباهت بسیاری به یک باشگاه طرفداری دارد.
پس اگر شما هم علاقمند به گنو/لینوکس هستید می‌توانید با دنبال کردن لاگ محل خودتان دوستان جدیدی
پیدا کنید و از تجربه دیگر علاقمندان نیز استفاده کنید و یا تجربه خود را با دیگران به‌اشتراک بگذارید.

\subsection{گنو/لینوکس چیست؟}
%\subsection{What is GNU/Linux?}
لاگ نقش بسیار مهمی در جنبش گنو/لینوکس دارد. اگر میخواهید بدانید این نقش به چه شکل است
و چرا وجود لاگ در این جنبش مهم است، باید اول از همه گنو/لینوکس را بشناسیم تا بعد درمورد
نقش لاگ در آن گفت‌وگو کنیم.
%To fully appreciate LUGs' (Linux User Groups') role in the GNU/Linux
%movement, it helps to understand what makes GNU/Linux unique.

گنو/لینوکس به عنوان یک سیستم عامل،بسیار قدرتمند است. اما به عنوان یک
{\bfseries ایده}
راجع به توسعه نرم‌افزار حتی قدرتمند تر هم می‌تواند باشد. گنو/لینوکس یک سیستم‌عامل
{\bfseries آزاد}
تحت لایسنس
\tfftnt{گنو}{GNU}
(و یا دیگر پروانه‌های متن‌باز/نرم‌افزار آزاد که گاها شامل تعدادی پکیج‌های انحصاری نیز هستند) 
نگهداری می‌شود. بنابراین سورس کد، رایگان و برای همیشه در دسترس همگان خواهد بود.
این پروژه توسط گروهی بدون ساختار از برنامه‌نویسان سرتاسر دنیا زیر نظر راهبری فنی
لینوس توروالدز و دیگر توسعه‌دهنده‌های کلیدی این پروژه نگهداری می‌شود.
گنو/لینوکس همانند یک جنبش فاقد ساختار مرکزی، بروکراسی و یا نهاد دیگری برای امور خود می‌باشد.
درحالی که این مزایایی دارد، چالش‌هایی برای اختصاص منابع انسانی، حمایت موثر، روابط عمومی،
تعلیم کاربران و آموزش را به‌همراه دارد.
%GNU/Linux as an operating system is powerful -- but GNU/Linux as an
%{\itshape {\bfseries idea}\/} about software development is even more so. GNU/Linux
%is a {\bfseries free} operating system: It's licensed under the GNU General
%Public Licence (and other open source / free software licences -- though 
%proprietary application software is sometimes also included in
%particular packagings). Thus, source code is freely available in
%perpetuity to anyone. It's maintained by a unstructured group of
%programmers world-wide, under technical direction from Linus Torvalds
%and other key developers. GNU/Linux as a movement has no central
%structure, bureaucracy, or other entity to direct its affairs. While
%this situation has advantages, it poses challenges for allocation of
%human resources, effective advocacy, public relations, user education,
%and training.

\begin{note}
این راهنما به بنیاد نرم‌افزار آزاد
\fftnt{پروژه گنو}{http://www.gnu.org/gnu/gnu-history.html}
به‌عنوان قدرت محرک اساسی در ایجاد و پیشبرد یک سیستم یک‌پارچه آزاد و یا همان متن‌باز اعتبار می‌دهد.
بنابراین به «توزیع‌ها» شامل سیستم عامل گنو به همراه کرنل لینوکس به عنوان «گنو/لینوکس» ارجاع می‌دهد.
البته این رایج نیست که به‌جای تمام توزیع‌ها از اصطلاح گنو/لینوکس استفاده شود و این خواسته
\tfftnt{اف‌اس‌اف}{Free Software Foundation}
برای اعتباردهی به طور گسترده مورد احترام قرار نمی‌گیرد. اما عادلانه بودن این خواسته کاملا واضح است.
\end{note}

%(This HOWTO credits the Free Software Foundation's
%\emph{GNU Project} \texttt{\abrurl}
% as the crucial motive force behind creating and furthering a free 
%aka open source integrated system.  Thus, it refers to «distributions» 
%comprising the GNU operating system atop the Linux kernel as «GNU/Linux".
%Yes, the term is awkward, and FSF's request for credit isn't widely 
%honoured; but the justice of FSF's claim is obvious.)

\begin{note}
مدیر این راهنما می‌داند که با وجود عادلانه بودن این مسئله دنیا قرار نیست آن را بپذیرد.
\end{note}
%(This HOWTO's maintainer is also fully aware that the world at large
%will never adopt this usage, justice notwithstanding.  If it seems 
%mannered, please indulge him, and respect the gesture.)

\subsection{چرا گنو/لینوکس منحصربه‌فرد خطاب می‌شود؟}
%\subsection{How is GNU/Linux unique?}

ساختار آزاد و بدون هسته مرکزی گنو/لینوکس بعید است تغییر کند زیرا ویژگی مفیدی است.
مردم آزاد هستند که به اراده خود وارد سیستم شده و به اراده خود از آن خارج شوند.
به‌همین دلیل این سیستم ‫به‌خوبی جواب داده و کار می‌کند:
{\bfseries
برنامه‌نویس‌های آزاد، برنامه‌نویس‌های خوشحال، و برنامه نویس‌های خوشحال برنامه‌نویس‌های کارآمد هستند.
\LTRfootnote{Free programmers are happy programmers are effective programmers}
}
%GNU/Linux's loose structure .  That's a good thing:
%It works precisely because people are free to come and go as they
%please: {\bfseries Free programmers are happy programmers are effective
%programmers}.

هرچند این عدم ساختار می‌تواند کاربران جدید را گیج کند.
کاربر جدید برای پشتیبانی، آموزش و تعلیم با چه کسی باید تماس بگیرد؟
از کجا بداند کدام توزیع گنو/لینوکس برای او مناسب است؟
%However, this loose structure can disorient the new user: Whom
%does she call for support, training, or education? How does she know
%what GNU/Linux is suitable for?

تا حدودی لاگ‌ها به این سوالات پاسخ می‌دهند و به همین دلیل است که لاگ‌ها برای این جنبش حیاتی بوده اند.
برای همین محله، روستا یا کلان‌شهر شما هیچ «نمایندگی رسمی» از شرکت لینوکس ندارد.
لاگ‌ها نقش‌هایی مشابه یک دفتر نمایندگی رسمی برای یک شرکت بزرگ چندملیتی را ایفا می‌کنند.
%In part, LUGs provide the answers, which is why LUGs have been vital to
%the movement: Because your town, village, or metropolis sports no
%Linux Corporation «regional office", the LUG takes on many of the same
%roles a regional office does for a large multi-national corporation.

گنو/لینوکس به‌خاطر فقدان ساختار‌های مرکزی یا بروکراسی برای اختصاص دادن منابع،
آموزش دادن کاربران و پشتیبانی محصولات خود غیرعادی و غیرمعمول است. تمام این کارها
از طرق گوناگونی انجام می‌شوند. مانند اینترنت، راهنماها،
\fftnt{نمایندگی فروش}{https://en.wikipedia.org/wiki/Value-added_reseller}
شرکت‌های پشتیبانی، کالج‌ها و دانشگاه‌ها. بااینحال به طور رو به رشدی در خیلی از جاهای
سرتاسر دنیا تمام این کارها به‌جای نمایندگی رسمی، توسط لاگ صورت می‌گیرند.
%GNU/Linux is unusual in neither having nor being burdened by central
%structures or bureaucracies to allocate its resources, train its users,
%and support its products. These jobs get done through diverse means: the
%Internet, consultants, VARs, support companies, colleges, and
%universities. However, increasingly, in many places around the globe,
%they are done by a LUG.

\subsection{گروه کاربری چیست؟}
%\subsection{What is a user group?}

گروه‌های کاربری کامپیوتر جدید نیستند. در حقیقت آنها در تاریخچه کامپیوترهای شخصی
بسیار مهم بودند. میکروکامپیوتر‌ها به‌طور کلی برای برآورده کردن تقاضا برای دسترسی شخصی
و مقرون به‌صرفه به منابع کامپیوتری از گروه‌های کاربری الکترونیک، رادیو آماتور و سایر گروه‌های
علاقمند به وجود آمدند. نهایتا غول‌هایی مانند
\tfftnt{آی‌بی‌ام}{IBM}،
کامپیوتر را به‌عنوان محصولی خوب و سودآور
کشف کردند. اما انگیزه ابتدایی آن در ابتدا از مردم،‌به‌ویژه تلاش‌های پیشگامانه
گروه‌های کامپیوتری \lr{SHARE} (۱۹۵۵-اکنون) و \lr{DECUS} (۱۹۶۱-۲۰۰۸) شروع شد.
%Computer user groups are not new. In
%fact, they were central to the personal computer's history:
%Microcomputers arose in large part to satisfy demand for affordable,
%personal access to computing resources from electronics, ham radio, and
%other hobbyist user groups.  Giants like IBM eventually discovered the
%PC to be a good and profitable thing, but initial impetus came from the
%grassroots, leading to groundbreaking efforts like SHARE (1955-present) 
%and DECUS (1961-2008).

در آمریکا گروه‌های کاربری طی زمان دچار تغییراتی نه‌چندان مثبتی شده اند.
منحل شدن انجمن کامپیوتر بوستون
\LTRfootnote{\url{https://en.wikipedia.org/wiki/Boston_Computer_Society}}
و مشکلات مالی آن، بزرگترین گروه کاربری به‌خوبی این تغییرات منفی را نشان می‌دهد.
در سرتاسر آمریکا اکثر گروه‌های کاربری کامپیوتر کاهش عضو داشته اند.
گروه‌های کاربری آمریکایی در روزهای اوج خود خبرنامه‌ها را تولید می‌کردند،
کتابخانه‌های اشتراک‌افزارها و دیسکت را نگه داری می‌کردند،
دورهمی و رویدادهای اجتماعی برگزار می‌کردند و
بعضی اوقات حتی سیستم‌های
\fftnt{بی‌بی‌اس}{https://en.wikipedia.org/wiki/Bulletin_board_system}
اجرا می‌کردند. با ظهور اینترنت، خیلی از سرویس‌هایی که گروه‌های کاربری
برای اولین‌بار فراهم کرده بودند به سرویس‌هایی مانند
\fftnt{کامپیوسرو}{https://en.wikipedia.org/wiki/CompuServe}
و وب مهاجرت کردند.
%In the USA, user groups have changed -- many for the worse --
%with the times. The financial woes and dissolution of the largest user
%group ever, the Boston Computer Society, were well-reported; but, all
%over the USA, most PC user groups have seen memberships decline.
%American user groups in their heyday produced newsletters, maintained
%shareware and diskette libraries, held meetings and social events, and,
%sometimes, even ran electronic bulletin board systems (BBSes). With the
%advent of the Internet, however, many services that user groups once
%provided migrated to things like CompuServe and the Web.

ظهور گنو/لینوکس با کشف اینترنت مصادف بود و به آن شدت بخشید.
گنو/لینوکس به‌اندازه اینترنت رشد می‌کرد و محبوب می‌شد.
اینترنت کاربران، توسعه‌دهنده‌ها و وندورهای جدیدی را آورد.
سپس همان فشاری که تعداد کاربران گروه‌های کاربری سنتی را کاهش می‌داد،
گنو/لینوکس را رشد داد و گروه‌های جدیدی را که منحصرا به گنو/لینوکس مرتبط بودند
را الهام بخشید.
%GNU/Linux's rise, however, coincided with and was intensified by the
%general public «discovering» the Internet. As the Internet grew more
%popular, so did GNU/Linux: The Internet brought new users,
%developers, and vendors. So, the same force that sent traditional user
%groups into decline propelled GNU/Linux forward, and inspired new groups 
%concerned exclusively with it.

برای اشاره به فقط یکی از تفاوت‌های لاگ با گروه‌های کاربری سنتی می‌توان گفت:
گروه‌های سنتی در جلسات به‌ناچار باید بر روی نرم‌افزارهایی که کاربران بین خود
توزیع و پخش می‌کردند نظارت می‌کردند. با اینکه کپی برداری غیرقانونی نرم‌افزارهای
انحصاری و محدود شده قطعا اتفاق می‌افتاد و جلوگیری از آن غیر ممکن بود، به‌طور رسمی
تلاش بر این بود که از این اتفاق جلوگیری شود.
اما در جلسات لاگ به‌طور کلی این طرز فکر اعمال نمی‌شود. کپی نامحدود گنو/لینوکس نه تنها ممنوع نیست،
بلکه باید از اهداف اصلی لاگ نیز باشد. طبق شواهد تجربی، گروه‌های کاربری سنتی برخلاف گنو/لینوکس
نمی‌توانند به‌طور قانونی نرم‌افزار‌ها را کپی و توزیع کنند و از این جهت با گنو/لینوکس ناسازگار اند.

%To give just one indication of how LUGs differ from traditional 
%user groups: Traditional groups must closely 
%monitor what software users redistribute at meetings.
%While illegal copying of restricted proprietary software certainly
%occurred, it was officially discouraged -- for good reason.
%At LUG meetings, however, that entire mindset simply does not apply:
%Far from being forbidden, unrestricted copying of GNU/Linux
%should be among a LUG's primary goals.  In fact, there is anecdotal
%evidence of traditional user groups having difficulty adapting to
%GNU/Linux's ability to be lawfully copied at will.

\begin{caveat}
برخی از توزیع‌های گنو/لینوکس را با بسته‌های نرم‌افزاری انحصاری
\tfftnt{باندل}{Bundle}
می‌کنند که این امر
دیگر اجازه بازنشر عمومی آنها را به دارنده محصول نمی‌دهد. اگر شک دارید شرایط لاینس را بررسی کنید.
قانونا به هر نحوی در هرجای لاگ باید از دعوت یا درخواست کپی نرم‌افزارهای انحصاری توزیع‌های محدودشده
 به‌طور جدی جلوگیری شده و برای تمام انجمن‌های آنلاین گروه‌های کاربری گنو/لینوکس
\tfftnt{خارج از موضوع}{Off topic}
اعلام شود.
\end{caveat}
%(Caveat:  A few distributions bundle GNU/Linux with proprietary
%software packages whose terms don't permit public redistribution.
%Check licence terms, if in doubt.  Offers or requests to copy 
%distribution-restricted proprietary software of any sort should be
%heavily discouraged anywhere in LUGs, and declared off-topic for all 
%GNU/Linux user group on-line forums, for legal reasons.)




\subsection{جلوگیری از فرسودگی و کاهش کاربران}
%\subsection{Avoiding Burnout and Decline}

از حدود سال ۲۰۰۳ در کشورهای توسعه یافته لاگ‌ها همانند گروه‌های کاربری سنتی
کاهش مشابهی را در تعداد کاربران و فعالیت‌ها مشاهده کردند. علل این اتفاق قابل بحث است
و ممکن است شامل موارد زیر باشد:

%Since around 2003, LUGs in developed countries have seen a decline
%similar to that of traditional user groups.  The causes can be debated,
%and might include:

\begin{itemize}
\item 
گنو/لینوکس به قدری موفق است که خیلی وقت‌ها به‌عنوان یک زیرساخت شناخته می‌شود تا چیزی جدید و جالب.
%\item GNU/Linux being so successful that it's often perceived as
%infrastructure rather than as something new and interesting.

\item
لاگ در هیاهو و سروصدای رسانه‌های اجتماعی دیگر گم می‌شود.
%\item LUGs getting lost in the noise of social media.
\item
بنیانگذاران و اعضای قدیمی لاگ‌ها به‌سمت علایق دیگرشان می‌روند و لاگ‌ها را ترک می‌کنند.
%\item Early adopters critical to making LUGs function moving
%on to other interests.

\item 
\fftnt{میت آپ}{http://linuxmafia.com/faq/Essays/meetup.html}
استعداد و انرژی موجود علاقمندان را جذب کرده و توجه‌ها را از روی لاگ‌ها بر می‌دارد.
%\item 
%\emph{Meetup.com} \texttt{http://linuxmafia.com/faq/Essays/meetup.html}
%, with its strong inward-facing focus, sucking
%away available talent and energy, and making LUGs less noticeable.

\item
نصب و استفاده از گنو/لینوکس به مراتب آسان تر شده و این امر باعث شده
توجه‌ها بیشتر به‌سمت موضوعات خاص‌تر منتقل شوند و توسط جوامع تکنیکال‌تر
و تخصصی‌تر‌ (دواپس، بیوانفورماتیک، رایانش ابری، رایانش نهفته و موارد دیگر)
موردبحث قرار گیرند.
%\item GNU/Linux becoming so much easier to install
%and use that focus has shifted to more-specialised topics better served
%by more-specialised technical communities (DevOps, bioinformatics,
%cloud computing, embedded computing, and many others).


\item
راهبران لاگ ‫به‌خوبی دانش را به نسل بعدی راهبران منتقل نمی‌کنند و زمانی که
کناره‌گیری می‌کنند هیچ‌کس برای تصاحب این صمت آماده نیست.
%\item LUG leaders poorly managing a generational transition, 
%leaving nobody ready to take over as they bow out.

\item
به‌طور کلی فراگیری بیشتر اینترنت و به‌خصوص سایت‌های همکاری مبتنی بر شهرت
\LTRfootnote{reputation-based collaborative sites}
مانند
\fftnt{استک اکسچنج}{http://unix.stackexchange.com/},
\fftnt{استک اورفلو}{http://stackoverflow.com/},
\fftnt{داک تایپ}{http://doctype.com/},
\fftnt{کد پراجکت}{http://www.codeproject.com/}
و
\fftnt{سرور فالت}{http://serverfault.com/}
لازم‌ به ذکر است که با افزایش مهارت سرچ کاربران نیاز به لاگ‌ها کمتر شده
و فعالیت اصلی کاربران به‌سمت سایت‌های
\fftnt{نرم‌‌افزار به‌عنوان سرویس}{https://en.wikipedia.org/wiki/Software_as_a_service}
می‌رود. لاگ‌ها نمی‌توانند با مقیاس و شبکه‌ای که این سایت‌ها دارند رقابت کنند.

%\item Greater ubiquity of the Internet generally, and
%specifically reputation-based collaborative sites like 
%
%\emph{StackExchange} \texttt{http://unix.stackexchange.com/},
%\emph{StackOverflow} \texttt{http://stackoverflow.com/},
%\emph{Doctype} \texttt{http://doctype.com/}, 
%\emph{Codeproject} \texttt{http://www.codeproject.com/}, and 
%\emph{Serverfault} \texttt{http://serverfault.com/}, not to mention 
%users becoming skilled at Web-searching, making LUGs far less
%pragmatically necessary, and the main action involving SaaS sites having 
%site scale and network effects with which LUGs cannot compete.
\end{itemize}


چند نکته آزموده شده برای جلوگیری از افول لاگ:
%A few time-tested tips for averting LUG flameout:

\begin{itemize}

\item
اتوماسیون دوست شماست. هرکاری که می‌تواند با اسکریپت انجام شود،
باید اسکریپت شود.
%\item Automation is your friend.  Any task that can be scripted, 
%should be scripted.

\item
تمام سیستم لاگ خود را از جنبه‌های تکنیکال و اجتماعی برای تک نقطه شکست
(\ftnt{SPoFs}{https://en.wikipedia.org/wiki/Single_point_of_failure})
بررسی کنید. تلاش کنید و مطمئن شوید که درصورت بروز مشکل راه بازگشتی وجود خواهد داشت.
از همه‌چیز بکاپ بگیرید و آنها را تست کنید.
{\bfseries
مطمئن شوید امور مهم لاگ فقط وابسته به یک نفر نیست.
(در لاگ نباید قدرت به‌طور انحصاری در دست یکنفر یا یک گروه خاص قرار گیرد)
}
%\item Check all your LUG's systems, both technical and social,
%for single points of failure (SPoFs). Keep trying to make sure there are
%fallbacks if anything or anyone fails. Do (and test) backups.  Ensure that nothing
%important can be done by only one person.

\item
مراقب انجام بیش از حد و افراطی فعالیت در لاگ یا یک عضو خاص لاگ باشید.
فعالیت کمتر و سبک‌تر بهتر از خطر خروج مردم از لاگ می‌باشد.
پس بهتر است طوری لاگ را مدیریت کنید که روی اعضای لاگ فشار وارد نشود.
%\item Beware of your LUG, or any individual in it, committing 
%to carrying out too much work, or with too great frequency.  It's 
%better for a LUG to do less, or have its functions occur less often,
%than risk people wearing out and leaving.

\item
به‌یاد داشته باشید اگر به مردم خوش نگذرد، آنها برای مدت طولانی
نخواهند ماند و از مشارکت دست خواهند کشید. درصورتی که گروه به‌جای تکنیکال شدن
بیشتر اجتماعی شد، عصبانی نشوید. این رویکردی سالم است.
%\item Remember that if people aren't having fun, they won't
%continue for long.  E.g., if your group becomes less technical and 
%more social, don't fret. It's probably a healthy thing.

\item
با دقت از دارایی‌های مهم خود نگهداری کنید. برای مثال مالکیت دامنه سایت لاگ،
مکان برگزاری جلسات و روابطی که با شرکت‌ها و کمپانی‌ها دارید.
افراد مشکل‌سازی که گاها به‌سمت لاگ کشیده می‌شوند را دور نگه‌دارید.
حتی اگر شما چنین فردی را دیدید، و برای پس‌گرفتن دامنه‌تان از کسی که قصد آسیب زدن به
لاگ را دارد درگیر شدید (که اتفاق خواهد افتاد)، موجب دور شدن افراد مهم و کلیدی از لاگ می‌شود.
%\item Carefully guard your significant assets, such as domain
%ownership, difficult-to-acquire meeting venues, and the names of key
%corporate contacts, and keep them away from problematic people sometimes
%drawn to LUGs.  Even if you, say, wrestle your domain away from someone
%who's suddenly decided to destroy the LUG (which does happen), the
%strife will drive away key people.
\end{itemize}

\subsection{خلاصه}
%\subsection{Summary}
برای رشد جنبش گنو/لینوکس، درمیان دیگر الزامات، لاگ‌ها هم باید گسترش یافته و موفق شوند.
به‌دلیل ماهیت غیرمعمول گنو/لینوکس، لاگ‌ها باید نقشی مشابه با «نمایندگی رسمی"
شرکت‌های بزرگ کامپیوتری مانند
\tfftnt{آی‌بی‌ام}{IBM}، \tfftnt{مایکروسافت}{Microsoft} و \tfftnt{سان}{Sun}
ایفا کند. لاگ وظیفه و پتانسیل آموزش، پشتیبانی، تعلیم کاربران، هماهنگی مشاوران،
طرفداری گنو/لینوکس به‌عنوان یک راهکار کامپیوتری و یا حتی رابطی برای انتشار اخبار محلی را برعهده دارد.
%For the GNU/Linux movement to grow, among other requirements,
%LUGs must proliferate and succeed. Because of GNU/Linux's
%unusual nature, LUGs must provide some of the same functions a «regional
%office» provides for large computer corporations like IBM, Microsoft,
%and Sun. LUGs can and must train, support, and educate users,
%coordinate consultants, advocate GNU/Linux as a computing solution,
%and even serve as liaison to local news outlets.

